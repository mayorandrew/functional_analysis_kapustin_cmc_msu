\documentclass[12pt,a4paper, titlepage]{article}
\usepackage[utf8]{inputenc}
\usepackage{amsmath}
\usepackage{amsfonts}
\usepackage{amssymb}
\usepackage{cmap}
\usepackage[russian]{babel}

\renewcommand{\Im}{\mathop{\mathrm{Im}}\nolimits}
\renewcommand{\Re}{\mathop{\mathrm{Re}}\nolimits}

\begin{document}

\author{Косовец Дмитрий \and Петрушкина Анастасия \and Ручкин Дмитрий \\
под редакцией Иванова Олега и Талгата Даулбаева \thanks{Отдельное спасибо тем, кто давал полезные советы и помогал исправлять ошибки: Гусейнову Алексею, Деревенцу Егору, Картавцу Евгению, Козлову Кириллу, Кунцьо Степану, Маслову Дмитрию, Соловьёву Антону и др.}}
\title{ВМиК, 4-й курс, 3-й поток \\ Конспекты курса по функциональному анализу \thanks{Версия 0.4 --- полный курс (лекции 1-13, параграфы 1-14) }}
\date{2008 год \\ (ред. 2014 год)}


\maketitle

%\pagebreak

\textbf{Литература по курсу:}
\begin{enumerate}
\item В.А. Ильин, Э.Г. Позняк "Основы математического анализа, часть 2", М., 1973.
\item А.Н. Колмогоров, С.В. Фомин "Элементы теории функций и функционального анализа", М., 1976.
\item Л.А. Люстерник, В.И. Соболев "Элементы функционального анализа", М., 1951
\end{enumerate}

\parindent=0cm

\pagebreak

\section*{\S 1. Открытые и замкнутые множества на прямой}

 Рассматриваем пространство действительных чисел - $\mathbb{R}$.\\
Множество $E\subset\mathbb{R}$.\\

Операции над множествами:
\begin{itemize}
\item Объединение: $F = E_1 \cup E_2$;
\item Пересечение: $F = E_1 \cap E_2$;
\item Разность: $F = E_1 \setminus E_2$;
\item Симметрическая разность: $F = E_1 \Delta E_2 = (E_1 \setminus E_2) \cup (E_2 \setminus E_1)$
\item Дополнение: $ \complement E = \mathbb{R} \setminus E$.
\end{itemize}\

\textit{Окрестностью точки} называется любой интервал, содержащий эту точку.\\

Точка $x$ называется \textit{предельной точкой} множества $E$, если в любой окрестности содержится хотя бы одна точка множества $E$, отличная от точки $x$.
Если точка $x$ множества $E$ не является его предельной точкой, то она называется \textit{изолированной точкой} множества $E$.\\

Множество всех предельных точек $E$ называется его \textit{производным множеством} и обозначается $E'$.\\

Возможны такие ситуации:
\begin{itemize}
\item $E' \subset E$, тогда $E$ --- \textit{замкнутое} множество;
\item $E \subset E'$, тогда $E$ --- \textit{плотное в себе} множество;
\item $E = E'$, тогда $E$ --- \textit{совершенное} множество.
\end{itemize}

\textbf{Определение.} Множество $\bar E = E \cup E'$ называется \textit{замыканием} множества $E$.\\

\textbf{Примеры:} \medskip
\begin{enumerate}
\item  $ E = \bigcup \limits_{n=1}^{\infty}{\left\lbrace \dfrac 1 n \right\rbrace},\qquad E' = \{ 0 \},$ \smallskip\\
$E$ --- не замкнуто, не плотное в себе, никакое;
\medskip
\item $E = \bigcup \limits_{n=1}^{\infty}{\left\lbrace \dfrac 1 n \right\rbrace} \cup \{ 0 \},\qquad E'= \{ 0 \}$ \smallskip\\
		$E$ --- замкнуто, не плотное в себе;
\medskip
\item $E = (a, b),\qquad E' = [a, b]$ \smallskip\\
		$E$ --- не замкнуто, плотное в себе;
\medskip
\item $E = [a, b],\qquad E' = [a, b]$ \smallskip\\
		$E$ --- совершенное;
\medskip
\item $E = \mathbb{Q},\qquad E' = \mathbb{R}$ \smallskip\\
		$E$ --- не замкнуто, плотное в себе;
\medskip
\item $E = \mathbb{R},\qquad E' = \mathbb{R}$ \smallskip\\
		$E$ --- совершенное;
\medskip
\item $E = \varnothing,\qquad E' = \varnothing$ \smallskip\\
		$E$ --- совершенное;
\end{enumerate}

Для любого множества $E$ производное множество $E'$ и замыкание $\bar E$\\ всегда будут являться замкнутыми. 
Объединение конечного числа замкнутых множеств --- замкнутое множество, бесконечного числа --- не всегда. Например:
$$
E_n = \left[\frac 1 n ; 1\right]; \quad E = \bigcup^{\infty}_{n=1} E_n = (0, 1].
$$

Точка X множества E называется \textit{внутренней}, если она принадлежит E вместе с некоторой своей окрестностью.

$\mathrm{int}E$ --- множество всех внутренних точек $E$.

Множество E называется \textit{открытым}, если все его точки внутренние.\\

Пересечение конечного числа открытых множеств является открытым множеством.

Пересечение бесконечного числа открытых множеств, вообще говоря, открытым не является:
$$
E_n = \left(- \frac 1 n ; \frac 1 n \right); \quad E = \bigcap^{\infty}_{n=1} E_n = \{0\}.
$$

\textbf{Утверждение.} $E$ - замкнутое множество $\Rightarrow \complement E$ --- открытое множество.\\
\textbf{Доказательство.}
Возьмём произвольную точку $x \in \complement E$. $E$ --- замкнутое множество $\Rightarrow$ оно содержит все свои предельные точки. $E \not\ni x \Rightarrow x$ не является предельной точкой $E$. Это означает, что существует окрестность $v(x)$, целиком принадлежащая $\complement E$. Таким образом, точка $x$ является внутренней точкой $\complement E$. В силу произвольности выбора $x$ в $\complement E$ это множество является открытым. $\quad \Box$\\ \\

\textbf{Утверждение.} $E$ --- открытое множество $\Rightarrow$ $\complement E$ --- замкнутое множество.\\
\textbf{Доказательство.}
Предположим, что это не так. Тогда существует точка $x_0$, предельная для $\complement E$, но не принадлежащая ему. Следовательно, $x_0 \in E$. $E$ --- открытое множество, $x_0 \in E \Rightarrow$ существует окрестность $v(x_0)$, целиком принадлежащая $E$. Это противоречит тому, что $x_0$ --- предельная точка $\complement E$. Значит, наше предположение неверно $\Rightarrow \complement E$ --- замкнутое множество. $\quad \Box$\\ \\

Объединение любого числа открытых множеств является открытым множеством.

Пересечение любого числа замкнутых множеств является замкнутым множеством.\\

$E = \bigcap_{\alpha} E_{\alpha} \quad \Rightarrow \quad \complement E = \cup_{\alpha} \complement E_{\alpha}$\\

\textbf{Утверждение.} Если A --- открытое множество, а B --- замкнутое, то $A \setminus B$ --- открытое множество, $B \setminus A$ --- замкнутое множество.\\
\textbf{Доказательство.} $A \setminus B = A \cap \complement B$. $\quad \Box$\\ \\

\textbf{Теорема 1.}
Любое открытое множество на прямой может быть представлено в виде конечного или счетного объединения попарно непересекающихся интервалов.\\
$E = \bigcup\limits_{n=1}^{\infty}I_n$. \\
\textbf{Доказательство.} \\
Возьмем произвольную точку $x \in E$. \\
Рассмотрим $I(x)$ --- объединение всех интервалов, содержащих $x$ и содержащихся в $E$.\\
$I(x)$ --- открытое множество. \\
$a = \inf I(x)$, $b = \sup I(x)$. \\
Так как $E$ --- открытое, то $x \in int E$. Значит, $a < x < b$. \\

Докажем, что $I(x) = (a, b)$ \\
Рассмотрим любую точку $y \in (a, b)$. Не ограничивая общности, положим $a < y < x$.
По определению точной нижней грани, $\exists y'$: $a < y' < y < x$.\\
По определению точной нижней грани $\exists (c, d) \subset I(x)$, такое что $y' \in (c, d)$. \\
Из $y' \in (c, d)$ и $x \in (c,d)$ следует $y \in (c,d)$. Значит, $y \in I(x)$. В силу произвольности выбора $y$, $I(x) = (a, b)$.\\

Тогда для любого $y \in I(x)$ выполняется $I(x) = (a, b) \subset I(y)$. Значит, $x \in I(y)$, из чего следует, что $I(y) = I(x)$ для любого $y \in I(x)$.

Тогда $I(x_1)$ и $I(x_2)$ либо не имеют общих точек, либо совпадают.
$\quad \Box$\\ \\

\textbf{Следствие 1.}
Любое замкнутое множество на прямой получается удалением из прямой конечного или счетного числа интервалов.\\

\textbf{Следствие 2.}
Любое совершенное множество на прямой получается удалением из прямой конечного или счетного числа попарно непересекающихся интервалов, концы которых не совпадают.\\

\textbf{Пример (совершенное Канторово множество).}

Рассмотрим сегмент [0, 1]:
$$
G = \left(\frac13; \frac23 \right) \cup \left(\frac19; \frac29 \right) \cup \left(\frac79; \frac89 \right) \cup \dots
$$
$$
K = [0; 1] \setminus G
$$
$$
\frac13 + 2 * \frac19 + 4 * \frac{1}{27} + \dots = \frac13 \sum^{\infty}_{k=0} \left(\frac23 \right)^k = 1
$$

Любое число $x$ из отрезка $[0; 1]$ может быть представлено в троичной системе следующим образом:
$$
x = \frac{a_1}{3} + \frac{a_2}{3^2} + \dots + \frac{a_n}{3^n} + \dots ,
$$
где числа $a_i$ могут принимать значения $0$, $1$ и $2$. У числа может быть не одно такое представление ($0.1222..._3 = 0.2_3$). Заметим, что у чисел из множества $K$ существует хотя бы одно троичное представление, в котором числа $a_i \neq 1 \ \forall i$. Ведь в множество $K$ не входят интервалы $\left(\dfrac13; \dfrac23\right)$, $\left\lbrace \left(\dfrac19; \dfrac29\right), (\dfrac79; \dfrac89)\right\rbrace$ и так далее, числа которых как раз соответствуют троичным представлениям, где часть коэффициентов равна единице. А для некоторых "граничных" чисел, входящих в $K$ и имеющих троичное представление с одним единичным коэффициентом (например, $\dfrac13$), существуют соответствующие троичные представления, где нет единичных коэффициентов:
$$
\frac13 = \frac13 + \frac{0}{3^2} + \dots + \frac{0}{3^n} + \dots = \frac03 + \frac{2}{3^2} + \dots + \frac{2}{3^n} + \dots
$$
Таким образом, каждой точке множества $K$ можно поставить в соответствие последовательность $a_1, a_2, \dots, a_n, \dots$, где $a_i = \lbrace 0; 2 \rbrace$. Совокупность таких последовательностей образует множество мощности континуума $\Rightarrow$ множество $K$ имеет мощность континуума, хотя длина его равна 0. Кроме того, оно замкнуто и является совершенным (по следствию 2).\\

Множество $K$ называется \textit{канторовым множеством}.


\section*{\S 2. Измеримые множества}
$$
\Delta = (a, b), \; |\Delta| = b - a
$$
Считаем, что допустимы не только $a,b \in \mathbb{R}$, но и $a = -\infty$, $b = \infty$. В большинстве теорем это не влияет не доказательство, иначе такие случаи будут рассматриваться отдельно.

\textbf{Определение 1.} \textit{Покрытием множества} $E$ \textit{на прямой} называется конечная или счетная система интервалов, объединение которых содержит $E$.
$$
s(E): E \subset \bigcup^\infty_{n=1} \Delta_n
$$
\textit{Длина покрытия} $s(E)$:
$$ \sigma(s) = \sum^\infty_{n=1} |\Delta_n| $$
\textit{Внешняя мера} множества $E$:
$$
|E|^* = \inf_{s(E)} \sigma(s)
$$

Внешняя мера не обладает ни конечной, ни счётной аддитивностью.\\

Свойства внешней меры:
\begin{enumerate}
\item $E_1 \subset E_2 => |E_1|^* \leqslant |E_2|^*.$\\
Этот факт следует из того, что любое покрытие $E_2$ будет одновременно являться покрытием и для $E_1$.
\item $E = \bigcup\limits^\infty_{n=1} E_n \Rightarrow |E|^* \leqslant \sum\limits^\infty_{n=1} |E_n|^*.$ \smallskip\\
\textbf{Доказательство.}
Если мера $|E|^* = \infty$ или $|E_n|^* = \infty$, то свойство очевидно.
Фиксируем произвольное $\varepsilon > 0$. По определению меры как точной нижней грани, для любого номера $n$ найдётся покрытие $s_n(E_n)$ множества $E_n$ системой интервалов $\lbrace \Delta^n_k \rbrace$ такое, что 
$$
\sum^\infty_{k=1} |\Delta^n_k| < |E_n|^* + \frac{\varepsilon}{2^n}.
$$
Рассмотрим $s = \bigcup\limits^\infty_{n=1} s_n(E_n)$. $s$ является покрытием множества $E$, поэтому 
$$
|E|^* \leqslant \sigma(s) \leqslant \sum^\infty_{n=1} \sum^\infty_{k=1} |\Delta^n_k| < \sum^\infty_{n=1} (|E_n|^* + \frac{\varepsilon}{2^n}) = \sum^\infty_{n=1} |E_n|^* + \varepsilon.
$$
Устремив $\varepsilon$ к 0, получим требуемое неравенство. $\quad \Box$\\

Введём понятие \textit{расстояния между множествами}:
$$
\rho(E_1, E_2) \stackrel{def}{=} \inf_{\substack{x \in E_1 \\ y \in E_2}} \rho(x, y) = |x - y|.
$$

\item $\rho(E_1, E_2) > 0 => |E_1 \cup E_2|^* = |E_1|^* + |E_2|^*$\smallskip\\
\textbf{Доказательство.}
$d = \rho(E_1, E_2)$.
Из определения меры как точной нижней грани следует, что
$$
\forall \varepsilon > 0 \quad \exists s(E_1 \cup E_2) = \lbrace \Delta_n \rbrace :
\quad \sum^\infty_{n=1} |\Delta_n| < |E_1 \cup E_2|^* + \frac{\varepsilon}{2}
$$
Разобьём каждый интервал покрытия $s(E_1 \cup E_2)$ на интервалы длины, меньшей $\dfrac{d}{2}$, а концы этих новых интервалов ("точки соприкосновения"), в свою очередь, покроем интервалами, общая сумма длин которых меньше $\dfrac{\varepsilon}{2}$. Таким образом,
$$
\forall \varepsilon > 0 \quad \exists \widetilde{s}(E_1 \cup E_2) = \lbrace \widetilde{\Delta}_n \rbrace :
\quad \sum^\infty_{n=1} |\widetilde{\Delta}_n| < |E_1 \cup E_2|^* + \varepsilon, \quad |\widetilde{\Delta}_n| < \frac{d}{2} \; \forall n
$$
Поскольку $|\widetilde{\Delta}_n| < \dfrac{d}{2} \; \forall n$, то интервалы $\widetilde{\Delta}_n$, покрывающие точки $E_1$, не содержат точек $E_2$, а интервалы, покрывающие точки $E_2$, не содержат точек $E_1$. А это значит, что покрытие $\widetilde{s}(E_1 \cup E_2)$ распадается на два непересекающихся покрытия $\widetilde{s_1}(E_1)$ и $\widetilde{s_2}(E_2)$. Тогда
$$
\sum^\infty_{n=1} |\widetilde{\Delta}_n| = \sigma(\tilde{s}) = \sigma(\tilde{s_1}) + \sigma(\tilde{s_2}) 
\geqslant |E_1|^* + |E_2|^*.
$$
С другой стороны,
$$
\sum^\infty_{n=1} |\widetilde{\Delta}_n| < |E_1 \cup E_2|^* + \varepsilon.
$$
Следовательно, $|E_1|^* + |E_2|^* \leqslant |E_1 \cup E_2|^*$. Но по второму свойству внешней меры верно и неравенство $|E_1|^* + |E_2|^* \geqslant |E_1 \cup E_2|^*.$ Итак, получаем, что  $|E_1|^* + |E_2|^* = |E|^*$. $\quad \Box$\\
\item $\forall \varepsilon > 0 \mbox{ и } \forall E \ \exists G \text{ --- открытое, } G \supset E: \; |G|^* < |E|^* + \varepsilon$\smallskip\\
\textbf{Доказательство.}
В качестве $G$ можно взять объединение всех интервалов, составляющих такое покрытие $s(E)$ множества $E$, что $\sigma(s) < |E|^* + \varepsilon$
\end{enumerate}


\textbf{Определение 2.} Множество $E$ называется \textit{измеримым} (по Лебегу), если $\forall \varepsilon > 0$ найдётся открытое множество $G$, содержащее $E$ и такое, что $|G \setminus E|^* < \varepsilon$. При этом $|E| \equiv |E|^*$ называется \textit{мерой} измеримого множества $E$.\\

Из определения меры Лебега и свойства 4 внешней меры следует, что мера множества равна нулю тогда и только тогда, когда внешняя мера множества равна нулю ($|E| = 0 \Leftrightarrow |E|^* = 0$).\\






%%%%%%%%%%%%%%%%%%%%%%%
%      Lecture 2      %
%%%%%%%%%%%%%%%%%%%%%%%






\textbf{Теорема 1.} Любое открытое множество на прямой измеримо, а его мера равна сумме длин (мер) составляющих его попарно не пересекающихся интервалов.\\
\textbf{Доказательство.} Достаточно взять $G = E$. Поскольку множество $E$ --- открытое, $\inf{\sigma(s(E))}$ будет достигнут на покрытии, совпадающем с разбиением $E$ на попарно непересекающиеся интервалы. $\quad \Box$\\

\textbf{Теорема 2.} Объединение конечного или счётного числа измеримых множеств является измеримым множеством.\smallskip\\
\textbf{Доказательство.} Пусть $E = \bigcup\limits_{n=1}^\infty E_n $.\\
$E_n$ измеримо $\forall n$, поэтому
$$
\forall \varepsilon > 0 \; \forall E_n \; \exists G_n \mbox{ --- открытое, } G_n \supset E_n, \; |G_n \setminus E_n|^* < \frac{\varepsilon}{2^n}.
$$
Рассмотрим $G = \bigcup\limits_{n=1}^\infty G_n $. Получаем, что $E \subset G$. $G$ --- открытое. Кроме того, если $x \in (G \setminus E)$, то $x \not \in E_n \; \forall n \text{ и } \exists k: \; x \in G_k \Rightarrow x \in (G_k \setminus E_k)$. Поэтому $(G \setminus E) \; \subset \; \bigcup\limits_{n=1}^\infty (G_n \setminus E_n)$.\\

Из свойства 2 внешней меры получаем, что
$$
|G \setminus E |^* \leqslant \sum_{n=1}^\infty |G_n \setminus E_n|^* < \varepsilon \sum_{n=1}^\infty 2^{-n} = \varepsilon. \quad \Box
$$ 

\textbf{Теорема 3.} Любое замкнутое множество $F$ измеримо на прямой.\\
\textbf{Доказательство.}\\
а) Сначала рассмотрим случай, когда множество $F$ ограничено. По свойству 4 внешней меры $\forall \varepsilon > 0 \; \exists G$ --- открытое, такое что $F \subset G$, $|G|^* < |F|^* + \varepsilon$. Множество $F$ --- замкнутое, поэтому множество $(G \setminus F)$ является открытым. А это значит, что оно представимо в виде суммы $G \setminus F = \bigcup_{n=1}^\infty \Delta_n$ попарно не пересекающихся интервалов $\Delta_n$.\\

Для любого $\Delta = (a, b)$ за $\Delta^\alpha$ будем обозначать интервал $\Delta^\alpha = (a + \alpha, b - \alpha)$, а за $\overline{\Delta}{}^\alpha$ --- сегмент $\overline{\Delta}{}^\alpha = [a + \alpha, b - \alpha]$ (имеется в виду, что $\alpha < \frac{b-a} 2$, в противном случае получаем пустые множества). Для каждого номера $m$ определим множества $E_m = \bigcup\limits_{n=1}^m \Delta_n$, $E_m^\alpha = \bigcup\limits_{n=1}^m \Delta_n^\alpha$ и $\overline{E}{}_m^\alpha = \bigcup\limits_{n=1}^m \overline{\Delta}{}_n^\alpha$.\\

Так как $\forall \alpha > 0$ и для всех $m$ множество $\overline{E}{}_m^\alpha$ не имеет общих точек с множеством $F$, то в силу свойства 3 внешней меры
$$
|\overline{E}{}_m^\alpha \cup F|^* = |\overline{E}{}_m^\alpha|^* + |F|^*.
$$
Кроме того, $\forall \alpha > 0$ и для всех $m$ множество $\overline{E}{}_m^\alpha \cup F$ содержится в $G$, поэтому
$$
|\overline{E}{}_m^\alpha \cup F|^* \leqslant |G|^*.
$$
Получаем, что
$$
|\overline{E}{}_m^\alpha|^* + |F|^* \leqslant |G|^* < |F|^* + \varepsilon.
$$
Так как мы рассматриваем случай ограниченного $F$, то $|F|^* < \infty$. Таким образом, $|\overline{E}{}_m^\alpha|^* < \varepsilon$ ($\forall \alpha > 0$ и для всех $m$). Последовательно переходя в этом неравенстве к пределу при $\alpha \to 0 + 0$ и $m \to \infty$, получим, что $|G \setminus F|^* = \bigcup\limits_{n=1}^\infty |\Delta_n| \leqslant \varepsilon$. А это означает, что $F$ измеримо.\\

б) Если множество $F$ не является ограниченным, то мы можем представить его в виде суммы $F = \bigcup\limits_{n=1}^\infty F_n$, где $F_n = F \cap [-n; n]$. $F$ --- замкнутое $\Rightarrow$ множество $F \cap [-n; n]$ тоже является замкнутым. Получаем, что каждое $F_n$ замкнуто и ограничено, а значит, и измеримо в силу пункта а). Но тогда и само множество $F$ тоже является измеримым по теореме 2. Таким образом, теорема полностью доказана. $\quad \Box$\\


\textbf{Теорема 4.} Если множество $E$ измеримо, то и его дополнение $\complement E$ измеримо.\\
\textbf{Доказательство.}
По определению измеримости $\forall  n \in \mathbb{R}, \ n > 0$ найдётся открытое множество $G_n$, содержащее $E$ и такое, что $|G_n \setminus E|^* < \dfrac 1 n$. Тогда $F_n = \complement G_n$ --- замкнутое множество.\\ 

Заметим, что $\forall A, \ \forall B \quad A \setminus B = \complement B \setminus \complement A$. Поэтому $\complement E \setminus F_n = \complement E \setminus \complement G_n = G_n \setminus E$. А это значит, что
$$
|CE \setminus F_n|^* = |G_n \setminus E|^* < \frac 1 n.
$$
Рассмотрим множество $F = \bigcup\limits_{n=1}^\infty F_n$. Так как $\complement E \setminus F \subset \complement E \setminus F_n$, то $|\complement E \setminus F|^* \leqslant |\complement E \setminus F_n|^* < \dfrac 1 n$. Поскольку это верно для любого $n$, то $|\complement E \setminus F|^* = 0$, а, следовательно, это множество измеримо и $|\complement E \setminus F| = 0$. \\

Получаем, что $\complement E = (\complement E \setminus F) \cup F$. Множество $(\complement E \setminus F)$ измеримо, множество $F$ также измеримо в силу теорем 2 и 3. Следовательно, $\complement E$ является измеримым множеством. $\quad \Box$\\

\textbf{Следствие.} Для того, чтобы множество $E$ было измеримо, необходимо и достаточно, чтобы для $\forall \varepsilon > 0$ нашлось замкнутое множество $F \subset E$ такое, что $|E \setminus F|^* < \varepsilon$.\\
\textbf{Доказательство.}
Из теоремы следует, что измеримость множества $E$ эквивалентна измеримости множества $\complement E$, то есть эквивалентна требованию
$$
\forall \varepsilon > 0 \;  \exists G \mbox{ --- открытое, } G \supset \complement E, \; |G \setminus \complement E|^*< \varepsilon.
$$
В силу того, что $\complement E_1 \setminus \complement E_2 \equiv E_2 \setminus E_1$, это эквивалентно требованию
$$
\forall \varepsilon > 0 \;  \exists F = \complement G \mbox{ --- замкнутое, } F \subset E, \; |E \setminus F|^* = |\complement F \setminus \complement E|^* = |G \setminus \complement E|^*< \varepsilon,
$$
a это выполняется по условию следствия. $\quad \Box$\\

\textbf{Теорема 5.} Пересечение конечного или счётного числа измеримых множеств измеримо.\\
\textbf{Доказательство.}
Пусть даны измеримые множества $E_1, E_2, \dots$, их пересечение --- $E = \bigcap\limits_{n=1}^\infty E_n$. Тогда верно равенство $\complement E = \bigcup\limits_{n=1}^\infty \complement E_n$, а, следовательно, и равенство $E = \complement (\bigcup\limits_{n=1}^\infty \complement E_n)$. Для любого $n$ $\complement E_n$ измеримо по теореме 4 $\Rightarrow$ по теореме 2 $\bigcup\limits_{n=1}^{\infty}\complement E_n$ измеримо $\Rightarrow$ по теореме 4 измеримо $E = \complement \bigcup\limits_{n=1}^{\infty}\complement E_n$. $\quad \Box$\\

\textbf{Теорема 6.} Если множества $A$ и $B$ измеримы, то множество $A \setminus B$ также измеримо.\\
\textbf{Доказательство.} Вытекает из теорем 4, 5 и тождества $A \setminus B \equiv A \cap \complement B$. $\quad \Box$\\

\textbf{Теорема 7 ($\sigma$-аддитивность меры).} Пусть измеримое множество $E$ представимо в виде конечного или счётного объединения попарно не пересекающихся измеримых множеств. Тогда его мера равна сумме мер этих множеств.\\ 
\textbf{Доказательство.}\\
а) Сначала рассмотрим случай, когда все $E_n$ ограничены.\\
По следствию из теоремы 4 для любого $\varepsilon > 0$ и для каждого номера $n$ найдётся замкнутое множество $F_n \subset E_n$ такое, что $|E_n \setminus F_n| < \dfrac{\varepsilon}{2^n}$ (все фигурирующие в доказательстве множества измеримы, поэтому вместо внешней меры будем писать просто меру). Все множества $F_n$ ограничены, замкнуты и попарно не пересекаются. Более того, из-за замкнутости расстояние между ними больше 0. Поэтому в силу свойства 3 внешней меры для любого конечного $m$ выполняется равенство
$$
\left| \bigcup_{k=1}^m F_k \right| = \sum_{k=1}^m |F_k|.
$$
Так как сумма всех множеств $F_k$ содержится в $E$, то для любого номера $m$
$$
\sum_{k=1}^m |F_k| = \left| \bigcup_{k=1}^m F_k \right| \leqslant |E|.
$$
С другой стороны, $E_n = (E_k \setminus F_k) \cup F_k$, поэтому $|E_k| \leqslant |E_k \setminus F_k| + |F_k| < |F_k| + \dfrac \varepsilon {2^k}$. Получаем, что
$$
\sum_{k=1}^m |E_k| < \sum_{k=1}^m |F_k| + \varepsilon \leqslant |E| + \varepsilon.
$$
Перейдём в этом неравенстве к пределу при $m \to \infty$, $\varepsilon \to 0$. Получаем, что
$$
\sum_{k=1}^\infty |E_k| \leqslant |E|.
$$
С другой стороны, $\bigcup\limits_{k=1}^\infty E_k = E$, поэтому в силу свойства 2 внешней меры
$$
\sum_{k=1}^\infty |E_k| \geqslant |E|.
$$
Из двух последних неравенств следует, что $\sum\limits_{k=1}^\infty |E_k| = |E|$, а это и требовалось доказать.\\

б) Пусть теперь множества $E_n$ не обязательно являются ограниченными. Тогда введём в рассмотрение множества $E_n^k = E_n \cap (k - 1 \leqslant |x| < k)$, которые будут являться ограниченными.\\

Так как $E = \sum\limits_{n=1}^\infty \sum\limits_{k=1}^\infty E_n^k$, то
$$
|E| = \sum\limits_{n=1}^\infty \sum\limits_{k=1}^\infty |E_n^k| = \sum\limits_{n=1}^\infty |E_n|.
$$
Таким образом, теорема полностью доказана. $\quad \Box$\\

\textbf{Определение 3.} Множество $G$ называется \textit{множеством типа $G_\delta$}, если оно представимо в виде пересечения счётного числа открытых множеств $G_n$ ($G = \bigcap\limits_{n=1}^\infty G_n$), и \textit{множеством типа $F_\sigma$}, если $E$ представимо в виде объединения счётного числа замкнутых множеств $F_n$ ($F = \bigcup\limits_{n=1}^\infty F_n$).\\

\textbf{Теорема 8.} 
Для любого измеримого множества $E$ существует множество $E_1$ типа $G_\delta$ и множество $E_2$ типа $F_\sigma$ такие, что $E_1 \supset E \supset E_2$ и $|E_1| = |E| = |E_2|$.\\
\textbf{Доказательство.} В силу измеримости $E$ и следствия из теоремы 4 для любого номера $n \in \mathbb{N}$ $\exists G_n \supset E, \ \exists F_n \subset E$ такие, что
$$
|G_n \setminus E| < \frac 1 n, \quad |E \setminus F_n| < \frac 1 n.
$$
Положим $E_1 = \bigcap\limits_{n=1}^\infty G_n$, $E_2 = \bigcup\limits_{n=1}^\infty F_n$. Так как для любого номера $n$
$$
E_1 \setminus E \subset G_n \setminus E, \; E \setminus E_2 \subset E \setminus F_n,
$$
то верны неравенства
$$
|E_1 \setminus E| < \frac 1 n, \quad |E \setminus E_2| < \frac 1 n.
$$
В силу произвольности $n$ это означает, что $|E_1 \setminus E| = 0$ и $|E \setminus E_2| = 0$, а, значит, $|E_1| = |E| = |E_2|$.

$$E \subset E_1 \subset G_n\text{, значит }|E| \leqslant |E_1| \leqslant|G_n| = |(G_n \setminus E) \cup E| = |G_n \setminus E| + |E| < \frac1n + |E|$$
$\quad \Box$\\


\textbf{Пример неизмеримого множества.}

Рассмотрим единичную окружность.

Выберем $\alpha > 0$.

$\Phi_\beta$ --- множество всех точек, которые можно совместить друг с другом поворотом на $\Pi n \alpha$, содержащее точку $\beta$. В этом классе счетное число точек. Выберем такое множество $B$, что для любых двух $\beta_1, \beta_2 \in B$ классы не совпадают ($\Phi_{\beta_1} \neq \Phi_{\beta_2}$).

$\Phi_0$ содержит по одной точке из каждого класса $\Phi_\beta$. Поскольку $B$ имеет мощность континуум, то $\Phi_0$ тоже имеет мощность континуум.

Очевидно, что вся окружность $C$ представима как $C = \bigcup\limits_{n = -\infty}^\infty \Phi_n$, где $\Phi_n$ --- поворот множества $\Phi_0$ на $\Pi n \alpha$.

Если $\Phi_0$ измеримо, то 
$$2\Pi = |C| = |\bigcup\limits_{n = -\infty}^\infty \Phi_n| = \sum\limits_{n = -\infty}^\infty |\Phi_n| = \sum\limits_{n = -\infty}^\infty |\Phi_0|$$
что приводит нас к противоречию.


%%%%%%%%%%%%%%%%%%%%%%%
%      Lecture 3      %
%%%%%%%%%%%%%%%%%%%%%%%






\section*{\S 3. Измеримые функции}

Обозначим символом $E[f(x) > a]$ множество
$$
E[f(x) > a] = \lbrace x \in E: \; f(x) > a \rbrace.
$$
В дальнейшем мы рассматриваем только функции, определённые на измеримых множествах, и допускаем, что они могут принимать значения $-\infty , +\infty$.\\

\textbf{Определение 1.} Функция $f(x)$, определенная на измеримом множестве $E$, называется \textit{измеримой на $E$}, если $\forall a \in \mathbb R$ $E[f(x) \geqslant a]$ --- измеримое множество.\\

Свойства измеримых функций:

\begin{enumerate}
\item Для того, чтобы функция $f(x)$ была измеримой на множестве $E$, необходимо и достаточно, чтобы при любом $a \in \mathbb R$ одно из множеств
$$
E [f(x) > a], \qquad E [f(x) \leqslant a], \qquad E [f(x) < a]
$$
было измеримо.\\
\textbf{Доказательство.}\\
Тот факт, что измеримость $\forall a \in \mathbb R$ множества $E[f(x) > a]$ является необходимым и достаточным условием измеримости функции $f(x)$ на множестве $E$, следует из следующих соотношений:
$$
E[f(x) > a] = \bigcup_{n=1}^\infty E[f(x) \geqslant a + \frac 1 n],
$$
$$
E[f(x) \geqslant a] = \bigcap_{n=1}^\infty E[f(x) > a - \frac 1 n].
$$
Тот факт, что измеримость $\forall a \in \mathbb R$ множества $E[f(x) \leqslant a]$ является необходимым и достаточным условием измеримости функции $f(x)$ на множестве $E$, следует из соотношения
$$
E [f(x) \leqslant a] = E \setminus E[f(x) > a].
$$
Тот факт, что измеримость $\forall a \in \mathbb R$ множества $E[f(x) < a]$ является необходимым и достаточным условием измеримости функции $f(x)$ на множестве $E$, следует из соотношения
$$
E [f(x) < a ] = E \setminus E [f(x) \geqslant a]. \quad \Box
$$

\item Если функция $f(x)$ измерима на множестве $E$, то она измерима и на любом измеримом подмножестве $E_1 \subset E$.\\
\textbf{Доказательство.}
Это непосредственно следует из тождества
$$
E_1[f(x) \geqslant a] \equiv E [f(x) \geqslant a] \cap E_1.
$$

\item Если функция $f(x)$ измерима на множестве $E_k$ (при всех номерах $k$), то она измерима и на их объединении $E = \bigcup\limits_{k=1}^\infty E_k$.\\
\textbf{Доказательство.}
Это непосредственно следует из тождества
$$
E[f(x) \geqslant a] = \bigcup_{k=1}^\infty E_k[f(x) \geqslant a].
$$

\item Любая функция измерима на множестве меры 0 (так как любое подмножество меры 0 имеет меру 0 и измеримо).\\

\textbf{Определение 2.} Функции $f(x)$ и $g(x)$ называются эквивалентными на множестве $E$, если $|E[f(x) \neq g(x)]| = 0$.

\item Если $f(x)$ измерима на множестве $E$, то и любая эквивалентная ей на $E$ функция $g(x)$ измерима на $E$.\\
\textbf{Доказательство.}
Рассмотрим множества $E_0 = E[f(x) \neq g(x)]$ и $E_1 = E \setminus E_0$. В силу свойства 2 $f(x)$ измерима на $E_1$, $f(x) = g(x)$ на $E_1$ $\Rightarrow$ $g(x)$ измерима на $E_1$. С другой стороны, по свойству 4 $g(x)$ измерима на $E_0$, так как это множество имеет меру 0 (в силу эквивалентности $f(x)$ и $g(x)$). Получаем, что $g(x)$ измерима на всём $E$ (в силу свойства 3). $\quad \Box$\\

\textbf{Определение 3.} Говорят, что какое-то свойство выполняется почти всюду на множестве $E$, если множество точек, на которых это свойство не выполняется, имеет меру 0.

\item Если функция $f(x)$ непрерывна почти всюду на измеримом множестве $E$, то она измерима на этом множестве.\\
\textbf{Доказательство.} Обозначим через $R \subset E$ подмножество всех точек разрыва $f(x)$. Поскольку $f(x)$ непрерывна на $E$ почти всюду, $R$ имеет меру 0. Поэтому в силу свойств 3 и 4 достаточно доказать измеримость $f(x)$ на множестве $E_1 = E \setminus R$. В силу теоремы 8 $\S 2$ существует множество $E_2$ типа $F_\sigma$, содержащееся в $E_1$ и такое, что $|E_2| = |E_1| = |E|$. Опять же, в силу свойств 3 и 4 достаточно доказать, что $f(x)$ измерима на множестве $E_2$. Но $E_2$ является множеством типа $F_\sigma$, поэтому оно представимо в виде счётной суммы замкнутых множеств $F_n$. На каждом из $F_n$ функция $f(x)$ является непрерывной $\Rightarrow$ при любом вещественном $a$ множество $F_n[f(x) \geqslant a]$ замкнуто, а значит, и измеримо. Таким образом, $f(x)$ измерима на всех $F_n$, поэтому она измерима и на $E_2$. Свойство доказано. $\quad \Box$\\

\textbf{Замечание.} Эквивалентность $f(x)$ на множестве $E$ некоторой непрерывной функции следует отличать от непрерывности $f(x)$ почти всюду на $E$. Например, функция Дирихле
$$
D(x)=\begin{cases}
1,&\text{если $x \in \mathbb{Q}$}\\
0,&\text{если $x \notin \mathbb{Q}$}
\end{cases}
$$
разрывна в каждой точке, но эквивалентна на сегменте $[0; 1]$ непрерывной функции $g(x) \equiv 0$ (поскольку на этом сегменте $D(x) \neq g(x)$ только на множестве рациональных точек, которое счётно и потому имеет меру 0) и, следовательно, является измеримой на $[0; 1]$.
\end{enumerate}

\textbf{Теорема 1}. Пусть функция $f(x)$ измерима на множестве $E$. Тогда функции $|f(x)|$, $c \cdot f(x)$, $f(x) + c$ (где c = const) также измеримы на $E$. Множество $E[f(x) > g(x)]$ измеримо в том случае, если $g(x)$ --- измеримая функция.\\
\textbf{Доказательство.}\\
1) Достаточно рассмотреть следующие соотношения, выполняющиеся для любого вещественного $a$:
$$
E[|f(x)| \geqslant a]=\begin{cases}
E[f(x) \geqslant a] \cup E[f(x) \leqslant -a],&\text{если $a \geqslant 0$}\\
E,&\text{если $a < 0$}
\end{cases}
$$
$$
E[f(x) + c \geqslant a] = E[f(x) \geqslant a - c],
$$
$$
E[c \cdot f(x) \geqslant a]=\begin{cases}
E[f(x) \geqslant \frac a c],&\text{если $c > 0$}\\
E[f(x) \leqslant \frac a c],&\text{если $c < 0$}
\end{cases}
$$
Из них следует, что $E[|f(x)| \geqslant a]$ и $E[f(x) + c \geqslant a]$ являются измеримыми множествами, множество $E[c \cdot f(x) \geqslant a]$ измеримо при $c \neq 0$, поэтому соответствующие функции измеримы на $E$ (при $c = 0$ функция $c \cdot f(x) \equiv 0$ и также является измеримой).\\

2) Занумеруем все рациональные числа $r_k$ действительной оси, тогда
$$
E[f(x) > g(x)] = \bigcup_{r_k} (E [f(x)> r_k] \cap E[g(x) < r_k]).
$$
Поэтому в случае измеримости функции $g(x)$ множество $E[f(x) > g(x)]$ также будет являться измеримым. $\quad \Box$\\

\textbf{Теорема 2.} Пусть функции $f(x)$ и $g(x)$ измеримы на множестве $E$. Тогда функции $f(x) \pm g(x)$, $f(x)\cdot g(x)$, $\dfrac{f(x)}{g(x)}$ (при $g(x) \neq 0$) также измеримы на множестве $E$.\\
\textbf{Доказательство.} Рассмотрим следующее соотношение:
$$
E[f(x) \pm g(x) \geqslant a] = E[f(x) \geqslant \mp g(x) + a].
$$
В силу теоремы 1 из него следует, что функции $f(x) \pm g(x)$ измеримы на множестве $E$.
$$
E[f^2(x) > a]=\begin{cases}
E[|f(x)| > \sqrt{a}],&\text{если $a \geqslant 0$}\\
E,&\text{если $a < 0$}
\end{cases}
$$
Из этого неравенства вытекает, что функция $f^2(x)$ является измеримой на $E$.
$$
f(x) \cdot g(x) = \frac 1 4 [(f(x) + g(x))^2 - (f(x) - g(x))^2]
$$
Так как измеримость квадрата измеримой функции только что была доказана, функция $f(x) \cdot g(x)$ также измерима на $E$.

Если $g \neq 0$, то
$$
E[\frac 1 {g(x)} > a]=\begin{cases}
E[g(x) > 0] \cap E[g(x)< \frac 1 a], &\text{если $a > 0$}\\
E[g(x) > 0], &\text{если $a = 0$}\\
E[g(x) > 0] \cup E[g(x) < \frac 1 a], &\text{если $a < 0$}
\end{cases}
$$
Из этих соотношений вытекает измеримость функции $\dfrac{1}{g(x)}$ $\Rightarrow$ функция $f(x) \cdot \dfrac{1}{g(x)} = \dfrac{f(x)}{g(x)}$ также является измеримой на $E$. $\quad \Box$\\

\textbf{Теорема 3.} Пусть E --- измеримое множество, на котором определена последовательность измеримых функций ${f_n(x)}$.
Тогда $\underline{f}(x) = \varliminf\limits_{n \to \infty} f_n(x)$ и $\overline{f}(x) = \varlimsup\limits_{n \to \infty} f_n(x)$ этой последовательности --- измеримые функции.\\
\textbf{Доказательство.} Рассмотрим функции
$$
\varphi(x) = \inf_n f_n(x), \qquad \psi(x) = \sup_n f_n(x).
$$
Они являются измеримыми на множестве $E$, так как
$$
E[\varphi(x) < a] = \bigcup_{n=1}^\infty E[f_n(x) < a],
$$
$$
E[\psi(x) > a] = \bigcup_{n=1}^\infty E[f_n(x) > a].
$$
Теперь представим функции $\underline{f}(x)$ и $\overline{f}(x)$ в виде
$$
\underline{f}(x) = \sup_{n \geqslant 1} \lbrace \inf_{k \geqslant n} f_k(x) \rbrace, \qquad
\overline{f}(x) = \inf_{n \geqslant 1} \lbrace \sup_{k \geqslant n} f_k(x) \rbrace.
$$
В силу измеримости функций $\varphi(x)$ и $\psi(x)$ функции $\underline{f}(x)$ и $\overline{f}(x)$ также являются измеримыми на $E$. $\quad \Box$\\

\textbf{Теорема 4.} Пусть $E$ --- измеримое множество, и на нем определена последовательность измеримых функций $\lbrace f_n(x) \rbrace$. Пусть $\lbrace f_n(x) \rbrace$ почти всюду сходится к функции $f(x)$. Тогда $f(x)$ измерима на $E$.\\
\textbf{Доказательство.}
Пусть $\lbrace f_n(x) \rbrace$ сходится к $f(x)$ на $E$ всюду, кроме множества $E_0$ меры 0. Получаем, что $f(x)$ измерима на множестве $E \setminus E_0$ (в силу теоремы 3, поскольку на $E \setminus E_0$ функция $f(x) = \varliminf\limits_{n \to \infty} f_n(x) = \varlimsup\limits_{n \to \infty} f_n(x) = \lim\limits_{n \to \infty} f_n(x)$) и измерима на множестве $E_0$, так как оно имеет меру 0. Следовательно, $f(x)$ измерима на $(E \setminus E_0) \cup E_0 = E$. $\quad \Box$\\

\textbf{Определение 4.} Пусть $E$ --- измеримое множество, $f_n(x) \ (n = 1,\ 2,\dots)$, $f(x)$ --- измеримые, почти всюду конечные на множестве $E$ функции. Говорят, что последовательность $\lbrace f_n(x) \rbrace$ \textit{сходится к $f(x)$ по мере} на множестве $E$, если для любого $\varepsilon > 0$
$$
\lim_{n\to\infty} |E[|f_n(x)-f(x)| \geqslant \varepsilon]| = 0,
$$
то есть если для любых $\varepsilon > 0$, $\delta > 0$ найдётся номер $N = N(\varepsilon, \delta)$ такой, что при любом номере $n \geqslant N$ справедливо неравенство
$$
|E[|f_n(x)-f(x)| \geqslant \varepsilon]| < \delta.
$$

\textbf{Теорема 5 (теорема Лебега).} Пусть $E$ --- измеримое множество конечной меры, функции $f_n(x) \ (n = 1,\ 2,\dots)$ и $f(x)$ измеримы и почти всюду конечны на $E$. Тогда из сходимости последовательности $\lbrace f_n(x) \rbrace$ к $f(x)$ почти всюду на $E$ вытекает сходимость $\lbrace f_n(x) \rbrace$ к $f(x)$ по мере на множестве $E$.\\
\textbf{Доказательство.}
Рассмотрим множества
$$
A = E[|f(x)| = +\infty],
$$
$$
A_n = E[|f_n(x)|=+\infty],
$$
$$
B=E\setminus E[\lim_{n\to\infty} f_n(x) = f(x)],
$$
$$
C = A \cup B \cup {\bigcup_{n=1}^\infty A_n}.
$$
Тогда по условию теоремы $|C| = 0$ и всюду на множестве $E \setminus C$ последовательность $\lbrace f_n(x) \rbrace$ сходится к $f(x)$, а все функции $f_n(x)$ и $f(x)$ имеют конечные значения.\\

Фиксируем произвольное $\varepsilon$. Рассмотрим множества
$$
E_n = E[|f_n(x) - f(x)| \geqslant \varepsilon],
$$
$$
R_n = \bigcup_{k=n}^\infty E_k, \qquad R = \bigcap_{n=1}^\infty R_n.
$$
Поскольку $E_n \subset R_n$, справедливо неравенство $|E_n| \leqslant |R_n|$, и для доказательства теоремы достаточно доказать, что $|R_n| \to 0$ при $n \to \infty$.\\

Сначала докажем, что $|R_n| \to |R|$ при $n \to \infty$. По построению $R_{n+1} \subset R_n$ для каждого номера $n$, поэтому для любого $n$
$$
R_n \setminus R = \bigcup_{k=n}^\infty (R_k \setminus R_{k+1}).
$$
Заметим, что суммируемые множества попарно не пересекаются. Поэтому для каждого $n$
$$
|R_n \setminus R| = \sum_{k=n}^\infty  |R_k \setminus R_{k+1}|.
$$
В силу того, что множество $E$ имеет конечную меру, $|R_n \setminus R| < \infty$. Поэтому ряд
$$
|R_1 \setminus R| = \sum_{k=1}^\infty  |R_k \setminus R_{k+1}|
$$
сходится, а его остаток $|R_n \setminus R| \to 0$ при $n \to \infty$. В силу того, что $R_n = (R_n \setminus R) \cup R$, выполняется равенство $|R_n| = |R_n \setminus R| + |R|$. Поскольку $|R_n \setminus R| \to 0$ при $n \to \infty$, то $|R_n| \to |R|$ при $n \to \infty$. Теперь для доказательства теоремы достаточно доказать, что $|R| = 0$. В силу того, что $|C| = 0$, достаточно доказать, что $R \subset C$.\\

Пусть $x_0$ --- любая точка, не принадлежащая $C$. Тогда для произвольного фискированного нами $\varepsilon > 0$ найдётся номер $N = N(x_0, \varepsilon)$ такой, что при любом $n \geqslant N$ верно неравенство $|f_n(x_0)-f(x_0)| < \varepsilon$. Это означает, что при $n \geqslant N$ точка $x_0 \notin E_n$ $\Rightarrow$ при $n \geqslant N$ точка $x_0 \notin R_n$ $\Rightarrow$ точка $x_0 \notin R$.\\

Итак, любая точка, не принадлежащая $C$, не принадлежит и $R$. Это означает, что $\complement C \subset \complement R$. Следовательно, $R \subset C$. Теорема доказана. $\quad \Box$\\

\textbf{Замечание 1.} Ключевым в теореме Лебега является ограничение конечности меры множества $E$. На множестве бесконечной меры из сходимости почти всюду сходимость по мере, вообще говоря, не следует. Пример:
$$
f_n(x)=\begin{cases}
1, &\text{если $x \in [n, n+1]$}\\
0 &\text{иначе}
\end{cases}
$$
Получаем, что
$$
f_n(x) \to f(x) = 0 \mbox{ на } \mathbb{R}, \mbox{ но при этом } |E[|f_n(x) - 0| > \frac 1 2 ]| = 1.
$$

\textbf{Замечание 2.} Из сходимости по мере, вообще говоря, не следует сходимость почти всюду. Например, рассмотрим такую систему сегментов:
$$
I_1 = [0; 1]
$$
$$
I_2 = \left[0; \frac12\right], \; I_3 = \left[\frac12; 1\right] 
$$
$$
I_4 = \left[0; \frac14\right], \; I_5 = \left[\frac14; \frac12\right], \;
I_6 = \left[\frac12; \frac34\right], \; I_7 = \left[\frac34; 1\right] \mbox{ и так далее.}
$$
Определим на сегменте $[0; 1]$ последовательность функций $\lbrace f_n(x) \rbrace$, где
$$
f_n(x)=\begin{cases}
1, &\text{если $x \in I_n$}\\
0 &\text{если $x \in [0; 1] \setminus I_n$}
\end{cases}
$$
Получаем, что последовательность $\lbrace f_n(x) \rbrace$ расходится в каждой точке сегмента $[0; 1]$, но при этом сходится к функции $f(x) \equiv 0$ по мере на этом же сегменте.\\

\textbf{Теорема 6 (теорема Рисса).} Пусть E --- измеримое множество конечной меры, функции $f_n(x) \ (n = 1,\ 2,\dots)$ и $f(x)$ измеримы и почти всюду конечны на $E$. Тогда, если последовательность $\lbrace f_n(x) \rbrace$ сходится к $f(x)$ по мере на множестве $E$, то из неё можно выделить подпоследовательность $\lbrace f_{n_k}(x) \rbrace$, сходящуюся к $f(x)$ почти всюду на множестве $E$.\\
\textbf{Доказательство.}
Не ограничивая общности, можем считать, что функции $f_n(x)$ и $f(x)$ принимают конечные значения всюду на множестве $E$ (если это не так, то мы можем, как в доказательстве теоремы 5, исключить из рассмотрения множество меры 0, где эти функции не конечны).\\

Последовательность $\lbrace f_n(x) \rbrace$ сходится к $f(x)$ по мере на множестве $E$, поэтому для любого номера $k \in \mathbb{N}$ найдётся номер $n_k$ такой, что для меры множества $E_k = E[|f_{n_k} - f(x)| \geqslant \dfrac 1 k]$ справедливо неравенство $|E_k| < \dfrac 1 {2^k}$. Положим $R_n = \bigcup\limits_{k=n}^\infty E_k, \; R = \bigcap\limits_{n=1}^\infty R_n$. Тогда $|R_n| \leqslant \sum\limits_{k=n}^\infty |E_k| < \sum\limits_{k=n}^\infty \dfrac{1}{2^{n}} \leqslant \dfrac{1}{2^{n-1}}$. Таким образом, $|R_n| \to 0$ при $n \to \infty$. Как и в теореме 5, доказываем, что $|R_n| \to |R|$ при $n \to \infty$. Тем самым мы получаем, что $|R| = 0$.

Докажем, что подпоследовательность $\lbrace f_{n_k}(x) \rbrace$ сходится к $f(x)$ всюду на множестве $E \setminus R$. Пусть $x$ --- произвольная точка $E \setminus R$. Тогда $x$ не принадлежит $R_N$ при некотором $N = N(x)$. Но это означает, что $x$ не принадлежит множеству $E_k$ при всех $k \geqslant N(x)$. Таким образом, для всех $k \geqslant N(x)$ $|f_{n_k}(x) - f(x)| < \dfrac 1 k$, то есть подпоследовательность $\lbrace f_{n_k}(x) \rbrace$ сходится к $f(x)$. $\quad \Box$\\

\textbf{Теорема 7.} Пусть E --- измеримое множество конечной меры, функции $f_n(x) \ (n = 1,\ 2,\dots)$, $f(x)$ и $g(x)$ измеримы и почти всюду конечны на $E$, последовательность $\lbrace f_n(x) \rbrace$ сходится к $f(x)$ и к $g(x)$ по мере на $E$. Тогда $f(x)$ и $g(x)$ эквивалентны.\\
\textbf{Доказательство.} %В силу того, что $\lbrace f_n(x) \rbrace$ сходится к $f(x)$ и к $g(x)$ по мере на множестве $E$, %для любого $\varepsilon > 0$ справедливы неравенства
%$$
%\left| E \left[|f_n(x) - f(x)| \geqslant \frac \varepsilon 2 \right] \right| = 0, \quad
%\left| E \left[|f_n(x) - g(x)| \geqslant \frac \varepsilon 2 \right] \right| = 0.
%$$
Тогда в силу соотношения
\begin{multline*}
\forall \varepsilon > 0 \quad E[|f(x)-g(x)| \geqslant \varepsilon] \subset \\
\subset \biggl( E \left[|f_n(x) - f(x)| \geqslant \frac \varepsilon 2 \right] \; \cup \; E\left[|f_n(x)-g(x)| \geqslant \frac \varepsilon 2 \right] \biggr),
\end{multline*}
для любого $\varepsilon > 0$ справедливо неравенство
\begin{multline*}
\left| E[|f(x)-g(x)| \geqslant \varepsilon] \right| \leqslant \\
\leqslant \left| E \left[|f_n(x) - f(x)| \geqslant \frac \varepsilon 2 \right] \right| + \left| E \left[|f_n(x) - g(x)| \geqslant \frac \varepsilon 2 \right] \right| \to 0.
\end{multline*}
Следовательно,
$$
\forall \varepsilon > 0 \quad \left| E[|f(x)-g(x)| \geqslant \varepsilon] \right| = 0.
$$
Далее, из соотношения
$$
E[f(x)\neq g(x)] = \bigcup_{n=1}^\infty E\left[|f(x) - g(x)| \geqslant \frac 1 n\right]
$$
следует, что
$$
|E[f(x)\neq g(x)]| \leqslant \sum_{n=1}^\infty \left| E\left[|f(x) - g(x)| \geqslant \frac 1 n\right] \right|.
$$
Все суммируемые нормы в правой части равенства равны 0, поэтому $|E[f(x)\neq g(x)]| = 0$, а это означает, что функции $f(x)$ и $g(x)$ эквивалентны. $\quad \Box$\\

\textbf{Теорема 8 (теорема Егорова).} Пусть $E$ --- измеримое множество конечной меры, функции $f_n(x) \ (n = 1,\ 2,\dots)$ и $f(x)$ измеримы и почти всюду конечны на $E$, последовательность $\lbrace f_n(x) \rbrace$ сходится к $f(x)$ почти всюду на $E$. Тогда для любого $\delta > 0$ существует такое измеримое множество $E_\delta \subset E$, что $|E_\delta| > |E| - \delta$ и на множестве $E_\delta$ последовательность $\lbrace f_n(x) \rbrace$ сходится к $f(x)$ равномерно.\\

\textbf{Теорема 9 (теорема Лузина).} Пусть $E$ --- измеримое множество конечной меры, функция $f(x)$ измерима и почти всюду конечна на множестве $E$. Тогда для любого $\varepsilon > 0$ существует множество $E_\varepsilon \subset E$ такое, что $|E_\varepsilon| > |E| - \varepsilon$, а функция $\varphi(x)$ такая, что $\varphi(x) = f(x)$ на $E_\varepsilon$ ("сужение"\ функции $f(x)$ на множество $E_\varepsilon$), является непрерывной на $E_\varepsilon$.\\




%%%%%%%%%%%%%%%%%%%%%%%
%      Lecture 4      %
%%%%%%%%%%%%%%%%%%%%%%%






\section*{\S 4. Интеграл Лебега}

\subsection*{4.1. Интеграл Лебега от ограниченной функции на измеримом множестве конечной меры}

$$
|f(x)| \leqslant M, \qquad |E| < + \infty.
$$

Назовем \textit{разбиением множества $E$} конечный набор $T$ его подмножеств, попарно не пересекающихся и составляющих его в объединении:
$$
E_i \cap E_j = \varnothing, \; i \neq j; \quad \bigcup_{k=1}^n E_k = E; \quad T = \lbrace E_k \rbrace_{k=1}^n.
$$
Рассмотрим на измеримом множестве $E$ конечной меры произвольную ограниченную функцию $f(x)$. Для произвольного разбиения $T = \lbrace E_k \rbrace_{k=1}^n$ множества $E$ обозначим символами $M_k$ и $m_k$ соответственно \textit{точную верхнюю} и \textit{точную нижнюю грани} функции $f(x)$ на множестве $E_k$:
$$
M_k = \sup_{x\in E_k} f(x), \quad m_k = \inf_{x\in E_k} f(x).
$$
Кроме того, определим \textit{верхнюю интегральную сумму $S_T$} и \textit{нижнюю интегральную сумму $s_T$} разбиения $T$ следующим образом:
$$
S_T = \sum_{k=1}^n M_k |E_k|, \quad s_T = \sum_{k=1}^n m_k |E_k|.
$$
Очевидно, что $s_T \leqslant S_T$ при любом разбиении $T$.\\

Для любой ограниченной на множестве конечной меры $E$ функции $f(x)$ как множество всех верхних интегральных сумм $\lbrace S_T \rbrace$, так и множество всех нижних интегральных сумм $\lbrace s_T \rbrace$ (отвечающих всевозможным разбиениям $T$ множества $E$) ограничено. Поэтому существует $\inf\limits_T S_T = \overline{I}$, который мы назовём \textit{верхним интегралом Лебега}, и существует $\sup\limits_T s_T = \underline{I}$, который мы назовём \textit{нижним интегралом Лебега}.\\

\textbf{Определение 1.} Если $\overline{I} = \underline{I} = I$, то функция $f(x)$ называется \textit{интегрируемой по Лебегу} на множестве $E$.При этом $I$ называется \textit{интегралом Лебега} от функции $f(x)$ по множеству $E$ и обозначается
$$
I = \int\limits_E f(x) dx.
$$
Разбиение $T^* = \lbrace E_i^* \rbrace_{i=1}^m$ будем называть \textit{измельчением} разбиения $T = \lbrace E_k \rbrace_{k=1}^n$, если для любого номера $i$, $1\leqslant i \leqslant m$, найдётся номер $\nu(i)$ такой, что $1 \leqslant \nu(i) \leqslant n$ и $E_i^* \subset E_{\nu (i)}$. При этом, очевидно, выполняется равенство $\bigcup\limits_{\nu(i) = k} E_i^* = E_k$.\\

Точная верхняя грань подмножества $E_i^* \subset E_k$ всегда не превосходит точную верхнюю грань всего множества $E_k$, поэтому для всех номеров $i$, для которых $\nu(i) = k$, справедливо неравенство $M_i^* \leqslant M_k$. Применим это неравенство:
\begin{multline*}
S_{T^*} = \sum_{i=1}^m M_i^* |E_i^*| = \sum_{k=1}^n \sum_{\nu(i) = k} M_i^* |E_k^*| \leqslant \\
\leqslant \sum_{k=1}^n \sum_{\nu(i) = k} M_k |E_i^*| = \sum_{k=1}^n M_k \sum_{\nu(i) = k} |E_i^*| = \sum_{k=1}^n M_k |E_k| = S_T.
\end{multline*}
Таким образом, выполняются неравенства $S_{T^*} \leqslant S_T, s_{T^*} \geqslant s_T$ (доказательство второго неравенства проводится аналогично).\\

Разбиение $T$ будем называть \textit{произведением} множеств $T_1$ и $T_2$, если оно состоит из множеств, являющихся пересечениями всевозможных пар элементов $T_1$ и $T_2$.\\

Очевидно, что $T$ является измельчением $T_1$ и $T_2$. Таким образом, для двух произвольных разбиений $T_1$, $T_2$ и их произведения $T$ справедливы неравенства $s_{T_1} \leqslant s_T, S_{T} \leqslant S_{T_2}$. Кроме того, $s_T \leqslant S_T$. Из этих неравенств следует, что $s_{T_1} \leqslant s_T \leqslant S_T \leqslant S_{T_2}$, то есть $s_{T_1} \leqslant S_{T_2}$ для любых двух произвольных разбиений $T_1$, $T_2$.\\

Фиксируем произвольное разбиение $T_2$. Так как для любого разбиения $T_1$ выполняется неравенство $s_{T_1} \leqslant S_{T_2}$, то $S_{T_2}$ является одной из верхних граней множества $\lbrace s_{T_1} \rbrace$, поэтому $\sup\limits_T s_{T_1} = \underline{I} \leqslant S_{T_2}$. Но так как $\underline{I} \leqslant S_{T_2}$ для произвольного фиксированного нами разбиения $T_2$, то $\underline{I}$ является одной из нижних граней множества $\lbrace S_{T_2} \rbrace$, а это означает, что $\inf\limits_T S_{T_2} = \overline{I} \geqslant \underline{I}$.\\

Итак, верхний и нижний интегралы Лебега связаны соотношением $\underline{I} \leqslant \overline{I}$.\\

\textbf{Теорема 1.} Если функция $f(x)$ интегрируема по Риману на сегменте $[a; b]$, то она интегрируема по Лебегу на этом сегменте, причем интегралы Лебега и Римана от $f(x)$ совпадают.\\
\textbf{Доказательство.} Римановское разбиение --- частный случай разбиения Лебега; точная верхняя грань подмножества не превосходит точной верхней грани всего множества, точная нижняя грань множества не превосходит точной нижней грани подмножества, поэтому 
$$
\underline{I }_R \leqslant \underline{I}_L \leqslant \overline{I}_L \leqslant \overline{I}_R.
$$
Интегрируемость функции $f(x)$ по Риману означает, что $\underline{I}_R = \overline{I}_R = I_R$. Из этого следует, что
$$
\underline{I}_L = \overline{I}_L = \underline{I}_R = \overline{I}_R = I_L = I_R. \quad \Box
$$

\textbf{Пример.} Рассмотрим следующую функцию:
$$
f(x)=\begin{cases}
0,&\text{если $x \in \mathbb{Q} \cap [0; 1]$}\\
1,&\text{если $x \in [0; 1] \setminus \mathbb{Q}$}
\end{cases}
$$
Она не интегрируема по Риману на сегменте $[0; 1]$, так как $\underline{I}_R = 0, \overline{I}_R = 1$. Разобьем сегмент $[0; 1]$ на два множества:
$$
E_1 = \mathbb{Q} \cap [0; 1], \quad E_2 = [0; 1] \setminus E_1
$$
Тогда $m_1 = M_1 = 0$, $m_2 = M_2 = 1$. Поэтому
$$
\left.
\begin{matrix}
S = \sum_{i=1}^2 M_i |E_i| = 1 \\
s = \sum_{i=1}^2 m_i |E_i| = 1
\end{matrix}
\right|
=>
I_L = 1.
$$
Таким образом, функция $f(x)$ не интегрируема по Риману на сегменте $[0; 1]$, но является интегрируемой по Лебегу на этом же сегменте.\\

\textbf{Теорема 2.} Любая ограниченная и измеримая на измеримом множестве $E$ конечной меры функция $f(x)$ интегрируема по Лебегу на этом множестве.\\
\textbf{Доказательство.} Положим $m = \inf\limits_E f(x)$, $M = \sup\limits_E f(x)$. С помощью точек $m = y_0 < y_1 < \dots < y_k = M$ разобьём сегмент $[m, M]$ на частичные полуинтервалы $(y_{k-1}, y_k] \ (k = 2,\ 3,\dots,\ n)$ и сегмент $[y_0, y_1]$ введём обозначение 
$$
\delta = \max_{1 \leqslant k \leqslant n} (y_k - y_{k-1}).
$$
Разобьём множество $E$ на сегменты
$$
E_1 = E[y_0 \leqslant f(x) \leqslant y_1],
$$
$$
E_k = E[y_{k-1} < f(x) \leqslant y_k], \; k = \overline{2,n}.
$$
Такое разбиение $T = \lbrace E_k \rbrace_{k=1}^n$ множества $E$ называется \textit{лебеговским разбиением} $E$. Верхняя и нижняя суммы $S_T$ и $s_T$, соответствующие лебеговскому разбиению $T$, называются \textit{лебеговскими верхней и нижней суммой}.\\

Заметим, что для любого номера $k$, $1 \leqslant k \leqslant n$, справедливы неравенства
$$
y_{k-1} \leqslant m_k \leqslant M_k \leqslant y_k.
$$
Получаем, что 
$$
0 \leqslant S_T - s_T = \sum_{k=1}^n (M_k - m_k) |E_k| \leqslant \sum_{k=1}^n (y_k - y_{k-1}) |E_k| \leqslant \delta |E|.
$$
Для любого разбиения $T$ справедливы неравенства
$$
s_T \leqslant \underline{I} \leqslant \overline{I} \leqslant S_T,
$$
поэтому
$$
0 \leqslant \overline{I} - \underline{I} \leqslant S_T - s_T \leqslant \delta |E|.
$$
В силу произвольности $\delta > 0$ из этого следует, что $\overline{I} = \underline{I} = I$. $\quad \Box$\\

Свойства интеграла Лебега:

\begin{enumerate}
\item 
$$
\int\limits_E 1 \ dx = |E|.
$$
Для доказательства достаточно заметить, что при $f(x) \equiv 1$ $s_T = S_T = |E|$ для любого разбиения $T$ множества $E$. 

\item Если функция $f(x)$ ограничена и интегрируема на множестве $E$ конечной меры и $\alpha$ --- произвольное вещественное число, то и функция $\alpha f(x)$ интегрируема на множестве $E$, причём
$$
\int\limits_E \alpha f(x) dx = \alpha \int\limits_E f(x) dx.
$$
\textbf{Доказательство.} Для произвольного разбиения $T = \lbrace E_k \rbrace$ множества $E$ обозначим верхнюю и нижнюю суммы функции $f(x)$ символами $S_T$ и $s_T$, а верхнюю и нижнюю суммы функции $\alpha f(x)$ --- символами $S_T^{(\alpha)}$ и $s_T^{(\alpha)}$. Тогда
$$
S_T^{(\alpha)}=\begin{cases}
\alpha S_T &\text{при $\alpha \geqslant 0$}\\
\alpha s_T &\text{при $\alpha < 0$}
\end{cases}, \quad
s_T^{(\alpha)}=\begin{cases}
\alpha s_T &\text{при $\alpha \geqslant 0$}\\
\alpha S_T &\text{при $\alpha < 0$}
\end{cases}.
$$
Обозначим через $\overline{I}$ и $\underline{I}$ верхний и нижний интегралы функции $f(x)$, а через $\overline{I}^{(\alpha)}$ и $\underline{I}^{(\alpha)}$ верхний и нижний интегралы функции $\alpha f(x)$. Тогда
$$
\overline{I}^{(\alpha)}=\begin{cases}
\alpha \overline{I} &\text{при $\alpha \geqslant 0$}\\
\alpha \underline{I} &\text{при $\alpha < 0$}
\end{cases}, \quad
\underline{I}^{(\alpha)}=\begin{cases}
\alpha \underline{I} &\text{при $\alpha \geqslant 0$}\\
\alpha \overline{I} &\text{при $\alpha < 0$}
\end{cases}.
$$
Так как $f(x)$ интегрируема на $E$, справедливо равенство
$$
\overline{I} = \underline{I} = \int\limits_E f(x) dx.
$$
А это значит, что
$$
\overline{I}^{(\alpha)} = \underline{I}^{(\alpha)} = \alpha \int\limits_E f(x) dx. \quad \Box
$$

\item Если функции $f_1(x)$ и $f_2(x)$ ограничены и интегрируемы по Лебегу на множестве конечной меры $E$, то функция $f_1(x) + f_2(x)$ интегрируема по Лебегу на множестве $E$, причём
$$
\int\limits_E [f_1(x) + f_2(x)] dx = \int\limits_E f_1(x) dx + \int\limits_E f_2(x) dx.
$$
\textbf{Доказательство.} Положим $f(x) = f_1(x) + f_2(x)$. Пусть $T = \lbrace E_k \rbrace$ --- произвольное разбиение множества $E$. Для функции $f(x)$ обозначим через $M_k$ и $m_k$ точные грани на множестве $E_k$, через $S_T$ и $s_T$ --- верхнюю и нижнюю суммы разбиения $T$, через $\overline{I}$ и $\underline{I}$ --- верхний и нижний интеграл Лебега. Аналогичные величины для функций $f_1(x)$ и $f_2(x)$ обозначим теми же символами, но с верхними индексами $(1)$ и $(2)$ соответственно.\\

Заметим, что точная верхняя грань суммы не больше суммы точных верхних граней слагаемых, а точная нижняя грань суммы не меньше суммы точных нижних граней слагаемых. Поэтому для любого номера $k$
$$
m_k^{(1)} + m_k^{(2)} \leqslant m_k \leqslant M_k \leqslant M_k^{(1)} + M_k^{(2)}.
$$
Значит, для любого разбиения $T$
$$
s_T^{(1)} + s_T^{(2)} \leqslant s_T \leqslant S_T \leqslant S_T^{(1)} + S_T^{(2)}.
$$
В свою очередь, это означает, что
$$
\underline{I}^{(1)} + \underline{I}^{(2)} \leqslant \underline{I} \leqslant \overline{I} \leqslant \overline{I}^{(1)} + \overline{I}^{(2)}.
$$
В силу интегрируемости функций $f_1(x)$ и $f_2(x)$ на множестве $E$
$$
\underline{I}^{(1)} = \overline{I}^{(1)} = \int\limits_E f_1(x) dx, \quad \underline{I}^{(2)} = \overline{I}^{(2)} = \int\limits_E f_2(x) dx.
$$
Из этого следует, что
$$
\underline{I} = \overline{I} = \int\limits_E f_1(x) dx + \int\limits_E f_2(x) dx.
$$
Это и означает справедливость доказываемого свойства. $\quad \Box$\\

\item Если множество $E$ представимо в виде $E = E_1 \cup E_2$, где $E_1$ и $E_2$ --- измеримые непересекающиеся множества конечной меры, функция $f(x)$ интегрируема по Лебегу на множествах $E_1$ и $E_2$, то $f(x)$ интегрируема по Лебегу и на множестве $E$, причём
$$
\int\limits_E f(x) dx = \int\limits_{E_1} f(x) dx + \int\limits_{E_2} f(x) dx.
$$
\textbf{Доказательство.} Заметим, что объединение произвольного разбиения $T_1$ множества $E_1$ и произвольного разбиения $T_2$ множества $E_2$ образует разбиение $T$ множества $E = E_1 \cup E_2$. Обозначим верхние суммы $f(x)$, отвечающие разбиениям $T_1$, $T_2$ и $T$, соответственно через $S_{T_1}$, $S_{T_2}$ и $S_T$, а нижние суммы $f(x)$, отвечающие разбиениям $T_1$, $T_2$ и $T$, соответственно через $s_{T_1}$, $s_{T_2}$ и $s_T$. Тогда
$$
S_T = S_{T_1} + S_{T_2}, \quad s_T = s_{T_1} + s_{T_2}.
$$
Обозначим верхний и нижний интегралы функции $f(x)$ на множестве $E_1$ через $\overline{I}^{(1)}$ и $\underline{I}^{(1)}$, на множестве $E_2$ --- через $\overline{I}^{(2)}$ и $\underline{I}^{(2)}$, на множестве $E$ --- через $\overline{I}$ и $\underline{I}$. Тогда
$$
\underline{I}^{(1)} + \underline{I}^{(2)} \leqslant \underline{I} \leqslant \overline{I} \leqslant \overline{I}^{(1)} + \overline{I}^{(2)}.
$$
В силу интегрируемости функции $f(x)$ на множествах $E_1$ и $E_2$
$$
\underline{I}^{(1)} = \overline{I}^{(1)} = \int\limits_{E_1} f(x) dx, \quad \underline{I}^{(2)} = \overline{I}^{(2)} = \int\limits_{E_2} f(x) dx.
$$
Из этого следует, что
$$
\underline{I} = \overline{I} = \int\limits_{E_1} f(x) dx + \int\limits_{E_2} f(x) dx.
$$
Это и означает справедливость доказываемого свойства. $\quad \Box$\\

\item
Если функции $f_1(x)$ и $f_2(x)$ ограничены и интегрируемы на множестве конечной меры $E$, и почти всюду на $E$ $f_1(x) \geqslant f_2(x)$, то
$$
\int\limits_E f_1(x) dx \geqslant \int\limits_E f_2(x) dx.
$$
\textbf{Доказательство.} При любом разбиении $T$ множества $E$ нижняя интегральная сумма функции $F(x) = f_1(x) - f_2(x)$ будет неотрицательна, поэтому $\underline{I} \geqslant 0$. В силу свойств 2 и 3 функция $F(x)$ интегрируема на $E$, причём
$$
\int\limits_E F(x) dx = \int\limits_E f_1(x) dx - \int\limits_E f_2(x) dx.
$$
Получаем, что
$$
\int\limits_E f_1(x) dx - \int\limits_E f_2(x) dx \geqslant 0,
$$
что и означает справедливость доказываемого свойства. $\quad \Box$\\
\end{enumerate}

\subsection*{4.2. Интеграл Лебега от неотрицательной измеримой функции на измеримом множестве конечной меры}
$$
|E| \leqslant + \infty, f(x) \geqslant 0
$$
Для любого $N > 0$ положим
$$
f_N(x) = min \lbrace N, f(x) \rbrace = \begin{cases}
f(x), &\text{если $f(x) \leqslant N$}\\
N, &\text{если $f(x) > N$}
\end{cases}.
$$
Функция $f_N(x)$ называется \textit{срезкой} функции $f(x)$. Заметим, что для любой измеримой на множестве $E$ функции $f(x)$ её срезка также будет измеримой, поскольку для любого вещественного $a$ является измеримым множество
$$
E[f_N(x) > a] = \begin{cases}
E[f(x) > a] &\text{при $a < N$}\\
\varnothing &\text{при $a \geqslant N$}
\end{cases}.
$$
Поэтому для любой измеримой на множестве $E$ функции $f(x)$ существует интеграл
$$
I_N = \int\limits_E f_N(x) dx.
$$

\textbf{Определение 2.} Если существует конечный предел $I = \lim\limits_{N \to \infty} I_N$, то функция $f(x)$ называется \textit{интегрируемой по Лебегу} на множестве конечной меры $E$, а указанный предел называется \textit{интегралом} от функции $f(x)$ по множеству $E$ и обозначается 
$$
I = \lim_{N\to + \infty} I_N = \int\limits_E f(x) dx.
$$

Убедимся в том, что неотрицательная интегрируемая на множестве $E$ функция $f(x)$ может обращатся в $+ \infty$ только на подмножестве $E_0 \subset E$, имеющем меру 0. Положим $E_0 = E[f(x) = + \infty]$. В силу свойств 4 и 5 предыдущего пункта выполняются неравенства
$$
I_N = \int\limits_E f_N(x) dx \geqslant \int\limits_{E_0} f_N(x) dx = \int\limits_{E_0} N dx \geqslant N |E_0|.
$$
Поскольку $f(x)$ интегрируема на множестве $E$, существует конечный предел $I = \lim\limits_{N \to \infty} I_N$, поэтому из записанных неравенств следует, что $|E_0| = 0$.\\

Отметим, что для неотрицательных интегрируемых функций справедливы свойства 2--5, установленные в пункте 4.1 для ограниченных неотрицательных интегрируемых функций (доказательства проводятся аналогично, с использованием функций срезки, которые являются ограниченными). \\

\textbf{Теорема 3 (о полной аддитивности).} Пусть $|E| < + \infty$, $f(x) \geqslant 0 $ и измерима на $E$, $E$ представимо в виде
$E = \bigcup\limits_{k=1}^\infty E_k, E_k \cap E_l = \varnothing$ при $k \neq l$. Тогда справедливы следующие два утверждения:\\
\begin{enumerate}
\item
Если $f(x)$ интегрируема на $E$, то $f(x)$ интегрируема на $E_k$ и справедливо равенство
$$
\int\limits_E f(x) dx = \sum_{k=1}^\infty \int\limits_{E_k} f(x) dx. \qquad (*)
$$
\item
Если $f(x)$ интегрируема на $E_k$ и ряд в правой части $(*)$ сходится, то $f(x)$ интегрируема на $E$ и $(*)$ выполняется.
\end{enumerate}
\textbf{Доказательство.}\\
а) Сначала докажем утверждения 1 и 2 для ограниченной неотрицательной интегрируемой функции $f(x)$. Пусть существует константа $M$ такая, что $|f(x)| \leqslant M$ всюду на $E$. Положим
$$
R_n = \bigcup\limits_{k =n+1}^\infty E_k, \quad \mbox{ тогда } |R_n| = \bigcup\limits_{k =n+1}^\infty |E_k|.
$$
Ряд
$$\bigcup\limits_{k =1}^\infty |E_k| = |E| \mbox{ --- сходится,}
$$
поэтому его остаток $|R_n| \to 0$ при $n \to \infty$.\\

Тогда на основании свойств 1, 4 и 5
$$
\int_E f(x) dx - \sum_{k=1}^n \int\limits_{E_k} f(x) dx = \int\limits_{R_n} f(x) dx \leqslant M \int\limits_{R_n} dx \leqslant M |R_n| \to 0 \mbox{ при } n \to \infty.
$$
Это и означает правильность утверждений 1 и 2 в случае ограниченной $f(x)$.\\

б) Пусть теперь $f(x)$ --- произвольная неотрицательная интегрируемая функция. Суммируемость $f(x)$ на каждом из множеств $E_k$ напрямую следует из неравенства
$$
\int\limits_{E_k} f_N(x) dx \leqslant \int\limits_E f_N(x) dx
$$
и неубывания по $N$ интеграла в левой части этого неравенства. Заметим, что функция срезки $f_N(x)$ является ограниченной, поэтому в силу пункта~а) 
$$
\int\limits_E f_N(x) dx = \sum_{k=1}^\infty \int\limits_{E_k} f_N(x) dx \leqslant \sum_{k=1}^\infty \int\limits_{E_k} f(x) dx.
$$
Переходя в этом неравенстве к пределу при $N \to \infty$, получим
$$
\int\limits_E f(x) dx \leqslant \sum_{k=1}^\infty \int\limits_{E_k} f(x) dx.
$$
С другой стороны, для любого номера $m$
$$
\int\limits_E f_N(x) dx = \sum_{k=1}^\infty \int\limits_{E_k} f_N(x) dx \geqslant \sum_{k=1}^m \int\limits_{E_k} f_N(x) dx.
$$
Последовательно переходя к пределу сначала при $N \to \infty$, а затем при $m \to \infty$, получим неравенство
$$
\int\limits_E f dx \geqslant \sum_{k=1}^\infty \int\limits_{E_k} f(x) dx
$$
Из двух полученных нами неравенств следует, что
$$
\int\limits_E f dx = \sum_{k=1}^\infty \int\limits_{E_k} f(x) dx,
$$
что и доказывает правильность утверждения 1.\\

Правильность утверждения 2 следует из неравенства для функции $f_N(x)$:
$$
\int\limits_E f_N(x) dx = \sum_{k=1}^\infty \int\limits_{E_k} f_N(x) dx \leqslant \sum_{k=1}^\infty \int\limits_{E_k} f(x) dx.
$$
Так как ряд в правой части этого неравенства сходится, функция $f(x)$ будет являться суммируемой на множестве $E$, а следовательно, для неё будет выполняться равенство $(*)$. $\quad \Box$\\

\textbf{Теорема 4 (об абсолютной непрерывности интеграла Лебега).} Пусть $|E| < + \infty$, $f(x)$ --- неотрицательная, интегрируемая на множестве $E$ по Лебегу функция. Тогда для любого $\varepsilon > 0$ найдётся число $\delta > 0$ такое, что для любого подмножества $e \subset E$, $|e| < \delta$, будет выполняться неравенство
$$
\int\limits_e f(x) dx < \varepsilon.
$$
\textbf{Доказательство.}\\
а) Сначала проведём доказательство в случае, когда функция $f(x)$ ограничена, то есть существует константа $M$ такая, что $|f(x)| \leqslant M$ всюду на $E$. Тогда
$$
\int\limits_e f(x) dx \leqslant M \int\limits_e dx = M |e| < M \delta < \varepsilon \mbox{ при } \delta < \frac \varepsilon M.
$$
б) Пусть теперь $f(x)$ --- произвольная неотрицательная, интегрируемая на $E$ функция. В силу интегрируемости для любого $\varepsilon > 0$ найдётся число $N = N(\varepsilon)$ такое, что
$$
\int\limits_E (f(x) - f_N(x)) dx < \frac \varepsilon 2.
$$
Но тогда
\begin{multline*}
\int\limits_e f(x) dx = \int\limits_e (f(x) - f_N(x)) dx + \int\limits_e f_N(x) dx < \\
< \frac \varepsilon 2 + N \int\limits_e dx = \frac \varepsilon 2 + N |e| < \frac \varepsilon 2 + N \delta < \varepsilon \quad \mbox{ при } \delta < \frac \varepsilon {2 N(\varepsilon)}.
\end{multline*}
Теорема доказана. $\quad \Box$\\

\textbf{Теорема 5.} Пусть множество $E$ имеет конечную меру, функция $f(x)$ неотрицательна и интегрируема по Лебегу на $E$, а $\int\limits_E f(x) dx = 0$. Тогда функция $f(x)$ эквивалентна тождественному нулю (то есть множество, на котором $f(x) \neq 0$, имеет меру 0).\\
\textbf{Доказательство.} Для любого $a > 0$ положим $E_a = E[f > a]$. Тогда
$$
\int\limits_E f(x) dx \geqslant \int\limits_{E_a} f(x) dx \geqslant a |E_a|.
$$
Следовательно, для любого $a > 0$
$$
|E_a| \leqslant \frac 1 a \int\limits_E f(x) dx = 0 \quad \Rightarrow \quad |E_a| = 0.
$$
Заметим, что
$$
E[f > 0] = \bigcup_{k=1}^\infty E \left[f > \frac 1 k \right]
$$
Поэтому
$$
|E[f(x) > 0]| \leqslant \sum_{k=1}^\infty \left| E \left[f > \frac 1 k \right]\right| = 0 \quad \Rightarrow \quad |E[f(x) > 0]| = 0.
$$
Теорема доказана. $\quad \Box$\\

\textbf{Теорема 6.} Пусть множество $E$ имеет конечную меру, $f_1(x)$ и $f_2(x)$ --- неотрицательные, измеримые на $E$ функции и $f_1(x) \geqslant f_2(x)$. Тогда, если функция $f_1$ интегрируема по Лебегу на $E$, то и $f_2$ интегрируема по Лебегу на $E$ и 
$$
\int\limits_E f_2(x) dx \leqslant \int\limits_E f_1(x) dx
$$
\textbf{Доказательство.} Заметим, что
$$
\int_E f_{2_N}(x) dx \leqslant \int_E f_{1_N}(x) dx \leqslant \int_E f_1(x) dx.
$$
Интеграл в левой части неравенства является неубывающим по $N$, поэтому функция $f_2(x)$ интегрируема на $E$. Справедливость неравенства
$$
\int\limits_E f_2(x) dx \leqslant \int\limits_E f_1(x) dx
$$
напрямую следует из свойства 5.  $\quad \Box$\\






%%%%%%%%%%%%%%%%%%%%%%%
%      Lecture 5      %
%%%%%%%%%%%%%%%%%%%%%%%






\subsection*{4.3. Интеграл Лебега для неограниченной функции любого знака}

Рассматриваем измеримое множество $E$ конечной меры и измеримую функцию $f(x)$, не являющуюся, вообще говоря, ограниченной на множестве $E$ и принимающую на этом множестве значения любых знаков. Введём в рассмотрение две неотрицательные функции
$$
f^+(x) = \frac 1 2 (|f(x)| + f(x)), \quad f^-(x) = \frac 1 2 (|f(x)| - f(x)).
$$
Очевидно, что
$$
f^+(x) + f^-(x) = |f(x)|, \quad f^+(x) - f^-(x) = f(x).
$$

\textbf{Определение 1.} Функция $f(x)$ называется \textit{интегрируемой} на множестве $E$, если на этом множестве интегрируемы функции $f^+(x)$, $f^-(x)$. При этом \textit{интегралом Лебега} от функции $f(x)$ по множеству $E$ называется
$$
\int\limits_E f(x) dx = \int\limits_E f^+ dx - \int\limits_E f^- dx.
$$

\textbf{Определение.} Совокупность всех интегрируемых на множестве $E$ функций обозначают символом $L(E)$ или $L^1(E)$. Запись $f(x) \in L(E)$ ($f(x) \in L^1(E)$) означает, что функция $f(x)$ измерима и интегрируема на множестве $E$.

Метрика --- интеграл модуля разности. Сходимость определеняется по ней.

Из сходимости в $L(E)$ вытекает сходимость по мере. Обратное неверно: $f_n(x) = \begin{cases}
n, &\text{ если $x \in \left[0, \dfrac 1 n \right]$} \medskip\\
0, &\text{ если $x \in \left(\dfrac 1 n, 1 \right]$}.
\end{cases}$

% А вообще это было на месте определения 2.\\

\textbf{Утверждение.} Измеримая на множестве $E$ функция $f(x)$ интегрируема на $E$ тогда и только тогда, когда функция $|f(x)|$ интегрируема на этом множестве.\\
\textbf{Доказательство.}\\
Необходимость:
$$
f(x) \in L(E) \quad \Rightarrow \quad f^+(x), f^-(x) \in L(E) \quad \Rightarrow \quad f^+(x) + f^-(x) = |f(x)| \in L(E).
$$
Достаточность: пусть функция $|f(x)| \in L(E)$. Так как функции $f^+(x) < |f(x)|$ и $f^-(x) < |f(x)|$, то в силу теоремы 6 пункта 4.3 $f^+(x), f^-(x) \in L(E)$. Следовательно, $f^+(x) - f^-(x) = f(x) \in L(E)$. $\quad \Box$\\

\textbf{Пример.} Рассмотрим функцию $f(x) = \frac {\sin x} x$ на множестве $[0, 1]$. Как известно, $\int\limits_0^1 \frac {\sin x} x dx$ сходится условно. Поэтому $\int\limits_E |f(x)| dx$ не существует. Следовательно, функция $f(x)$ не интегрируема по Лебегу на множестве $E$.\\

Для неограниченных интегрируемых функций произвольного знака справедливы свойства 2--5, установленные в пункте 4.1 для ограниченных неотрицательных интегрируемых функций (доказательства проводятся аналогично, с использованием функций $f^+(x)$ и $f^-(x)$, которые являются неотрицательными, и для которых свойства 2-5 тоже верны).\\

\textbf{Теорема 7 (о полной аддитивности).} Пусть множество $E$ представимо в виде $E = \bigcup\limits_{k=1}^\infty E_k,$ множества $E_k$ измеримы и $E_k \cap E_l = \varnothing$ при $k \neq l$. Тогда справедливы следующие два утверждения:\\
\begin{enumerate}
\item
Если $f(x)$ интегрируема на множестве $E$, то $f(x)$ интегрируема и на каждом из множеств $E_k$, причём справедливо равенство
$$
\int\limits_E f(x) dx = \sum_{k=1}^\infty \int\limits_{E_k} f(x) dx. \qquad (*)
$$
\item
Если функция $f(x)$ измерима и интегрируема на каждом из множеств $E_k$ и сходится ряд $\sum\limits_{k=1}^\infty \int\limits_{E_k} |f(x)| dx$, то $f(x)$ интегрируема на $E$ и выполняеся равенство $(*)$.
\end{enumerate}
\textbf{Доказательство.}
Если функция $f(x)$ интегрируема на множестве $E$, то по определению неотрицательные функции $f^+(x)$ и $f^-(x)$ также интегрируемы на $E$, а следовательно, к ним применима теорема 3. Поэтому функции $f^+(x)$ и $f^-(x)$ являются интегрируемыми на каждом из множеств $E_k$ и для них справедливы равенства
$$
\int\limits_E f^+(x) dx = \sum_{k=1}^\infty \int\limits_{E_k} f^+(x) dx, \quad \int\limits_E f^-(x) dx = \sum_{k=1}^\infty \int\limits_{E_k} f^-(x) dx. 
$$
Тогда по определению функция $f(x)$ интегрируема на каждом из множеств $E_k$ и справедливо равенство
\begin{multline*}
\int\limits_E f(x) dx = \int\limits_E f^+ dx - \int\limits_E f^- dx = \\
= \sum_{k=1}^\infty \int\limits_{E_k} f^+(x) dx - \sum_{k=1}^\infty \int\limits_{E_k} f^-(x) dx = \\
= \sum_{k=1}^\infty \int\limits_{E_k} (f^+(x) - f^-(x)) dx = \sum_{k=1}^\infty \int\limits_{E_k} f(x) dx.
\end{multline*}
Таким образом, мы доказали справедливость первой части теоремы.\\

Докажем вторую часть теоремы. Так как функция $f(x)$ измерима и интегрируема на каждом из множеств $E_k$, то в силу доказанного выше утверждения функция $|f(x)|$ также интегрируема на каждом из множеств $E_k$. Тогда, поскольку ряд $\sum\limits_{k=1}^\infty \int\limits_{E_k} |f(x)| dx$ сходится, для функции $|f(x)|$ справедливо второе утверждение теоремы 3. Следовательно, $|f(x)|$ интегрируема на всём множестве $E$. Тогда в силу доказанного выше утверждения и функция $f(x)$ интегрируема на всём $E$, а следовательно, справедливо равенство $(*)$. $\quad \Box$\\

\textbf{Теорема 8 (об абсолютной непрерывности).} Если функция $f(x)$ интегрируема на множестве $E$, то для любого $\varepsilon > 0$ найдётся число $\delta > 0$ такое, что для любого измеримого подмножества $e \subset E$, $|e| < \delta$, будет выполняться неравенство
$$
\left| \int\limits_e f(x) dx \right| < \varepsilon.
$$
\textbf{Доказательство.} Так как функция $f(x)$ интегрируема на множестве $E$, неотрицательная функция $|f(x)|$ также интегрируема на $E$. Тогда к $|f(x)|$ применима теорема 4 и справедливо неравенство $\int\limits_e |f(x)| dx < \varepsilon$. Следовательно,
$$
\left| \int\limits_e f(x) dx \right| \leqslant \int\limits_e |f(x)| dx < \varepsilon. \quad \Box
$$

\textbf{Определение 2.} Говорят, что последовательность интегрируемых на множестве $E$ функций $\lbrace f_n(x) \rbrace$ сходится к интегрируемой на $E$ функции $f(x)$ в $L(E)$, если
$$
\lim_{n\to\infty} \int\limits_E |f_n(x) - f(x)| dx = 0.
$$
\textbf{Замечание 1.} Из определения непосредственно следует, что
$$
\lim_{n\to\infty} \int\limits_E f_n(x) dx = \int\limits_E f(x)dx. \quad (**)
$$

\textbf{Замечание 2.} Если последовательность измеримых и интегрируемых на множестве $E$ функций $\lbrace f_n(x) \rbrace$ сходится к измеримой и интегрируемой на $E$ функции $f(x)$ в $L(E)$, то $\lbrace f_n(x) \rbrace$ сходится к $f(x)$ и по мере на $E$.\\
\textbf{Доказательство.} Для любого $\varepsilon > 0$ положим
$$
E_n =  E[|f(x) - f_n(x)| \geqslant \varepsilon].
$$
Тогда
$$
\int\limits_E |f_n(x) - f(x)| dx \geqslant \int\limits_{E_n} |f_n(x) - f(x)| dx \geqslant \varepsilon |E_n|.
$$
Следовательно, $|E_n| \to 0$ при $n\to \infty$, что и означает сходимость $\lbrace f_n(x) \rbrace$ к $f(x)$ по мере на $E$. $\quad \Box$\\

\textbf{Пример.} Рассмотрим последовательность $\{f_n(x)\}$, где
$$
f(x) = \begin{cases}
n, &\text{ если $x \in \left[0, \dfrac 1 n \right]$} \medskip\\
0, &\text{ если $x \in \left(\dfrac 1 n, 1 \right]$}.
\end{cases}
$$
Поскольку $\lim\limits_{n\to\infty} |E[|f_n(x)-0| \geqslant \varepsilon]| = 0$, то $\lbrace f_n(x) \rbrace$ сходится к $f(x) \equiv 0$ по мере на множестве $E = [0, 1]$. С другой стороны,
$$
\forall n \; \int\limits_E f_n(x) dx = 1, \quad \int\limits_E 0 \ dx = 0,
$$
поэтому сходимости $\lbrace f_n(x) \rbrace$ к $f(x) \equiv 0$ в $L(E)$ нет. Однако при некоторых дополнительных условиях из сходимости по мере на $E$ всё-таки следует сходимость в $L(E)$, что доказывает следующая теорема.\\

\textbf{Теорема 9 (теорема Лебега).} Если последовательность измеримых на множестве $E$ функций $\lbrace f_n(x) \rbrace$ сходится к измеримой на $E$ функции $f(x)$ по мере на $E$ и существует интегрируемая на множестве $E$ функция $F(x)$ такая, что для всех номеров $n$ и почти всех точек множества $E$ справедливо неравенство $|f_n(x)| \leqslant F(x)$, то последовательность $\lbrace f_n(x) \rbrace$ сходится к функции $f(x)$ в $L(E)$.\\
\textbf{Доказательство.}
В силу теоремы 6 параграфа 3 из последовательности $\lbrace f_n(x) \rbrace$ можно выделить подпоследовательность $\lbrace f_{n_k}(x) \rbrace$ $( k = 1, \ 2, \dots)$, сходящуюся к $f(x)$ почти всюду на $E$. Тогда, переходя в неравенстве $|f_{n_k}(x)| \leqslant F(x)$ к пределу при $k \to \infty$, получим, что для почти всех точек $E$ справедливо неравенство $|f(x)| \leqslant F(x)$. Значит, почти всюду на $E$ справедливо и неравенство $f(x) \leqslant F(x)$, а следовательно, в силу теоремы 6 функция $f(x)$ интегрируема на множестве $E$.\\

Для произвольного $\varepsilon > 0$ положим
$$
E_n =  E[|f(x) - f_n(x)| \geqslant \varepsilon].
$$
Тогда
\begin{multline*}
\int\limits_E |f(x) - f_n(x)| dx = \\
= \int\limits_{E_n} |f(x) - f_n(x)| dx + \int\limits_{E \setminus E_n} |f(x) - f_n(x)| dx \leqslant \\
\leqslant \int\limits_{E_n} 2F(x) dx + \varepsilon |E|.
\end{multline*}
Последовательность $\lbrace f_n(x) \rbrace$ сходится к $f(x)$ по мере на $E$, поэтому $|E_n| \to 0$ при $n \to \infty$. Значит, в силу теоремы 8 для любого $\theta > 0$
$$
\left| \int\limits_{E_n} F(x) dx \right| < \theta \quad \mbox{при } n \to \infty,
$$
то есть $\lim\limits_{n \to \infty} \int\limits_{E_n} F(x) dx = 0$. Второе слагаемое также можно устремить к 0 в силу произвольности $\varepsilon$. Таким образом, $\int\limits_E |f(x) - f_n(x)| dx \to 0$ при $\varepsilon \to 0$ и $n \to \infty$. Из этого следует, что последовательность $\lbrace f_n(x) \rbrace$ сходится к функции $f(x)$ в $L(E)$. $\quad \Box$\\

\textbf{Следствие.} Если последовательность измеримых на множестве $E$ функций $\lbrace f_n(x) \rbrace$ сходится к функции $f(x)$ почти всюду на $E$ и существует интегрируемая на множестве $E$ функция $F(x)$ такая, что для всех номеров $n$ и почти всех точек множества $E$ справедливо неравенство $|f_n(x)| \leqslant F(x)$, то функция $f(x)$ суммируема на $E$ и справедливо равенство (**).\\
\textbf{Доказательство.}
Так как последовательность измеримых функций $\lbrace f_n(x) \rbrace$ сходится к $f(x)$ почти всюду на $E$, то в силу теоремы 4 параграфа 3 функция $f(x)$ также будет измерима на $E$. Следовательно, по теореме 5 параграфа 3 из сходимости $\lbrace f_n(x) \rbrace$ к $f(x)$ почти всюду на $E$ следует сходимость $\lbrace f_n(x) \rbrace$ к $f(x)$ по мере на $E$. Но тогда в силу теоремы 9 функция $f(x)$ суммируема на $E$, $\lbrace f_n(x) \rbrace$ сходится к $f(x)$ по мере на $E$ и выполняется требуемое равенство (**). $\quad \Box$\\

\textbf{Теорема 10 (теорема Леви).} Пусть $\lbrace f_n(x) \rbrace$ - последовательность измеримых и интегрируемых на множестве $E$ функций, и пусть для всех номеров $n$ и для почти всех точек множества $E$ справедливо неравенство $f_n(x) \leqslant f_{n+1}(x)$. Пусть существует константа $M$ такая, что для всех номеров $n$ справедливо неравенство $\left| \int\limits_E f_n(x) dx \right| \leqslant M$. Тогда для почти всех точек $x \in E$ существует конечный предел $\lim\limits_{n\to\infty} f_n(x) = f(x)$, причём предельная функция $f(x)$ суммируема на множестве $E$ и справедливо равенство (**).\\
\textbf{Доказательство.} Не ограничивая общности, будем считать, что все $f_n(x) \geqslant 0$ почти всюду на $E$ (в противном случае вместо функций $f_n(x)$ можно рассматривать функции $g_n(x) = f_n(x) - f_1(x)$, которые по условию будут являться неотрицательными для почти всех точек $E$).\\

Так как последовательность $\lbrace f_n(x) \rbrace$ почти всюду на $E$ не убывает, то почти во всех точках $E$ определена предельная функция $f(x)$, которая принимает в этих точках либо конечные значения, либо равна $=\infty$. Если мы докажем, что $f(x)$ интегрируема на $E$, то из этого будет следовать, что $f(x)$ является конечной почти всюду на $E$, а следовательно, почти всюду на $E$ будет существовать конечный предел $\lim\limits_{n\to\infty} f_n(x) = f(x)$ и выполняться равенство (**).\\

Итак, для доказательства теоремы достаточно установить интегрируемость предельной функции $f(x)$ на множестве $E$.\\

Заметим, что для любого $N > 0$ последовательность $\lbrace (f_n)_N(x) \rbrace$ почти всюду на $E$ сходится к функции $(f)_N(x)$, причём для всех номеров $n$ и почти всех точек $E$ справедливо неравенство $(f_n)_N(x) \leqslant (f)_N(x)$. Кроме того, функция $(f_n)_N(x)$ является измеримой и ограниченной, а следовательно, и интегрируемой на множестве $E$. Поэтому применимо следствие из теоремы 9, в силу которого
$$
\lim_{n\to\infty} \int\limits_E (f_n)_N(x) dx = \int\limits_E (f)_N(x) dx.
$$
Из этого соотношения и очевидного неравенства
$$
\int\limits_E f_n(x) dx \geqslant \int\limits_E (f_n)_N(x)dx
$$
заключаем, что
$$
\lim\limits_{n\to\infty} \int\limits_E f_n(x) dx \geqslant \int\limits_E (f)_N(x) dx.
$$
Кроме того, по условию существует такая константа $M$, что для всех номеров $n$
$$
\int\limits_E f_n(x) dx \leqslant M.
$$
Следовательно, и
$$
\int\limits_E (f)_N(x) dx \leqslant M.
$$
Интеграл в левой части этого неравенства является неубывающим по $N$, поэтому существует конечный предел
$$
\lim\limits_{N\to\infty} \int\limits_E (f)_N(x) dx = f(x),
$$
а это и означает, что функция $f(x)$ интегрируема на множестве $E$. $\quad \Box$\\


\textbf{Следствие (формулировка теоремы Леви в терминах функциональных рядов).} Если каждая функция $u_n(x)$ неотрицательна почти всюду на множестве $E$, измерима и интегрируема на этом множестве, и если сходится ряд
$$
\sum_{n=1}^\infty \int\limits_E u_n(x) dx,
$$
то почти всюду на $E$ сходится ряд
$$
\sum_{n=1}^\infty u_n(x),
$$
причём сумма $S(x)$ этого ряда интегрируема на множестве $E$ и удовлетворяет условию
$$
\int\limits_E S(x) dx = \sum_{n=1}^\infty \int\limits_E u_n(x) dx.
$$

\textbf{Теорема 11 (теорема Фату).} Если последовательность измеримых и интегрируемых на множестве $E$ функций $\lbrace f_n(x) \rbrace$ сходится почти всюду на $E$ к предельной функции $f(x)$ и если существует константа $A$ такая, что для всех номеров $n$ справедливо неравенство $\int\limits_E |f_n(x)| dx \leqslant A$, то предельная функция $f(x)$ интегрируема на множестве $E$ и для неё справедливо неравенство $\int\limits_E |f(x)| dx \leqslant A$.\\
\textbf{Доказательство.} Введём в рассмотрение функции
$$
g_n(x) = \inf_{k \geqslant n} |f_k(x)|.
$$
Заметим, что последовательность $\lbrace g_n(x) \rbrace$ является неубывающей и почти всюду на $E$ сходится к $|f(x)|$, а каждая функция $g_n(x)$ неотрицательна и является измеримой в силу теоремы 3 параграфа 3. Кроме того, для любого $n$ справедливо неравенство $g_n(x) \leqslant |f_n(x)|$, из которого в силу теоремы 6 следует интегрируемость функций $g_n(x)$ на множестве $E$. Наконец, справедливо неравенство
$$
\int\limits_E g_n(x) dx \leqslant \int\limits_E |f_n(x)| dx \leqslant A.
$$
Получаем, что к последовательности $\lbrace g_n(x) \rbrace$ применима теорема 10. Следовательно, предельная функция $|f(x)|$ интегрируема (откуда сразу следует интегрируемость функции $f(x)$) и выполняется равенство
$$
\lim_{n\to\infty} \int\limits_E g_n(x) dx = \int\limits_E |f(x)| dx.
$$
Так как для любого номера $n$
$$
\int\limits_E g_n(x) dx \leqslant A,
$$
то верно и неравенство
$$
\int\limits_E |f(x)| dx \leqslant A.
$$
Таким образом, теорема полностью доказана. $\quad \Box$\\






%%%%%%%%%%%%%%%%%%%%%%%
%      Lecture 6      %
%%%%%%%%%%%%%%%%%%%%%%%






\textbf{Теорема 12 (теорема Лебега).} Пусть измеримое множество $E$ имеет конечную меру. Для того, чтобы ограниченная функция $f(x)$ была интегрируема на множестве $E$ по Лебегу, необходимо и достаточно, чтобы она была измерима.\\
\textbf{Доказательство.}
Достаточность доказана в теореме 2. Докажем необходимость.\\

Пусть функция $f(x)$ ограничена и интегрируема по Лебегу на измеримом множестве $E$. Следовательно, её верхний и нижний интегралы Лебега совпадают, а это значит, что существует последовательность разбиений $T_n = \lbrace E_k^{(n)} \rbrace$ множества $E$ такая, что соответствующие последовательности верхних $\lbrace S_n \rbrace$ и нижних $\lbrace s_n \rbrace$ сумм удовлетворяют условию $S_n - s_n < \frac 1 n$, причём каждое последующее разбиение $T_{n+1} = \lbrace E_k^{(n+1)} \rbrace$ является измельчением предыдущего разбиения $T_n = \lbrace E_k^{(n)} \rbrace$. (Для построения такой последовательности разбиений достаточно там, где это необходимо, брать произведение вводимых разбиений.)\\

По определению
$$
\begin{matrix}
S_n = \sum\limits_{k=1}^{m(n)} M_k^{(n)} |E_k^{(n)}|, &&\text{где } M_k^{(n)} = \sup\limits_{E_k^{(n)}} f(x); \\
s_n = \sum\limits_{k=1}^{m(n)} m_k^{(n)} |E_k^{(n)}|, &&\text{где } m_k^{(n)} = \inf\limits_{E_k^{(n)}} f(x).
\end{matrix}
$$
Определим две последовательности функций $\lbrace \overline{f}{}_n(x) \rbrace$ и $\lbrace \underline{f}{}_n(x) \rbrace$, где
$$
\begin{matrix}
\overline{f}{}_n(x) = M_k^{(n)} \mbox{на } E_k^{(n)}, \\
\underline{f}{}_n(x) = m_k^{(n)} \mbox{на } E_k^{(n)}.
\end{matrix}
$$
Для каждого номера $n$ обе функции $\overline{f}{}_n(x)$ и $\underline{f}{}_n(x)$ измеримы на множестве $E$, так как они представляют собой линейные комбинации характеристических функций измеримых множеств $E_k^{(n)}$. Кроме того, последовательность $\lbrace \overline{f}{}_n(x) \rbrace$ не возрастает, а последовательность $\lbrace \underline{f}{}_n(x) \rbrace$ не убывает на множестве $E$, причём для любого номера $n$ в каждой точке множества $E$ справедливы неравенства
$$
\underline{f}{}_n(x) \leqslant f(x) \leqslant \overline{f}{}_n(x).
$$
Положим
$$
\overline{f}(x) = \lim_{n\to\infty} \overline{f}{}_n(x), \quad \underline{f}(x) = \lim_{n\to\infty} \underline{f}{}_n(x),
$$
тогда в каждой точке множества $E$
$$
\underline{f}(x) \leqslant f(x) \leqslant \overline{f}(x).
$$
Из теоремы 10 (Леви) получаем, что
$$
\lim_{n\to\infty} \int\limits_E [\overline{f}{}_n(x) - \underline{f}{}_n(x)] dx = \int\limits_E [\overline{f}(x) - \underline{f}(x)] dx.
$$
С другой стороны, из определения функций $\overline{f}{}_n(x)$ и $\underline{f}{}_n(x)$ вытекает, что
$$
\int\limits_E \overline{f}{}_n(x) dx = S_n, \quad \int\limits_E \underline{f}{}_n(x) dx = s_n,
$$
причём по построению $\lim\limits_{n\to\infty} (S_n - s_n) = 0$. Следовательно,
$$
\int\limits_E [\overline{f}(x) - \underline{f}(x)] dx = 0.
$$
Кроме того, функция $[\overline{f}(x) - \underline{f}(x)]$ ограничена и измерима, а значит, и интегрируема на множестве $E$. Кроме того, эта функция ещё и неотрицательна, поэтому в силу теоремы 5 $\overline{f}(x) - \underline{f}(x) = 0$ почти всюду на $E$. Следовательно, $\overline{f}(x) = f(x) = \underline{f}(x)$ почти всюду на $E$, и поэтому из измеримости функций $\overline{f}{}_n(x)$ и $\underline{f}{}_n(x)$ вытекает измеримость функции $f(x)$ на множестве $E$. $\quad \Box$\\


\subsection*{4.4. Случай $|E| = +\infty$}

Мы рассматриваем случай, когда множество $E$ имеет бесконечную меру, но может быть представлено в виде суммы счётного числа множеств конечной меры (в таком случае говорят, что мера множества $E$ является \textit{$\sigma$-конечной}).\\

\textbf{Определение 1.} Говорят, что последовательность множеств $\lbrace E_n \rbrace$ \textit{исчерпывает множество $E$} с $\sigma$-конечной мерой, если для каждого номера $n$ $|E_n| < + \infty$, $E_n \subset E_{n+1}$ и $\bigcup\limits_{n=1}^\infty E_n = E$.

\textbf{Определение 2.} Измеримая функция $f(x)$, определённая на множестве $E$ с $\sigma$-конечной мерой, называется \textit{интегрируемой} на $E$, если она интегрируема на каждом измеримом подмножестве $A \subset E$ конечной меры и если для каждой последовательности $\lbrace E_n \rbrace$, исчерпывающей множество $E$, предел
$$
I = \lim_{n\to\infty} \int\limits_{E_n} f(x) dx
$$
существует и не зависит от выбора этой последовательности. Тогда $I$ называется \textit{интегралом Лебега от $f(x)$ по множеству $E$} и обозначается символом $I = \int\limits_E f(x) dx$.\\

\textbf{Теорема 1 (теорема Фубини).} Пусть функция $f(x, y)$ интегрируема на $\Pi = \lbrace (x, y): a \leqslant x \leqslant b, c \leqslant y \leqslant d \rbrace$. Тогда для почти всех $y \in [c, d]$ существует $\int\limits_a^b f(x, y) dx$, для почти всех $x \in [a,b]$ существует $\int\limits_c^d f(x, y) dy$ и
$$
\iint\limits_\Pi f(x, y) dx dy = \int\limits_c^d dy \int\limits_a^b f(x, y) dx = \int\limits_a^b dx \int\limits_c^d f(x, y) dy.
$$

\textbf{Замечание.} Обратное, вообще говоря, неверно.\\

Пример --- функция
$
f(x, y) =
\begin{cases}
\dfrac {xy}{(x^2 + y^2)^2} &\text{вне нуля} \\
0 &\text{при $x=y=0$}
\end{cases}
$
на множестве $K = [-1; 1] \times [-1; 1]$.


\section*{\S 5. Пространство $L_p$, $p \geqslant 1$.}

Рассматриваем случай, когда $E$ --- измеримое множество.\\

\textit{Линейным (векторным пространством)} над полем $P$ называется непустое множество $L$, на котором введены следующие операции:
\begin{enumerate}
\item операция сложения: каждой паре элементов $x, y$ множества $L$ ставится в соответствие элемент $L$, обозначаемый $x + y$;
\item операция умножения на скаляр (элемен поля $P$): любому элементу $\lambda \in P$ и любому элементу $x \in L$ ставится в соответствие элемент $L$, обозначаемый $\lambda x$.
\end{enumerate}

При этом должны выполняться следующие условия:
\begin{enumerate}
\item $x + y = y + x \quad \forall x, y \in L$;
\item $x + (y + z) = (x + y) + z \quad \forall x, y, z \in L$;
\item $\exists \theta \in L : \; x + \theta = x \quad \forall x \in L$;
\item $\forall x \in L \; \exists (-x) \in L : \; x + (-x) = \theta$;
\item $\alpha (\beta x) = (\alpha \beta) x \quad \forall x \in L$;
\item $1 \cdot x = x \quad \forall x \in L$;
\item $(\alpha + \beta) x = \alpha x + \beta x \quad \forall x \in L$;
\item $\alpha (x + y) = \alpha x + \alpha y \quad \forall x, y \in L$.
\end{enumerate}

В дальнейшем будем рассматривать пространства над полем действительных чисел $\mathbb{R}$.\\

Линейное пространство $L$ называется \textit{нормированным}, если любому элементу $f \in L$ ставится в соответствие действительное число (называемое \textit{нормой} этого элемента и обозначаемое символом $\|f\|_L$), и при этом выполняются следующие условия (\textit{аксиомы}):
\begin{enumerate}
\item $\forall f \in L \quad \|f\| \geqslant 0, \qquad \|f\| = 0 \leftrightarrow f = 0;$

\item $\forall f \in L, \; \forall a \in \mathbb{R} \quad \|a \cdot f\| = |a| \|f\|;$

\item $\forall f, g \in L \quad \|f + g\| \leqslant \|f\| + \|g\|.$
\end{enumerate}

\textbf{Определение 1.} Говорят, что функция $f(x)$ принадлежит пространству $L_p(E)$, если $f(x)$ измерима на множестве $E$, а функция $|f(x)|^p$ интегрируема на $E$.\\

Введём в пространстве $L_p(E)$ норму с помощью следующим образом:
$$
\|f\|_{L_p(E)} = \|f\|_p = \left( \int\limits_E |f(x)|^p dx \right)^{\frac 1 p}.
$$
Линейность этого пространства очевидна; докажем, что оно является нормированным.\\

Рассмотрим первую аксиому: неотрицательность введённой нормы очевидна, равно как и справедливость выражения $f = 0 \Rightarrow \|f\| = 0$; справедливость выражения $\|f\| = 0 \Rightarrow f = 0$ следует из теоремы 5 параграфа 4. Таким образом, первая аксиома выполняется. Справедливость аксиомы 2 также очевидна. Остаётся доказать, что в пространстве $L_p(E)$ выполняется аксиома 3 (так называемое \textit{неравенство треугольника}). Перед этим докажем несколько вспомогательных утверждений.\\

\textbf{Неравенство Юнга.} Пусть числа $p, q > 0$ и связаны соотношением $\dfrac 1 p + \dfrac 1 q = 1$. Тогда для любых чисел $a \geqslant 0$ и $b > 0$ выполняется неравенство
$$
a^{\dfrac 1 p} \cdot b^{\dfrac 1 q} \leqslant \frac a p + \frac b q.
$$
\textbf{Доказательство.} Рассмотрим функцию $\Psi(x) = x^\alpha - \alpha x, \; x \geqslant 0, \; \alpha \in (0, 1)$. Тогда $\Psi'(x) = \alpha(x^{\alpha - 1} - 1)$. Получаем, что $\Psi'(x) > 0$ при $x \in (0; 1)$ и  $\Psi'(x) < 0$ при $x \in (1; +\infty)$. Следовательно,
$$
\max\limits_{x \geqslant 0} \Psi(x) = \Psi(1) \quad \Rightarrow \quad \Psi(x) \leqslant \Psi(1), \mbox{ или } x^\alpha \leqslant \alpha x + (1 - \alpha).
$$
Рассмотрим $x = \dfrac a b, \; a \geqslant 0, \ b > 0$. Тогда $a^\alpha \cdot b^{1-\alpha } \leqslant \alpha a + (1 - \alpha ) b$.
Положим $\alpha = \dfrac 1 p$, тогда $1 - \alpha = \dfrac 1 q$. Подставляя эти значения в неравенство, получим
$$
a^{\dfrac 1 p} \cdot b^{\dfrac 1 q} \leqslant \frac a p + \frac b q. \quad \Box
$$

\textbf{Неравенство Гельдера.} Пусть $p > 1$, $\frac 1 p + \frac 1 q = 1$, $f(x) \in L_p(E)$, $g(x) \in L_q(E)$. Тогда $f(x) \cdot g(x)$ --- интегрируемая функция, и
$$
\int\limits_E |f(x) g(x)| dx \leqslant \left(\int\limits_E |f(x)|^p dx\right)^{\dfrac 1 p} \cdot \left(\int\limits_E |g(x)|^q dx\right)^{\dfrac 1 q}
$$
\textbf{Доказательство.} Введём в рассмотрение функции
$$
\varphi (x) = \frac {f(x)}{\|f\|_p}, \quad \gamma(x) = \frac {g(x)}{\|g\|_q}.
$$
Подставим в неравенство Юнга числа $a = |\varphi|^p$, $b = |\gamma|^q$:
$$
\frac{|f(x)\cdot g(x)|}{\|f\|_p\|g\|_q} \leqslant \frac {|f(x)|^p}{p\|f\|_p^p} + \frac{|g(x)|^q}{q\|g\|_q^q} \ ;
$$
$$
|f(x)\cdot g(x)| \leqslant \frac {|f(x)|^p}{p\|f\|_p^{p-1}} \|g\|_q + \frac{|g(x)|^q}{q\|g\|_q^{q-1}} \|f\|_p \ .
$$
Так как в правой части неравенства стоит интегрируемая функция, функция $f(x) \cdot g(x)$ также является интегрируемой. Поэтому
\begin{multline*}
\int\limits_E |f(x) g(x)| dx \leqslant \int\limits_E \left( \frac {|f(x)|^p}{p\|f\|_p^{p-1}} \|g\|_q + \frac{|g(x)|^q}{q\|g\|_q^{q-1}} \|f\|_p \right)dx = \\
= \frac {\|f\|_p \|g\|_q}{p} + \frac {\|f\|_p \|g\|_q}{q} = \|f\|_p \|g\|_q \left(\frac 1 p + \frac 1 q \right) = \|f\|_p \|g\|_q. \quad \Box
\end{multline*}

\textbf{Неравенство Минковского.} Пусть $f(x), g(x) \in L_p(E)$, $p \geqslant 1$. Тогда $f(x) + g(x) \in L_p(E)$ и
$$
\|f+g\|_p \leqslant \|f\|_p + \|g\|_p.
$$
\textbf{Доказательство.} Заметим, что
$$
|f+g|^p \leqslant 2^p (|f|^p + |g|^p) \quad \Rightarrow \quad f+g \in L_p(E) \quad \Rightarrow \quad |f+g|^{\dfrac p q} \in L_q(E).
$$
Поэтому в силу неравенства Гельдера
$$
\begin{matrix}
\int\limits_E |f||f+g|^{\frac p q} dx \leqslant \|f\|_p \|f+g\|^{\frac p q}_p \\
\int\limits_E |g||f+g|^{\frac p q} dx \leqslant \|g\|_p \|f+g\|_p^{\frac p q}
\end{matrix}
$$
Сложим эти два неравенства:
\begin{multline*}
\int\limits_E |f+g|^p dx = \int\limits_E |f+g| |f+g|^{p-1} dx \leqslant \\
\leqslant \int\limits_E(|f| + |g|) |f + g|^{\frac p q} dx \leqslant \\
\leqslant (\|f\|_p + \|g\|_p)\|f+g\|^{\frac p q}_p
\end{multline*}
Поделив это неравенство на $\|f+g\|^{\frac p q}_p$, получим требуемое неравенство. $\quad \Box$\\

Таким образом, аксиома 3 также выполняется, поэтому $L_p(E)$ является линейным нормированным пространством.\\

\textbf{Определение 2.} Последовательность $\lbrace f_n \rbrace$ в нормированном пространстве называется \textit{фундаментальной}, если
$$
\lim_{\substack{n\to\infty \\ m\to\infty }} \|f_n - f_m\| = 0.
$$

Линейное нормированное пространство $E$ называется \textit{полным (банаховым)}, если для любой фундаментальной последовательности $\lbrace f_n \rbrace$ пространства $E$ найдется $f \in E$ такое, что
$$
\lim_{n\to\infty} \|f_n - f\| = 0.
$$

\textbf{Теорема 1.} Пусть $E$ --- измеримое множество конечной меры, тогда пространство $L_p(E)$, $p \geqslant 1$ --- банахово.\\
\textbf{Доказательство.}
Рассмотрим фундаментальную последовательность $\lbrace f_n(x) \rbrace$. Тогда для любого $\varepsilon > 0$ найдётся такой номер $N$, что при $n, m \geqslant N$ выполняется неравенство
$$
\|f_n(x) - f_m(x)\|_p < \varepsilon.
$$
Следовательно, для любого $k \in \mathbb{N}$ найдётся такой номер $n_k$, что при $n, m \geqslant n_k$ выполняется неравенство
$$
\|f_n(x) - f_m(x)\|_p < \frac 1 {2^k}.
$$
Это, в свою очередь, означает, что существует последовательность номеров $n_1 < n_2 < \dots < n_k < n_{k+1} < \dots$ такая, что
$$
\|f_{n_{k+1}}(x) - f_{n_k}(x)\|_p < \frac 1 {2^k}.
$$
Запишем неравенство Гельдера для функций $|f_{n_{k+1}}(x) - f_{n_k}(x)|$ и $1$:
$$
\int\limits_E |f_{n_{k+1}}(x) - f_{n_k}(x)|dx \leqslant \|f_{n_{k+1}} (x) - f_{n_k} (x)\|_p |E|^{\frac 1 q} < \frac {|E|^{\frac 1 q}}{2^k}.
$$
Суммируя по $k$ от 1 до $\infty$ и учитывая тот факт, что $\sum\limits_{k=1}^\infty \dfrac 1 {2^k} = 1$, получаем, что
$$
\sum_{k=1}^\infty \int\limits_E |f_{n_{k+1}} (x) - f_{n_k}(x)| dx < |E|^{\frac 1 q}.
$$
В силу конечности $|E|$ это означает, что ряд
$$
\sum_{k=1}^\infty \int\limits_E |f_{n_{k+1}} (x) - f_{n_k}(x)| dx
$$
сходится. Следовательно, в силу следствия из теоремы Леви почти всюду на множестве $E$ сходится ряд 
$$
\sum_{k=1}^\infty \left| f_{n_{k+1}} (x) - f_{n_k} (x) \right|,
$$
а значит, сходится и ряд
$$
\sum_{k=1}^\infty \left(f_{n_{k+1}}(x) - f_{n_k}(x)\right).
$$
Добавим к этому ряду функцию $f_{n_1}(x)$. Получим, что последовательность $\lbrace f_{n_k}(x) \rbrace$ почти всюду на $E$ сходится к некоторой $f(x)$.\\

Рассмотрим последовательность $\lbrace f_{n_k}(x) - f_m(x) \rbrace$. В силу фундаментальности последовательности $\lbrace f_n(x) \rbrace$ для любого $\varepsilon > 0$ найдётся такой номер $N$, что при $n_k, m > N$ справедливо неравенство
$$
\|f_{n_k}(x) - f_m(x)\|_p \leqslant \varepsilon.
$$
С другой стороны, последовательность $\lbrace f_{n_k}(x) - f_m(x) \rbrace$ почти всюду на $E$ сходится к предельной функции $f(x) - f_m(x)$. Применяя теорему Фату (теорема 11 параграфа 4), получаем, что для любого $\varepsilon > 0$ найдётся такой номер $N$, что при $m > N$ справедливо неравенство
$$
\|f(x) - f_m(x)\|_p \leqslant \varepsilon,
$$
поэтому
$$
\lim_{m\to\infty} \|f(x) - f_m(x)\|_p = 0.
$$
Из этого следует, что пространство $L_p(E)$, $p \geqslant 1$, является банаховым. $\quad \Box$\\






%%%%%%%%%%%%%%%%%%%%%%%
%      Lecture 7      %
%%%%%%%%%%%%%%%%%%%%%%%






\textbf{Определение.} Функция, принимающая конечное или счетное число значений, называется \textit{простой}. Все различные значения простой функции можно обозначить как $c_k$, где $k = 1, 2, \dots \ .$\\

\textbf{Определение.} \textit{Характеристической функцией множества $E$} называют функцию
$$
\chi_E(x) = 
\begin{cases}
1, &x\in E; \\
0, &x\notin E.
\end{cases}
$$
Очевидно, что $\chi_E(x)$ измерима тогда и только тогда, когда измеримо само множество $E$.\\

Любую простую функцию можно представить в виде $\varphi (x) = \sum\limits_{k=1}^\infty c_k \chi_{E_k}(x)$, где для каждого $x$ из области определения только одно слагаемое отлично от нуля.\\

\textbf{Лемма 1.} Пусть $E$ --- измеримое множество, функция $f(x)$ неотрицательна и измерима на $E$. Тогда существует неубывающая последовательность простых функций, всюду сходящаяся к функции $f(x)$, причем на множестве конечных значений сходимость равномерна.\\
\textbf{Доказательство.} Введем в рассмотрение множества
$$
E_k^{(n)} = E \left[ \frac k {2^n} \leqslant f(x) < \frac {k+1}{2^n} \right], k = 0, 1, \dots; n = 1, 2, \dots;
$$
$$
E_\infty = E [ f(x) = + \infty].
$$
Тогда для любого номера $n \in \mathbb{N}$
$$
E = \left[ \bigcup_{k=1}^\infty E_k^{(n)} \right] \cup E_\infty.
$$
Рассмотрим последовательность $\lbrace f_n(x) \rbrace$, где
$$
f_n(x) = \frac k {2^n} \quad \mbox{на } E_k^{(n)}.
$$
Получаем, что для любого $n \in \mathbb{N}$
$$
0 \leqslant f(x) - f_n(x) \leqslant \frac 1 {2^n},
$$
то есть последовательность $\lbrace f_n(x) \rbrace$ сходится к функции $f(x)$ (причём на множестве конечных значений эта сходимость будет равномерной, поскольку верхняя оценка разности $f_n(x)$ и $f(x)$ не зависит от $x$).\\

Докажем, что последовательность $\lbrace f_n(x) \rbrace$ является неубывающей. Для этого каждое из множеств $E_k^{(n)}$ представим в виде
$$
E_k^{(n)} = E_{2k}^{(n+1)} \cup E_{2k+1}^{(n+1)}
$$
$$
\biggr(\mbox{другими словами, } \left[\frac k {2^n} ; \frac {k+1} {2^n}\right) = \left[\frac {2k} {2^{n+1}} ; \frac {2k+1} {2^{n+1}}\right) \cup \left[\frac {2k+1}{2^{n+1}} ; \frac {2k + 2} {2^{n+1}} \right) \biggl).
$$
Получаем, что на $E_{2k}^{(n+1)}$
$$
f_{n+1}(x) = \frac {2k}{2^{n+1}} = f_n(x),
$$
а на $E_{2k+1}^{(n+1)}$
$$
f_{n+1}(x) = \frac {2k+1}{2^{n+1}} = f_n(x) + \frac 1 {2^{n+1}}.
$$
Следовательно, последовательность $\lbrace f_n(x) \rbrace$ --- неубывающая.  $\quad \Box$\\

\textbf{Замечание.} В терминах доказанной теоремы рассмотрим последовательность $\lbrace \tilde f_n(x) \rbrace$, где
$$
\tilde f_n(x) =
\begin{cases}
n, & f_n(x) > n; \\
f_n(x), & f_n(x) \leqslant n.
\end{cases}
$$
Все $\tilde f_n(x)$ являются простыми функциями, последовательность $\lbrace \tilde f_n(x) \rbrace$ сходится к $f(x)$ при $n \to \infty$. При этом функции $\tilde f_n(x)$ принимают лишь конечное число значений.

\textbf{Теорема 2.} Пусть E --- (алт. множество конечной меры) ограниченное измеримое множество, $p \geqslant 1$. Тогда пространство непрерывных функций $C(E)$ всюду плотно в пространстве $L_p(E)$, то есть для любого $\varepsilon > 0$ и любой функции $f(x) \in L_p(E)$ найдётся функция $\varphi (x) \in C(E)$ такая, что 
$$
\|f(x) - \varphi(x)\|_{L_p(E)} < \varepsilon.
$$
\textbf{Доказательство.} Не ограничивая общности, считаем, что функция $f(x)$ всюду конечна (так как $f(x) \in L_p(E)$, множество, на котором она принимает бесконечные значения, имеет меру 0) и неотрицательна (обобщить теорему можно, используя неотрицательные функции $f^+(x)$ и $f^-(x)$).\\

Тогда в силу леммы 1 существует последовательность простых функций $\lbrace f_n(x) \rbrace$, всюду равномерно сходящаяся к функции $f(x)$ и неубывающая, причём все $f_n(x) \leqslant f(x)$. Следовательно, по теореме Леви для любого $\varepsilon > 0$ найдётся номер $n_0$ такой, что для всех $n \geqslant n_0$ справедливо неравенство $\|f_n(x) - f(x)\|_p < \varepsilon$. При этом функции $f_n(x)$ принимают лишь конечное число значений.\\

Другими словами, функцию $f(x)$ можно с любой точностью приблизить простой функцией $f_{n_0}(x)$, принимающей лишь конечное число значений, то есть имеющей вид
$$
f_m(x) = \sum_{k=1}^m C_k \chi_{E_k}(x).
$$

Следовательно, для доказательства теоремы достаточно доказать, что для любого $\varepsilon > 0$ и любой функции $f(x) \in L_p(E)$ найдётся функция $\varphi (x) \in C(E)$ такая, что 
$$
\|f_{n_0}(x) - \varphi(x)\|_p < \varepsilon
$$
$$(\mbox{тогда в силу } \|f_{n_0}(x) - f(x)\|_p < \varepsilon \mbox{ справедливо и } \|f(x) - \varphi(x)\|_p < 2 \varepsilon ).
$$

В силу следствия из теоремы 4 параграфа 2 и измеримости множества $E_k$ для любого $\varepsilon_k > 0$ найдутся замкнутое множество $F_k \subset E_k$ и открытое множество $G_k \supset E_k$ такое, что $|G_k \setminus F_k| < \varepsilon_k$.%^p$.

Введем $\psi_k(x) = \frac{\rho(x, \complement G_k)}{\rho(x, F_k) + \rho(x, \complement G_k)}$.
$$
\|\chi_{E_k}(x) - \psi_k(x)\|_p^p = \int\limits_{G_k \setminus F_k}|\chi_{E_k}(x) - \psi_k(x)|^pdx \leqslant |G_k \setminus F_k| < \varepsilon_k
$$

Таким образом, мы с наперед заданной точностью приближаем $\chi_{E_k}(x)$, что позволяет нам с любой наперед заданной точностью приблизить $f_{n_0}(x) = \sum\limits_{k = 1}^m C_k\chi_{E_k}(x)$.

$$
\|f(x)\|_{L_p(e)} = \left(\int\limits_{E}|f(x)|^p dx \right)^\frac1p \leqslant \left(\int\limits_{E}M^p dx \right)^\frac1p = M|E|^\frac1p = |E|^\frac1p\|f(x)\|_{C(E)}
$$

 %Тогда
%$$
%\|\chi_{E_k}(x) - \chi_{F_k}(x) \|_p = \left( \int\limits_{E_k \setminus F_k} 1 \ dx \right)^{\frac 1 p } = |E_k \setminus F_k|^{\frac %1 p} < \varepsilon_k.
%$$

%Введём функцию расстояния между точкой $x$ и множеством $F_k$:
%$$
%r_k(x) = \rho(x, F_k) = \inf_{y\in F_k} \rho(x, y).
%$$

%Теперь рассмотрим последовательность функций $\lbrace \varphi_k^{(n)} \rbrace$ следующего вида:
%$$
%\varphi_k^{(n)}(x) = \frac 1 {1 + n r_k(x)} = 
%\begin{cases}
%1, & x \in F_k; \\
%\frac 1 {1 + n r_k(x)}, & x \notin F_k.
%\end{cases}
%$$
%Очевидно, $\varphi_k^{(n)}(x) \to \chi_{F_k}(x)$ при $n \to \infty$. В силу теоремы Леви для любого $\varepsilon_k > 0$ %найдётся номер $N$ такой, что при всех $n \geqslant N$
%$$
%\|\varphi_k^{(n)} (x) - \chi_{F_k}(x)\|_p < \varepsilon_k.
%$$
%Кроме того, все функции $\varphi_k^{(n)}(x)$ являются непрерывными (так как непрерывны функции расстояния %$r_k(x)$), поэтому непрерывной будут функции $\chi_{F_k}(x)$, а также функция
%$$
%\varphi(x) = \sum_{k=1}^n c_k \varphi_k^{(n)}(x).
%$$
%Тогда 
%\begin{multline*}
%\|f_{n_0}(x) - \varphi(x)\|_p = \left\| \sum_{k=1}^m (c_k \varphi_k^{(n)} - c_k \chi_{E_k}(x)) \right\|_p = \\
%= \left\| \sum_{k=1}^m c_k (\varphi_k^{(n)} - \chi_{F_k} + \chi_{F_k} - \chi_{E_k}) \right\|_p \leqslant
%\sum_{k=1}^m |c_k| \biggl[ \| \varphi_k^{(n)} - \chi_{F_k} \|_p + \|\chi_{F_k} - \chi_{E_k}\|_p \biggr] \leqslant \\
%\leqslant \sum_{k=1}^m 2|c_k|\varepsilon_k < \varepsilon \quad \mbox{при } \varepsilon_k < \frac \varepsilon {2^{k+1} |c_k|}.
%\end{multline*}
Теорема доказана. $\quad \Box$\\

\textbf{Замечание.} Рассмотренные в теореме функции $\varphi_k^{(n)}$ являются непрерывными на всем пространстве $\mathbb R^m$. (алт. Можно заменить $C(E)$ на $C(R^m)$)\\

\textbf{Теорема 3 (о непрерывности в метрике $L_p$).} Пусть $E$ --- ограниченное измеримое множество. Тогда для любой функции $f(x) \in L_p(E)$ и любого $\varepsilon > 0$ найдется число $\delta > 0$ такое, что при $|h| < \delta$ справедливо неравенство
$\|f(x+h) - f(x)\|_{L_p(E)} < \varepsilon$, где функция $f(x)$ продолжена вне $E$ тождественным нулем.\\
\textbf{Доказательство.} В силу того, что множество $E$ ограничено, существует шар $S(0, r)$ (радиуса $r$ и с центром в начале координат), содержащий $E$ ($E \subset S(0, r)$).\\

Рассмотрим множество $E_1 = \overline{S(0, r+1)}$. Если $x \in E$, а $|h| < 1$, то $x + h \in E_1$. В силу теоремы 2 для любой функции $f(x) \in L_p(E)$ и любого $\varepsilon > 0$ существует функция $\varphi (x) \in C(E)$ такая, что $\|f(x) - \varphi(x)\|_{L_p(E_1)} < \varepsilon$. Тогда
\begin{multline*}
\|f(x+h) - f(x)\|_{L_p(E)} \leqslant \\
\leqslant \|f(x+h)-\varphi(x+h)\|_{L_p(E)} + \|\varphi (x+h)-\varphi(x)\|_{L_p(E)} + \|f(x)-\varphi(x)\|_{L_p(E)} \leqslant \\
\leqslant 2\|f(x) - \varphi (x) \|_{L_p(E_1)} + |E|^\frac1p\|\varphi(x+h) - \varphi(x)\|_{C(E_1)} < 2 \varepsilon + \varepsilon = 3 \varepsilon.
\end{multline*}
(тут важна теорема Кантора: непрерывность является равномерной непрерывностью на замкнутом ограниченном множестве) \\
Теорема доказана. $\quad \Box$\\


\section*{ \S 6. Метрические и нормированные пространства}

\textbf{Определение 1.} Множество $M$ называется \textit{метрическим пространством}, если каждой паре $(x, y)$ элементов этого множества поставлено в соответствие неотрицательное число $\rho(x, y)$ (называемое \textit{метрикой} или \textit{расстоянием} между элементами $x$ и $y$), удовлетворяющее следующим условиям (\textit{аксиомам}):
\begin{enumerate}
\item 
$\rho(x, y) = \rho(y, x)$ (аксиома симметрии)
\item 
$\rho(x, y) = 0 \Leftrightarrow x = y$ (аксиома тождества)
\item
$\rho(x, z) \leqslant \rho(x, y) + \rho(y, z)$ (так называемая \textit{аксиома треугольника})
\end{enumerate}

Дискретная метрика: $\rho(x, y) = \begin{cases}
1, & x \neq y, \\
0, & x = y.
\end{cases}
$

\textbf{Определение 2.} Элемент $x$ метрического пространства $M$ называется \textit{пределом последовательности} $\lbrace x_n \rbrace$ (обозначается $x = \lim\limits_{n\to\infty} x_n$), если $\rho(x,x_n) \to 0$ при $n \to \infty$. Сама последовательность $\lbrace x_n \rbrace$ называется тогда \textit{сходящейся}. \\

\textbf{Определение 3.} Последовательность $\lbrace x_n \rbrace$ элементов метрического пространства $M$ называется \textit{фундаментальной}, если для любого $\varepsilon > 0$ найдётся номер $n_0(\varepsilon)$ такой, что при $n, m \geqslant n_0$  $\rho (x_m, x_n) < \varepsilon$ (другими словами, $\rho(x_m, x_n) \to 0 \mbox{ при } n \to\infty \mbox{ и } m\to\infty$).\\

Последовательности в метрическом пространстве обладают следующими свойствами: 
\begin{enumerate}
\item
$x_n \to x, x_n \to y \quad \Rightarrow \quad x = y$.\\
Для доказательства достаточно заметить, что в силу аксиомы 3 $\rho(x, y) \leqslant \rho(x, x_n) + \rho(y, x_n)$. $x_n \to x, x_n \to y$, поэтому при $n \to \infty$ $\rho(x, x_n) \to 0$ и $\rho(y, x_n) \to 0$. Следовательно, $\rho(x, y) = 0$, то есть $x = y$.

\item
$x_n \to x \quad \Rightarrow \quad x_{n_k} \to x$.\\

$\forall \varepsilon>0 ~\exists N: ~\forall n>N ~\rho(x_n, x) < \varepsilon \Rightarrow \forall n_k>N ~\rho(x_n, x) < \varepsilon$.

\item $x_n \to x \quad \Rightarrow \quad \rho(x_n, \theta) \leqslant K \; \forall \theta$.\\
Опять же, в силу аксиомы треугольника для любого $n$ верно неравенство $\rho(x_n, \theta) \leqslant \rho (x_n, x) + \rho (x, \theta) \leqslant L + \rho(x, \theta) = K$.
\end{enumerate}

Введём следующие обозначения:

\begin{itemize}
\item \textit{Шаром с центром в точке $a$ и радиусом $r$} называется множество точек $S(a, r) =\lbrace \rho(a, x) < r \rbrace$;

\item \textit{Замкнутым шаром с центром в точке $a$ и радиусом $r$} называется множество точек $S(a, r) =\lbrace \rho(a, x) \leqslant r \rbrace$;

\item Множество называется \textit{ограниченным}, если оно содержится в каком-либо шаре;

\item \textit{Окрестностью точки $a$} называется любой шар $S(a, r)$;

\item В метрическом пространстве $M$ точка $a$ называется \textit{предельной точкой} множества $X$ ($X \subset M$), если в любой окрестности точки $a$ содержится хотя бы одна точка множества $X$, отличная от $a$, то есть если $\forall r \; S(a, r) \cap \lbrace X \setminus \{a\} \rbrace \neq \varnothing$;

\item \textit{Замыканием} множества $X$ называется множество, полученное присоединением к $X$ всех его предельных точек;

\item Множество $X$ называется \textit{замкнутым}, если оно совпадает со своим замыканием ($X = \overline{X}$);

\item Множество $X$ называется \textit{открытым}, если замкнуто его дополнение $\complement X = M \setminus X$;

\item Множество $X$ называется \textit{всюду плотным в метрическом пространстве $M$}, если $\overline{X} = M$;

%Любое подмножество метрического пространства --- метрическое пространство.

\item Множество $X$ называется \textit{нигде не плотным в метрическом пространстве $M$}, если любой шар пространства $M$ содержит в себе шар без точек множества $X$. \\
\textbf{Пример.} Любое конечное множество чисел.

\end{itemize}

\textbf{Пример.} Рассмотрим на множестве действительных чисел следующую метрику:
$$
\rho (x, y) = \begin{cases}
1, & x \neq y, \\
0, & x = y.
\end{cases}
$$
В таком метрическом пространстве ни одно множество не будет иметь предельных точек, поэтому любое множество будет одновременно и замкнутым, и открытым.\\

\textbf{Определение 3.} Метрическое пространство $M$ называется полным, если любая фундаментальная последовательность в $M$ сходится к некоторому пределу, являющемуся элементом $M$.\\

Рассмотрим множество всех числовых последовательностей вещественных чисел $x = (\xi_1,\ \xi_2,\ \dots )$ таких, что $\sum\limits_{i=1}^\infty | \xi_i |^p < + \infty$ ($p \geqslant 1$). Для его элементов $x = (\xi_1,\ \xi_2,\ \dots )$ и $y = (\eta_1,\ \eta_2,\ \dots )$ определим расстояние по формуле
$$
\rho_p(x, y) = \left(\sum_{i=1}^\infty |\xi_i - \eta_i|^p \right)^{\frac 1 p}.
$$
Эта метрика удовлетворяет трём аксиомам метрического пространства. Полученное пространство называется \textit{пространством $l_p$}.\\

\textbf{Утверждение.} $l_p$ --- полное пространство.\\
\textbf{Доказательство.} Рассмотрим в $l_p$ произвольную фундаментальную последовательность $\lbrace x_n \rbrace$, где $x_n = (\xi_1^{(n)},\ \xi_2^{(n)},\ \dots )$:
$$
\forall \varepsilon > 0 \; \exists N = N(\varepsilon): \; \forall n, m \geqslant N \quad
\rho_p(x_n, x_m) = \left(\sum_{i=1}^\infty |\xi_i^{(n)} - \xi_i^{(m)}|^p \right)^{\frac 1 p} < \varepsilon
$$
Получаем, что
$$
\sum_{i=1}^\infty |\xi_i^{(n)} - \xi_i^{(m)}|^p < \varepsilon^p \quad \Rightarrow \quad
|\xi_i^{(n)} - \xi_i^{(m)}|^p  < \varepsilon^p \quad \Rightarrow \quad |\xi_i^{(n)} - \xi_i^{(m)}|  < \varepsilon.
$$
А это означает, что последовательность $\lbrace \xi_i^{(n)} \rbrace$ сходится к некоторому $\xi_i$. Поэтому последовательность $\lbrace x_n \rbrace$ сходится к некоторому $x = (\xi_1,\ \xi_2,\ \dots )$. Докажем, что $x \in l_p$.\\

Так как
$$
\sum_{i=1}^\infty |\xi_i^{(n)} - \xi_i^{(m)}|^p < \varepsilon^p,
$$
то для любого числа $k$ будет верно неравенство
$$
\sum_{i=1}^k |\xi_i^{(n)} - \xi_i^{(m)}|^p < \varepsilon^p.
$$
Устремляя к бесконечности сначала $m$, а затем $k$, а также возводя неравенство в степень $\frac 1 p$, получаем неравенство
$$
\sum_{i=1}^\infty \left( | \xi_i - \xi_i^{(n)} |^p \right)^{\frac 1 p} \leqslant \varepsilon.
$$
Таким образом, при любом $n$ $(x - x_n) \in l_p$. Кроме того, сами $x_n$ также являются элементами $l_p$\\

Рассмотрим два числа, $a$ и $b$. Очевидно, что $|a + b| \leqslant |a| + |b|$. Если $|a| > |b|$, то
$$
|a + b| \leqslant 2|a| \quad \Rightarrow \quad |a + b|^k \leqslant 2^k |a|^k \leqslant 2^k (|a|^k + |b|^k).
$$
Если же $|a| \leqslant |b|$, то
$$
|a + b| \leqslant 2|b| \quad \Rightarrow \quad |a + b|^k \leqslant 2^k |b|^k \leqslant 2^k (|a|^k + |b|^k).
$$
Таким образом, неравенство $|a + b|^k \leqslant 2^k (|a|^p + |b|^k)$ выполняется всегда. Положим $a = \xi_i - \xi_i^{(n)}$, $b = \xi_i^{(n)}$, $k = p$. Получаем, что
\begin{multline*}
\left( \sum_{i=1}^\infty | \xi_i |^p \right)^{\frac 1 p} = \left( \sum_{i=1}^\infty | \xi_i - \xi_i^{(n)} + \xi_i^{(n)} |^p \right)^{\frac 1 p} \leqslant \\
\leqslant \left( \sum_{i=1}^\infty 2^p \bigl( | \xi_i - \xi_i^{(n)} |^p + | \xi_i^{(n)} |^p \bigr) \right)^{\frac 1 p}
= 2 \left( \sum_{i=1}^\infty | \xi_i - \xi_i^{(n)} |^p + \sum_{i=1}^\infty | \xi_i^{(n)} |^p \right)^{\frac 1 p}.
\end{multline*}
Теперь применим то же самое неравенство для $a = \sum\limits_{i=1}^\infty | \xi_i - \xi_i^{(n)} |^p$, $b = \sum\limits_{i=1}^\infty | \xi_i^{(n)} |^p$, $k = \frac 1 p$:
\begin{multline*}
\left( \sum_{i=1}^\infty | \xi_i |^p \right)^{\frac 1 p} \leqslant
2 \left( \sum_{i=1}^\infty | \xi_i - \xi_i^{(n)} |^p + \sum_{i=1}^\infty | \xi_i^{(n)} |^p \right)^{\frac 1 p} \leqslant \\
\leqslant 2^{1 + \frac 1 p} \left( \left( \sum_{i=1}^\infty | \xi_i - \xi_i^{(n)} |^p \right)^{\frac 1 p} +
\left( \sum_{i=1}^\infty | \xi_i^{(n)} |^p \right)^{\frac 1 p} \right).
\end{multline*}
В силу того, что $(x - x_n)$ и $x_n$ являются элементами $l_p$, справедливы неравенства
$$
\left( \sum_{i=1}^\infty | \xi_i - \xi_i^{(n)} |^p \right)^{\frac 1 p} < + \infty, \qquad
\left( \sum_{i=1}^\infty | \xi_i^{(n)} |^p \right)^{\frac 1 p} < + \infty.
$$
Но тогда справедливо и неравенство
$$
\left( \sum_{i=1}^\infty | \xi_i |^p \right)^{\frac 1 p} < + \infty,
$$
а оно означает, что $x \in l_p$. $\quad \Box$\\

\textbf{Теорема 1 (о вложенных шарах).}
Пусть в полном метрическом пространстве имеется последовательность вложенных друг в друга замкнутых шаров, радиусы которых стремятся к нулю:
$$
\overline{S_1(a_1, r_1)} \supset \overline{S_2(a_2, r_2)} \supset \dots, \quad r_n \to 0 \mbox{ при } n\to\infty.
$$
Тогда существует одна и только одна точка, принадлежащая всем этим шарам.\\
\textbf{Доказательство.} Рассмотрим последовательность $\lbrace a_n \rbrace$ центров этих шаров. Так как $\overline{S}{}_{n+p} \subset \overline{S}{}_n$, то $a_{n+p} \in \overline{S}{}_n$. Следовательно, $\rho (a_{n+p}, a_n) < r_n$ и поэтому стремится к нулю при $n\to\infty$ равномерно по всем $p$. Это означает, что последовательность $\lbrace a_n \rbrace$ является фундаментальной. Так как по условию задачи рассматриваемое метрическое пространство является полным, то $\lbrace a_n \rbrace$ сходится к некоторому элементу $a$ этого же пространства.\\

Возьмём любой шар $\overline{S}{}_k$. Тогда точки $a_k,\ a_{k+1},\ \dots$ принадлежат $\overline{S}{}_k$. Так как шар по условию замкнут, предел $a$ последовательности $\lbrace a_n \rbrace_{n=k}^{\infty}$ также принадлежит $\overline{S}{}_k$. Следовательно, предел $a$ последовательности $\lbrace a_n \rbrace_{n=1}^{\infty}$ принадлежит всем шарам.\\

Допустим, что существует точка $b$, принадлежащая всем шарам и отличная от $a$ (то есть $\rho(a, b) = \delta > 0$). Так как $a, b \in \overline{S}{}_n \; \forall n$, то справедливо неравенство
$$
\delta = \rho(a, b) \leqslant \rho(a, a_n) + \rho(a_n, b) \leqslant 2 r_n \rightarrow 0 \mbox{ при } n\to\infty.
$$
Мы получили противоречие; следовательно, всем шарам принадлежит только одна точка. $\quad \Box$\\ 

\textbf{Определение 4.} Множество $X$ метрического пространства $M$ называется \textit{множеством 1-й категории}, если его можно представить в виде не более чем счётного объединения нигде не плотных множеств. Множество, не являющееся множеством 1-й категории, называется \textit{множеством 2-й категории}.\\

Множество рациональных точек на $\mathbb{R}$ является множеством 1-й категории, множество иррациональных точек --- множеством 2-й категории.






%%%%%%%%%%%%%%%%%%%%%%%
%      Lecture 8      %
%%%%%%%%%%%%%%%%%%%%%%%






\textbf{Теорема 2 (теорема Бэра о категориях).} Полное метрическое пространство является множеством 2-й категории.\\
\textbf{Доказательство.} Пусть это не так. Тогда рассматриваемое пространство $M$ представимо в виде $M = \bigcup\limits_{n=1}^\infty X_n$, где множества $X_i$, $i = 1,\ 2,\ \dots$ нигде не плотны.\\

Рассмотрим шар $\overline{S(a, 1)}$, где $a$ --- произвольная точка пространства. Так как множество $X_1$ нигде не плотно, то внутри шара $\overline{S(a, 1)}$ найдётся подшар $\overline{S_1(a_1, r_1)}$ радиуса $r_1 < 1$, не содержащий точек $X_1$. Так как множество $X_2$ нигде не плотно, то внутри шара $\overline{S_1(a_1, r_1)}$ найдётся подшар $\overline{S_2(a_2, r_2)}$ радиуса $r_2 < \frac 1 2$, не содержащий точек $X_2$. Проводя аналогичные рассуждения дальше ($r_n < \frac1n$), получим последовательность замкнутых шаров, вложенных друг в друга:
$$
\overline{S_1(a_1, r_1)} \supset \overline{S_2(a_2, r_2)} \supset \dots,
$$
причём шар $\overline{S_k(a_k, r_k)}$ не содержит точек ни одного из множеств $X_1$, $X_2$, $\dots$, $X_k$. По теореме 1 существует точка $a \in M$, принадлежащая всем этим шарам. Но тогда эта точка не принадлежит ни одному из множеств $X_n$, объединением которых является $M \ni a$. Мы получили противоречие $\Rightarrow$ наше предположение неверно. Теорема доказана. $\quad \Box$\\

Рассмотрим оператор $A: M \to M$. Оператор $A$ называется \textit{сжимающим отображением (сжимающим оператором) на $M$}, если существует число $\alpha < 1$ такое, что для всех $x, y \in M$ справедливо неравенство 
$$
\rho (Ax, Ay) \leqslant \alpha \rho (x, y).
$$
\textit{Неподвижной точкой} оператора $A$ называется точка, удовлетворяющая условию $Ax = x$.\\

\textbf{Теорема 3 (Принцип сжимающих отображений).}
Пусть $M$ --- полное метрическое пространство,  $A$ --- сжимающее отображение на $M$. Тогда $A$ имеет единственную неподвижную точку в $M$.\\
\textbf{Доказательство.}
Фиксируем произвольный элемент $x_0 \in M$ и построим для него итерационную последовательность $\lbrace x_n \rbrace$ следующим образом:
$$
\forall n \in \mathbb{N} \quad x_n = Ax_{n-1}.
$$
Заметим, что
\begin{gather*}
\rho(x_2, x_1) = \rho (Ax_1, Ax) \leqslant \alpha \rho(x_1, x_0) = \alpha \rho(Ax_0, x_0);\\
\dots\\
\rho(x_{n+1}, x_n) \leqslant \alpha\rho(x_n, x_{n-1}) \leqslant \alpha^n \rho(Ax_0, x_0).
\end{gather*}

Тогда
\begin{multline*}
\rho(x_{n+p}, x_n) \leqslant \rho(x_{n+1}, x_n) + \rho (x_{n+2}, x_{n+1})+ \dots + \rho(x_{n+p}, x_{n+p-1}) \leqslant \\
\leqslant \alpha^n \rho(Ax_0, x_0) + \alpha^{n+1} \rho(Ax_0, x_0) + \dots + \alpha^{n+p-1} \rho(Ax_0,x_0) = \\
= \frac {\alpha^n (1 - \alpha^p)}{1-\alpha} \rho (Ax_0, x_0)
\leqslant \frac {\alpha^n}{1-\alpha} \rho(Ax_0,x_0).
\end{multline*}
Так как $\alpha < 1$, то $\rho(x_{n+p}, x_n) \to 0$ при $n\to\infty$ равномерно по всем $p$. Это означает, что последовательность $\lbrace x_n \rbrace$ является фундаментальной. По условию $M$ --- полное метрическое пространство; следовательно, существует точка $x \in M$, являющаяся пределом $\lbrace x_n \rbrace$ при $n\to\infty$. Докажем неподвижность $x$:
\begin{multline*}
\rho (Ax, x) \leqslant \rho(Ax, x_n) + \rho(x_n, x) = \\
= \rho(Ax, Ax_{n-1}) + \rho (x_n, x) \leqslant \\
\leqslant \alpha \rho (x, x_{n-1}) + \rho(x_n, x) \to 0 \text{ при } n\to\infty.
\end{multline*}
Устремив $n$ к бесконечности, получим, что $\rho (Ax, x) = 0$; следовательно, точка $x$ действительно является неподвижной.\\
Утверждение о том, что неподвижная точка единственна, докажем от противного. Пусть существуют две неподвижных точки: $Ax = x, Ay = y$. Тогда 
$$
\rho (x, y) = \rho(Ax, Ay) \leqslant \alpha\rho(x,y) \Rightarrow \rho(x,y) = 0,
$$
то есть $x = y$. Теорема полностью доказана. $\quad \Box$\\

\textbf{Пример.} Одним из применений принципа сжимающих отображений является доказательство существования и единственности решения интегрального уравнения Фредгольма 2-го рода. Пусть $K(s, t)$ --- действительная функция, $K(s, t) \in L_2(\Pi)$, где $\Pi$ --- квадрат $(a \leqslant t,s \leqslant b)$ (это условие, вообще говоря, можно заменить условием
$
\int\limits_a^b dt \int\limits_a^b K^2(s,t)ds < + \infty
$),
и пусть, кроме того, функция $f(s) \in L_2 (a, b)$. Докажем, что тогда интегральное уравнение 
$$
x(s) = \lambda \int\limits_a^b K(s,t)x(t) dt + f(s)
$$
имеет при достаточно малых значениях параметра $\lambda$ единственное решение $x(s) \in L_2(a, b)$.\\

Рассмотрим соответствующий оператор
\begin{multline*}
Ax = \lambda \int\limits_a^b K(s,t)x(t) dt + f(s) = \lambda A_1 x + f(s),\\
\text{где } A_1 x = \int\limits_a^b K(s,t)x(t) dt.
\end{multline*}
Докажем, что оператор $A$ переводит каждую функцию $x(s) \in L_2(a, b)$ в функцию, также принадлежащую $L_2(a, b)$. Так как $f(s) \in L_2 (a, b)$, то достаточно доказать, что оператор $A_1$ обладает тем же свойством.\\

Так как $K(s, t) \in L_2(\Pi)$, то при каждом фиксированном $s \in [a, b]$ функция $K(s, t)$, являясь функцией $K(t)$, принадлежит пространству $L_2([a, b])$. Функция $x(t)$ также принадлежит $L_2([a, b])$. Но тогда и функции $(K(s, t) + x(t)), (K(s, t) - x(t)) \in L_2([a, b]) \; \Rightarrow$ функции $(K(s, t) + x(t))^2, (K(s, t) - x(t))^2$ интегрируемы по $t$ на $[a, b]$ $\Rightarrow$ функция $K(s, t) x(t) = \frac 1 4 \left( (K + x)^2 - (K - x)^2 \right)$ также является интегрируемой по $t$ на $[a, b]$. Следовательно, для всех $s \in [a, b]$ существует интеграл
$$
y(s) = \int\limits_a^b K(s,t) x(t) dt.
$$
Применяем к $y^2(s)$ неравенство Коши-Буняковского:
$$
y^2(s) = \left( \int\limits_a^b K(s,t) x(t) dt \right)^2 \leqslant \int\limits_a^b K^2(s,t) dt \int\limits_a^b x^2(t) dt.
$$
Интеграл $\int\limits_a^b x^2(t) dt$ представляет собой некоторую постоянную. Функция $K^2(s ,t)$ интегрируема на $\Pi$, поэтому в силу теоремы Фубини функция $\int\limits_a^b K^2(s,t) dt$ интегрируема по $s$ на $[a, b]$. Значит, и функция $y^2(s)$ является интегрируемой по $s$ на $[a, b]$, причём справедливо неравенство
$$
\int\limits_a^b y^2(s) ds \leqslant \int_a^b \left( \int\limits_a^b K^2(s,t) dt \int\limits_a^b x^2(t) dt \right) ds.
$$
Оценим теперь $\rho(Ax, Az)$:
\begin{multline*}
\rho(Ax, Az) = \\
= \left( \int\limits_a^b\left[ \lambda \int\limits_a^b K(s,t) x(t)dt - \lambda \int\limits_a^b K(s,t)z(t)dt \right]^2 ds \right)^{\frac 1 2} = \\
= |\lambda| \left( \int\limits_a^b \left[ \int\limits_a^b K(s,t) [x(t) - z(t)] dt \right]^2ds \right)^{\frac 1 2} \leqslant \\
\leqslant |\lambda|\left( \int\limits_a^b \left[ \int\limits_a^b K^2(s,t)dt \right] \left[ \int\limits_a^b[x(t)-z(t)]^2dt \right] ds \right)^{\frac 1 2} = \\
=|\lambda|\left( \int\limits_a^b ds \int\limits_a^b K^2(s,t)dt \right)^{\frac 1 2} \left( \int\limits_a^b[x(t)-z(t)]^2dt \right)^{\frac 1 2} = \\
= |\lambda|\left( \int\limits_a^b ds \int\limits_a^b K^2(s,t)dt\right)^{\frac 1 2} \rho(x,z).
\end{multline*}

Таким образом, при
$$
|\lambda| < \frac 1 {\left(\int\limits_a^b ds \int\limits_a^b K_2(s,t)dt\right)^{\frac 1 2}} 
$$
мы находимся в условиях применимости принципа сжимающих отображений, то есть у оператора $A$ существует единственная неподвижная точка. А это и означает существование и единственность решения интегрального уравнения Фредгольма 2-го рода. $\quad \Box$\\

\textbf{Задача.} Рассмотрим в произвольном метрическом пространстве $M$ оператор $A$:
$$
A:M\to M, \quad \rho(Ax, Ay) < \rho(x,y).
$$
Можно ли обобщить теорему о неподвижной точке (другими словами, в любых ли метрических пространствах оператор $A$ имеет неподвижную точку)?\\
\text{[Правильный ответ --- нет]}\\
Пример: $M = (0, 1) \cup (1, \infty)$, $A(x) = \frac1x$.

Перейдём к рассмотрению нормированных пространств.\\

\textit{Линейным многообразием} $L$ в линейном пространстве $X$ называется непустое подмножество пространства $X$, обладающее тем свойством, что для любых элементов $x, y \in L$ их линейная комбинация $\alpha x + \beta y$ также принадлежит $L$.\\

\textbf{Определение 4.} Линейное многообразие в нормированном пространстве называется \textit{подпространством}, если оно замкнуто относительно сходимости по норме.\\

\textbf{Теорема 4 (теорема Рисса).} Пусть $X$ --- подпространство в нормированном пространстве $M$, не совпадающее с $M$. Тогда для любого $\varepsilon \in (0, 1)$ найдётся элемент $y \in M \setminus X, \; \|y\| = 1$ и такой, что  $\forall x \in X \; \|x - y\| > 1 - \varepsilon$.\\
\textbf{Доказательство.} Выберем произвольный элемент $y_0 \in M \setminus X$ ($M \setminus X \neq \varnothing$, так как $X$ --- подпространство $M$, не совпадающее с $M$). Рассмотрим величину
$$
d = \inf_{x\in X} \|x - y_0\|.
$$
От противного доказывается, что $d > 0$ (если бы $d$ равнялось нулю, то существовала бы последовательность $\lbrace x_n \rbrace$, $x_n \in X$, сходящаяся к $y_0 \notin X$; тем самым нарушалась бы замкнутость относительно сходимости по норме).\\

Таким образом, для любого $\varepsilon > 0$ найдётся элемент $x_0 \in X$ такой, что $0 < d \leqslant \|x_0 - y_0\| < d + d \varepsilon$. Тогда выберем элемент $y = \dfrac {x_0 - y_0} {\|x_0 - y_0\|}$. Очевидно, что $\|y\| = 1$. То, что $y \notin X$, доказывается от противного (если бы $y$ принадлежал $X$, то и элемент $\|x_0 - y_0\| y + x_0 = y_0$ принадлежал бы $X$, а этого быть не может). Проверим, что он удовлетворяет требуемому неравенству (учитывая, что $x_0 - \|x_0 - y_0\| x \in X$):
\begin{multline*}
\|x - y\| = \left\|x - \frac {x_0 - y_0}{\|x_0 - y_0\|} \right\| = \\
= \frac {\left\| y_0 - (x_0 - \|x_0 - y_0\| x)\right\|}{\|x_0 - y_0\|} > \frac d {d + d\varepsilon} = \\
= \frac 1 {1+ \varepsilon} > 1 - \varepsilon. \quad \Box
\end{multline*}


\section*{ \S 7. Линейные операторы}

\textbf{Определение 1.} Оператор $A$, действующий из линейного нормированного пространства $X$ в линейное нормированное пространство $Y$ над одним и тем же полем чисел ($\mathbb R$ или $\mathbb C$), называется \textit{линейным оператором}, если 
\begin{enumerate}
\item
$\forall x_1, x_2 \in X \; A(x_1 + x_2) = Ax_1 + Ax_2$;
\item
$\forall x \in X, \; \forall \lambda \in \mathbb R \quad (\lambda \in \mathbb C) \; A \lambda x = \lambda A x_1$.
\end{enumerate}

\textbf{Эквивалентное условие.}
$$\forall x_1, x_2 \in X, \alpha, \beta \in \mathbb{R} \quad (\alpha, \beta \in \mathbb{C}) \; A(\alpha x_1 + \beta x_2) = \alpha Ax_1 + \beta Ax_2$$

\textbf{Определение 2.} Оператор $A: X \to Y$ называется \textit{непрерывным в точке x}, если из сходимости по норме последовательности $\lbrace x_n \rbrace$ к $x$ в пространстве $X$ следует сходимость по норме последовательности $\lbrace A x_n \rbrace$ к $A x$ в пространстве $Y$.\\

\textbf{Определение 3.} Оператор $A: X \to Y$ называется \textit{непрерывным (во всем пространстве)}, если он непрерывен в каждой точке. \\

\textbf{Теорема 1.} Для того, чтобы линейный оператор был непрерывным, необходимо и достаточно, чтобы он был непрерывен хотя бы в одной точке.\\
\textbf{Доказательство.} Необходимость очевидна; докажем достаточность. Пусть в некоторой точке $x_0 \in X$ оператор $A$ непрерывен: $ x_n \to x_0 \; \Rightarrow \; Ax_n \to Ax_0$. Пусть теперь $x$ --- произвольная точка пространства $X$ и $x_n \to x$. Тогда $x_n - x + x_0 \to x_0$, поэтому в силу непрерывности оператора $A$ в точке $x_0$
$$
A(x_n - x + x_0) = Ax_n - Ax + Ax_0 \to Ax_0.
$$
Из этого следует, что $Ax_n \to Ax$. $\quad \Box$\\

(Следующий пример не проверял --- ред.)

\textbf{Пример.} Рассмотрим линейное пространство непрерывных на сегменте $[0; 1]$ функций и оператор $A f(t) = f(0)$ на нём.
\begin{enumerate}
\item
Сначала введем метрику $C[0; 1]: \; \|f(t)\| = \max\limits_{t\in[0, 1]} |f(t)|$. В этой метрике оператор $A$ будет являться непрерывным, так как он непрерывен в нуле: если $\|f_n(t) - 0\| \to 0$, то $\|A f_n(t) - 0\| = \|Af_n(t)\| \to 0$.

\item
Теперь введём метрику $L_1([0, 1]): \; \|f(t)\| = \int\limits_0^1 |f(t)| dt$. Здесь непрерывности уже не будет. Например, возьмём последовательность функций следующего вида:
$$
f_n(x)=\begin{cases}
n, &\text{если $x = 0$} \\
- \frac {n^3} 2 x + n &\text{если $x \in (0; \frac 2 {n^2})$} \\
0, &\text{если $x \in [\frac 2 {n^2}, 1])$}
\end{cases}
$$
Тогда в нуле получаем $\|f_n(t) - 0 \| = \|f_n(t)\| \to 0$, но при этом $\|A f_n(t) - 0\| = \|f_n(0) - 0\| = n \not\to 0$.
\end{enumerate}

\textbf{Определение 4.} Оператор $A: X \to Y$ называется \textit{ограниченным}, если найдётся константа $M$ такая, что для всех $x \in X$ будет справедливо неравенство $\|Ax\|_Y \leqslant M \|x\|_X$. При этом минимальная из таких констант называется \textit{нормой оператора $A$}: $\|A\| \equiv inf M = \sup\limits_{x\neq0} \frac {\|Ax\|}{\|x\|}$.\\

\textbf{Теорема 2.} Для того, чтобы линейный оператор был непрерывным, необходимо и достаточно, чтобы он был ограничен.\\
\textbf{Доказательство.} Сначала докажем необходимость. Пусть оператор $A$ непрерывен. Предположим, что он не ограничен (то есть "универсальной" константы $M$ не существует). Тогда найдётся последовательность элементов $\lbrace x_n \rbrace$ такая, что $\|Ax_n\| > n \|x_n\|$. Введём в рассмотрение элементы
$$
\xi_n = \frac {x_n} {n \|x_n\|}.
$$
Тогда
$$
 \|\xi_n\| = \frac{\|x_n\|}{n \|x_n\|} = \frac 1 n \to 0 \text{ при } n \to \infty.
$$
С другой стороны,
$$
\|A\xi_n\| = \frac {\|Ax_n\|}{n\|x_n\|} > \frac {n\|x_n\|} {n\|x_n\|} = 1 \not\to 0 = \|A0\|.
$$
Мы получили противоречие ($\{\xi_n\} \to 0$, но $\{A\xi_n\} \not \to A0$); следовательно, наше предположение неверно и оператор $A$ является ограниченным.\\

Теперь докажем достаточность. Пусть линейный оператор $A$ ограничен, то есть существует константа $M$ такая, что для всех $x$ справедливо $\|Ax\| \leqslant M \|x\|$. Пусть $x_n \to x$ при $n\to\infty$, то есть $\|x_n - x\| \to 0$ при $n\to\infty$. Но тогда
$$
\|Ax - Ax_n\| = \|A(x - x_n)\| \leqslant M \|x - x_n\| \to 0 \text{ при } n\to\infty,
$$
то есть оператор $A$ является непрерывным. $\quad \Box$\\

В дальнейшем будем рассмаривать ограниченные операторы. 

\textbf{Определение.} \textit{Нормой оператора $A$} называется наименьшая из таких ограничивающих констант $M$: $\|A\| = \inf\limits_{M: \|Ax\| \leqslant M\|x\|, \forall x \in X}M$. \\
\textbf{Эквивалентные определения} $\|A\| = \sup\limits_{\|x\| \neq 0} \dfrac {\|Ax\|} {\|x\|}$, $\|A\| = \sup\limits_{\|x\| \leqslant 1} \|Ax\|$. \\
\textbf{Доказательство:} 
 $\|A\| = \sup\limits_{\|x\| \neq 0} \dfrac {\|Ax\|} {\|x\|}$ очевидно следует из определения $\|A\| = \inf\limits_{M: \|Ax\| \leqslant M\|x\|, \forall x \in X}M = \inf\limits_{M: \frac{\|Ax\|}{\|x\|} \leqslant M, \forall x \in X}M$. \\
Покажем, что $\|A\| = \sup\limits_{\|x\| \leqslant 1} \|Ax\|$ \\
Действительно, если $\|x\| \leqslant 1$, то
$$
\|Ax\| \leqslant \|A\| \|x\| \leqslant \|A\|,
$$
поэтому и
$$
\sup_{\|x\| \leqslant 1} \|Ax\| \leqslant \|A\|.
$$
С другой стороны, для любого $\varepsilon > 0$ существует элемент $\xi_\varepsilon$ такой, что
$$
\|A\xi_\varepsilon\| > ( \|A\| - \varepsilon ) \|\xi_\varepsilon\|.
$$
(рассматриваем случай $\|A\| \neq 0, \|\xi_\varepsilon\| \neq 0$). Возьмём 
$$
x_\varepsilon = \frac {\xi_\varepsilon}{\|\xi_\varepsilon\|}.
$$
Тогда
$$
\|A x_\varepsilon\| = \frac {\|A\xi_\varepsilon\|}{\|\xi_\varepsilon\|} > \frac {( \|A\| - \varepsilon ) \|\xi_\varepsilon\|} {\|\xi_\varepsilon\|} = \|A\| - \varepsilon.
$$
Так как $\|x_\varepsilon\| = 1$, то
$$
\sup_{\|x\| \leqslant 1} \|Ax\| \geqslant \|A x_\varepsilon\| > \|A\| - \varepsilon,
$$
поэтому $\sup\limits_{\|x\| \leqslant 1} \|Ax\| \geqslant \|A\|$. Но перед этим мы получили неравенство $\sup\limits_{\|x\| \leqslant 1} \|Ax\| \leqslant \|A\|$. Эти два неравенства означают, что на самом деле $\sup\limits_{\|x\| \leqslant 1} \|Ax\| = \|A\|$. $\quad \Box$\\ \\

\textbf{Утверждение.} Пусть даны два линейных нормированных пространства, $X$ и $Y$. Тогда совокупность всех линейных ограниченных операторов, действующих из $X$ в $Y$ (будем обозначать её $L(X \to Y)$) образует линейное нормированное пространство.\\
\textbf{Доказательство.} Линейность этого пространства очевидна (в качестве нулевого элемента выбирается нулевой оператор). Покажем, что выполняются аксиомы нормированного пространства. $\|A\| = \sup\limits_{\|x\| \leqslant 1} \|A x\| \geqslant 0$; Если $\|A\| = 0$, то $\|A x\| = 0$ для всех $x$ таких, что $\|x\| \leqslant 1$. Но тогда $A x = 0$ и для всех $x$, и, следовательно, $A = 0$. Получили, что первая аксиома выполняется.
$$
\|\lambda A\| = \sup_{\|x\| \leqslant 1} \|\lambda A x\| = |\lambda | \sup_{\|x\| \leqslant 1} \|Ax\| = |\lambda | \|Ax\|,
$$
то есть вторая аксиома также выполняется.
$$
\|A + B\| =  \sup_{\|x\| \leqslant 1} \|Ax + Bx\| \leqslant \sup_{\|x\| \leqslant 1} \|Ax\| + \sup_{\|x\| \leqslant 1} \|Bx\| = \|A\| + \|B\|.
$$
Таким образом, выполняется и третья аксиома, то есть пространство является нормированным. $\quad \Box$\\

\textbf{Теорема 3.} Пусть $X$ --- линейное нормированное пространство, $Y$ --- банахово пространство. Тогда $L(X \to Y)$ --- банахово пространство.\\
\textbf{Доказательство.} Пусть дана фундаментальная последовательность линейных ограниченных операторов $\lbrace A_n \rbrace$:
$$
\|A_m - A_n\| \to 0 \text{ при } m,n \to \infty.
$$
Тогда для любого $x$
$$
\|A_m x - A_n x\| \leqslant \|A_m - A_n\| \|x\| \to 0 \text{ при } m,n \to \infty.
$$
Следовательно, при каждом фиксированном $x$ последовательность $\lbrace A_n x \rbrace$ является фундаментальной; в силу полноты пространства $Y$ это означает, что $\lbrace A_n x \rbrace$ имеет некоторый предел $y \in Y$. Таким образом, каждому элементу $x \in X$ ставится в соответствие элемент $y \in Y$, то есть задаётся некоторый оператор $A: X \to Y$.\\

Докажем его ограниченность. По условию $\|A_m - A_n\| \to 0 \text{ при } m,n \to \infty$; значит,
$$
\left| \|A_m\| - \|A_n\| \right| \leqslant \|A_m - A_n\| \to 0 \text{ при } m,n \to \infty,
$$
то есть числовая последовательность $\lbrace \|A_n\| \rbrace$ фундаментальна, а поэтому и ограничена. Значит, существует такая константа $C$, что $\|A_n\| \leqslant C$ для всех $n$. Следовательно, для всех $n$ справедливо и неравенство $\|A_n x\| \leqslant \|A_n\| \|x\| \leqslant C \|x\|$. Получаем, что
$$
\|Ax\| = \lim_{n\to\infty} \|A_n x\| \leqslant C \|x\|.
$$
А это как раз и означает, что оператор $A$ является ограниченным. Кроме того, оператор $A$, очевидно, линеен. Таким образом, $A$ принадлежит рассматриваемому пространству линейных ограниченных операторов.\\

Далее, для любого $\varepsilon > 0$ найдётся номер $m$ такой, что 
$$
\forall x \in X, \|x\| \leqslant 1 \quad \|A_m x - A_n x\| \leqslant \|A_m - A_n\| \|x\| < \varepsilon , \varepsilon > 0; m, n \geqslant N.
$$
Переходя в этом неравенстве к пределу при $m \to \infty$, получаем, что
$$
\forall x \in X, \|x\| \leqslant 1, \; \forall n \geqslant N \quad \|Ax - A_n x\| \leqslant \varepsilon. 
$$
Но тогда для $n \geqslant N$
$$
\|A - A_n\| = \sup_{\|x\| \leqslant 1} \|Ax - A_n x\| \leqslant \varepsilon.
$$
Следовательно, линейный ограниченный оператор $A = \lim\limits_{n\to\infty} A_n$ в смысле сходимости по норме в рассматриваемом пространстве линейных ограниченных операторов, поэтому это пространство является банаховым. $\quad \Box$\\






%%%%%%%%%%%%%%%%%%%%%%%
%      Lecture 9      %
%%%%%%%%%%%%%%%%%%%%%%%






\textbf{Определение.} Пространство $X^* = L(X, \mathbb{R})$ называется \textit{сопряженным} для линейиного нормированного пространства $X$.

\textbf{Следствие из теоремы 3.} Пространство $X^*$, сопряжённое с линейным нормированным пространством $X$, является банаховым пространством.\\

\textbf{Теорема 4 (теорема Банаха-Штейнгауза, принцип равномерной ограниченности).}
Пусть $X$ и $Y$ --- банаховы пространства, на которых задана последовательность линейных ограниченных операторов: $\{A_n\} \in L(X\to Y)$. Тогда, если последовательность $\{\|A_n x\|\}$ ограничена для любого $x\in X$, то последовательность норм операторов также будет ограниченной, то есть найдётся константа $C$ такая, что $\|A_n\| \leqslant C$.\\
\textbf{Доказательство.} Предположим обратное. Тогда множество $\lbrace \|A_n x\| \rbrace$ не ограничено на любом замкнутом шаре. Действительно, если $\|A_n x\| \leqslant c$ для всех $n$ и всех $x$ из некоторого замкнутого шара $\overline{S(x_0, \varepsilon)}$, то для любого $\xi \in X$, $\xi \neq 0$, элемент
$$
\frac {\xi \cdot \varepsilon}{\|\xi\|} + x_0
$$
принадлежит этому шару и, следовательно, для него при всех $n$ выполняются неравенства $\|A_n x\| \leqslant M$. Тогда
$$
\frac{\|A_n \xi\| \cdot \varepsilon} {\|\xi\|} - \|A_n x_0\| \leqslant \left\| \frac{\|A_n \xi\| \cdot \varepsilon} {\|\xi\|} + A_n x_0 \right\| \leqslant M.
$$
Отсюда получаем, что
$$
\|A_n \xi\| \leqslant \frac{M + \|A_n x_0\|}{\varepsilon} \|\xi\|.
$$
Последовательность $\|A_n x\|$ ограничена, поэтому для всех $n$ будет справедливо неравенство $\|A_n \xi\| \leqslant \frac{2M}{\varepsilon}\|\xi\|$. Но из него следует, что $\|A_n\| \leqslant \frac{2M}{\varepsilon}$, что противоречит сделанному нами предположению. Итак, при нашем предположении множество $\lbrace \|A_n x\| \rbrace$ не ограничено на любом замкнутом шаре.\\

Пусть теперь $\overline{S}{}_0 (x_0, r_0)$ --- произвольный замкнутый шар в $X$. Тогда в силу того, что $\lbrace \|A_n x\| \rbrace$ не ограничена на этом шаре, существуют номер $n$ и элемент $x_1 \in \overline{S}{}_0$ такие, что $\|A_{n_1}x_1\| > 1$. Так как оператор $A_{n_1}$ непрерывен, то это неравенство выполняется в некотором замкнутом шаре $\overline{S}{}_1 (x_1, r_1) \subset \overline{S}{}_0, r_1 < 1$. На $\overline{S}{}_1$ последовательность $\lbrace \|A_n x\| \rbrace$ снова не ограничена, поэтому снова найдутся номер $n_2 > n_1$ и элемент $x_2 \in \overline{S}{}_1$ такие, что $\|A_{n_2}x_2\| > 2$. Опять же, в силу непрерывности оператора $A_{n_2}$ это свойство выполняется в некотором замкнутом шаре $\overline{S}{}_2 (x_2, r_2) \subset \overline{S}{}_1, r_2 < \frac12$ и так далее ($r_n < \frac1n$).\\

Таким образом, мы получаем последовательность вложенных друг в друга замкнутых шаров, радиусы которых стремятся к $0$ при $n\to\infty$. Следовательно, будет существовать точка $x$, принадлежащая всем этим шарам. Но тогда в этой точке для всех $k$ справедливо неравенство $\|A_{n_k}x\| > k$, а это противоречит условию теоремы о том, что для любого $x\in X$ последовательность $\|A_n x\|$ является ограниченной. Следовательно, наше предположение неверно. Теорема доказана. $\quad \Box$\\

\textbf{Следствие.} Пусть $X$ и $Y$ --- банаховы пространства, задана последовательность линейных ограниченных операторов $A_n \in L(X\to Y)$ и существует последовательность $\lbrace x_n \rbrace$, $x_n \in X$ такая, что $\|x_n\| \leqslant 1$, а $\|A_n x_n\| \to +\infty$ при $n\to\infty$. Тогда найдётся элемент $x \in X$, $\|x_0\| \leqslant 1$, такой, что $\overline{\lim\limits_{n\to\infty}} \|A_n x_0\| = +\infty$.\\
\textbf{Доказательство.} Пусть это не так, то есть для всех $x \in X$ с $\|x_0\| \leqslant 1$ последовательность $\lbrace \|A_n x\| \rbrace$ ограничена. Тогда при $\xi \neq 0$ элемент $x = \frac{\xi}{\|\xi\|}$ будет иметь норму $\|x\| = 1$, а также будет справедливо неравенство
$$
\frac{\|A_n\xi\|}{\|\xi\|} = \|A_n x\| \leqslant M \text{, где $M$ --- некоторая константа}.
$$
Следовательно, $\|A_n\xi\| \leqslant M \|\xi\|$, то есть для любого $\xi \in X$ последовательность $\lbrace \|A_n\xi\| \rbrace$ является ограниченной. Следовательно, в силу теоремы Банаха-Штейнгауза найдётся константа $C$ такая, что $\|A_n\| \leqslant C$. Получаем, что
$$
\|A_n x_n\| \leqslant \|A_n\| \|x_n\| \leqslant C,
$$
что противоречит условию. Значит, наше предположение неверно. $\quad \Box$\\

\textbf{Пример.} Используя теорему Банаха-Штейнгауза, докажем существование непрерывной периодической функции, для которой ряд Фурье расходится. Итак, пусть $f(x) \in \mathbb C [-\pi ; \pi]$, $f(-\pi) = f(\pi)$. Сумма ряда Фурье этой функции имеет вид
$$
S_n(x) = \frac {a_0} 2 + \sum_{k=1}^n a_k \cos kx + b_k \sin kx,
$$
где
$$
a_k = \frac 1 \pi \int\limits_{-\pi}^\pi f(t) \cos kt dt, \quad b_k = \frac 1 \pi \int\limits_{-\pi}^\pi f(t) \sin kt dt.
$$
Перепишем $S_n(x)$:
\begin{multline*}
S_n(x) = \frac {a_0} 2 + \sum_{k=1}^n a_k \cos kx + b_k \sin kx = \\
= \frac 1 {2\pi} \int\limits_{-\pi}^\pi f(t) dt +
\sum_{k=1}^\infty \frac 1 \pi \int\limits_{-\pi}^\pi \cos k(t-x) f(t) dt = \\
= \frac 1 {2\pi} \int\limits_{-\pi}^\pi \frac{\sin (n+ \frac 1 2)(t-x)}{\sin \frac{t-x} 2 } f(t) dt.
\end{multline*}
Определим функцию $g(t)$ следующим образом:
$$
g(t) = \begin{cases}
\dfrac 1 {2 \tan \frac t 2 } - \dfrac 1 t, & t \neq 0; \\
0, & t = 0.
\end{cases}
$$

Тогда
\begin{multline*}
S_n(0) = \frac 1 {2\pi} \int\limits_{-\pi}^\pi \frac {\sin (n+ \frac 1 2) t}{\sin \frac t 2} f(t) dt =  \\
= \frac 1 \pi \int\limits_{-\pi}^\pi \frac{\sin nt} t f(t) dt +
\frac 1 \pi \int\limits_{-\pi}^\pi g(t) \sin nt f(t) dt +
\frac 1 {2\pi} \int\limits_{-\pi}^\pi \cos nt f(t) dt = \\
= \frac 1 {\pi} \int\limits_{-\pi}^{\pi} \frac{\sin nt}{t} f(t) dt + o(1), \text{ где $o(1)\to 0$ при $n\to\infty$}.
\end{multline*}
Рассмотрим оператор
$$
A_n f(x) = \frac 1 \pi \int\limits_{-\pi}^\pi \frac {\sin nt} t f(t) dt.
$$
Оператор $A_n$ действует из пространства $\widetilde C$ (пространства непрерывных периодических функций) в пространство $R_1$ и любой функции $f(x) \in \widetilde C$ ставит в соответствие частичную сумму её ряда Фурье (с точностью до $O(1)$). Рассмотрим следующую последовательность функций $f_n$:
$$
f_n(x) = \mathrm{sgn} \ x \cdot \sin nx, \quad \|f_n\| \leqslant 1.
$$
Тогда
\begin{multline*}
A_n f_n = \frac 1 \pi \int\limits_{-\pi}^\pi \frac {\sin^2 nt}{|t|} dt =
\frac 2 \pi \int\limits_0^\pi \frac {\sin^2 nt} t dt = \\
= \frac 2 \pi \int\limits_0^{\pi n} \frac {\sin^2 y} y dy >
\frac 1 \pi \int\limits_1^{\pi n} \frac{1 - \cos 2y}{y}dy = \\
= \frac 1 \pi \ln \pi n - \frac 1 \pi \int\limits_1^{\pi n} \frac{\cos 2y}{y}dy.
\end{multline*}
Интеграл
$$
\frac 1 \pi \int\limits_1^{\pi n} \frac{\cos 2y}{y}dy
$$
сходится, поэтому
$$
A_n f_n = \frac 1 \pi \ln \pi n + O(1).
$$
Получаем, что при $n\to\infty$ $A_n f_n \to \infty$. Следовательно, в силу следствия к теореме 4 существует такая функция $f_0(x) \in \widetilde C$, что $\limsup\limits_{n\to\infty} A_n f_0(x) = +\infty$, то есть ряд Фурье этой функции в нуле расходится. $\quad \Box$\\


\section*{ \S 8. Обратные операторы}

Пусть есть два линейных нормированных пространства --- $X$ и $Y$. Рассмотрим оператор $A: X \to Y$ с областью определения $D(A) = X$ и областью значений $R(A) \subset Y$.\\

Если для любого $y \in R(A)$ уравнение $Ax = y$ имеет единственное решение, то говорят, что определен \textit{обратный оператор} $A^{-1}: R(A) \to X$, то есть $X = A^{-1} Y$. Очевидно, что $A A^{-1} = E$ и $A^{-1} A = E$ --- тождественные операторы на $R(A)$ и $X$ соответственно.\\

Если для оператора $A: X \to Y$ существует оператор $A^{-1}: R(A) \to X$ такой, что
$$
A^{-1} A x = x, \quad A A^{-1} y = y,
$$
то операторы $A$ и $A^{-1}$ называются \textit{взаимно обратными}. Если выполняется только неравенство $A^{-1} A x = x$, то оператор $A^{-1}$ называется \textit{левым обратным} оператором для $A$; если выполняется только неравенство $A A^{-1} y = y$, то оператор $A^{-1}$ называется \textit{правым обратным} оператором для $A$.\\

Легко показать, что оператор, обратный к линейному, также является линейным. Пусть оператор $A$ является линейным. Рассмотрим
$$
x = A^{-1} (\alpha y_1 + \beta y_2) - \alpha A^{-1} y_1 - \beta A^{-1} y_2.
$$
Тогда
\begin{multline*}
Ax = A A^{-1} (\alpha y_1 + \beta y_2) - A (\alpha A^{-1} y_1 )- A(\beta A^{-1} y_2) = \\
= A A^{-1} (\alpha y_1 + \beta y_2) - \alpha A A^{-1} y_1 - \beta A A^{-1} y_2 = \\
= (\alpha y_1 + \beta y_2) - \alpha y_1 - \beta y_2 = 0. 
\end{multline*}
Cледовательно, 
$$
x = A^{-1} Ax = A^{-1} 0 = 0,
$$
то есть
$$
A^{-1} (\alpha y_1 + \beta y_2) = \alpha A^{-1} y_1 + \beta A^{-1} y_2.
$$
Таким образом, оператор $A^{-1}$ также является линейным.\\

\textbf{Теорема 1.} Пусть A --- линейный оператор, отображающий линейное нормированное пространство $X$ на линейное нормированное пространство $Y$, причём существует такая константа $m > 0$, что $\|Ax\| \geqslant m \|x\|$ для всех $x \in X$. Тогда существует обратный линейный ограниченный оператор $A^{-1}$, $\|A^{-1} y \| \leqslant \frac 1 m \|y\|$.\\
\textbf{Доказательство.} Докажем, что уравнение $Ax = y$ имеет единственное решение. Предположим, что их два: $y = Ax_1 = Ax_2$. Тогда $Ax_1 - Ax_2 = A(x_1 - x_2) = 0$. Следовательно,
$$
m \|x_1 - x_2\| \leqslant \|Ax_1 - Ax_2\| = 0 \quad \Rightarrow \quad x_1 = x_2.
$$
Таким образом, существует обратный оператор $A^{-1}$. Он линеен в силу линейности оператора $A$ и, кроме того, ограничен, так как для всех $y \in R(A)$ справедливо неравенство
$$
\|A^{-1} y \| \leqslant \frac 1 m \|A A^{-1} y\| = \frac 1 m \|y\|. \quad \Box
$$

\textbf{Теорема 2 (теорема Неймана).} Пусть $X$ --- банахово пространство, оператор $A \in L (X \to X)$, и пусть $\|A\| \leqslant q < 1$. Тогда оператор $(I - A)$ имеет обратный линейный ограниченный оператор $(I - A)^{-1}$, $\|(I - A)^{-1}\| \leqslant \frac 1 {1-q}$.\\
\textbf{Доказательство.} Определим операторы, являющиеся степенями оператора $A$, следующим образом:
$$
A^0 = I, \quad A^n = A(A^{n-1}) \text{ при } n = 1, \ 2, \ \dots.
$$

Для линейных ограниченных операторов $A$ и $B$ в банаховом пространстве $X$ справедливо неравенство $\|AB\| \leqslant \|A\|\|B\|$, поскольку для любого $x \in X$ справедливо неравенство
$$
\|ABx\| \leqslant \|A\| \ \|Bx\| \leqslant \|A\| \ \|B\| \ \|x\|. 
$$
Поэтому $\|AA\| \leqslant \|A\| \ \|A\| = \|A\|^2$, и аналогично $\|A^n\| \leqslant \|A\|^n \leqslant q^n$ для всех $n$.\\

Введём оператор $S_n = \sum\limits_{k=0}^n A^k$. Тогда
$$
(I - A)S_n = (I - A) \sum_{k=0}^n A^k  = I - A^{n+1}.
$$
Но $A^{n+1} \to 0$ при $n\to\infty$, так как $\|A^{n+1}\| \leqslant \|A\|^{n+1} \leqslant q^{n+1} \to 0$ при $n\to\infty$ в силу $q < 1$. Следовательно, существует обратный оператор $S = (I - A)^{-1} = \sum\limits_{k=0}^\infty A^k$.\\

Легко  заметить, что оператор $S$ является линейным. Кроме того, он является ограниченным:
$$
\|S\| = \left\| \sum_{k=0}^\infty A^k \right\| \leqslant \sum_{k=0}^\infty \|A^k\| \leqslant \sum_{k=0}^\infty \|A\|^k \leqslant \sum_{k=0}^\infty q^k = \frac 1 {1 - q}. \quad \Box
$$

\textbf{Теорема 3.} Пусть $X$ --- банахово пространство, $A, A^{-1} \in L(X\to X)$ и существует линейный ограниченный оператор $\Delta A$ такой, что $ \|\Delta A\| < \frac 1 {\|A^{-1}\|}$. Тогда оператор $B = A + \Delta A$ имеет обратный оператор $B^{-1}$, причём 
$$
\|B^{-1} - A^{-1}\| \leqslant \frac {\|\Delta A\| \|A^{-1}\|^2} {1 - \|\Delta A\| \|A^{-1}\|}.
$$
\textbf{Доказательство.} Представим оператор $B$ в виде $A + \Delta A = A(I + A^{-1} \Delta A)$. Из условия задачи следует, что
$$
\|A^{-1} \Delta A\| \leqslant \|A^{-1}\| \|\Delta A\| < 1.
$$
Следовательно, в силу теоремы Неймана у оператора $I + A^{-1} \Delta A$ существует обратный оператор $(I + A^{-1} \Delta A)^{-1}$. Следовательно, произведение $(I + A^{-1} \Delta A)^{-1} A^{-1}$ является обратным оператором к $B$, и при этом справедливы следующие неравенства:
\begin{multline*}
\|B^{-1} - A^{-1}\|  = \|(I + A^{-1} \Delta A)^{-1} A^{-1} - A^{-1}\| \leqslant \\
\leqslant \|A^{-1}\| \|(I + A^{-1} \Delta A)^{-1} - I\| \leqslant
\|A^{-1}\| \sum_{n=1}^\infty \|(A^{-1} \Delta A)^n \| = \\
= \|A^{-1}\| \frac{\|A^{-1}\Delta A\|} {1 - \|A^{-1}\Delta A\| } \leqslant
\|A^{-1}\| \frac{\|A^{-1}\|\|\Delta A\|} {1 - \|A^{-1}\| \|\Delta A\| }. \quad \Box
\end{multline*}

\textbf{Теорема 4 (теорема Банаха об обратном операторе).} Пусть $X$ и $Y$ --- банаховы пространства, линейный ограниченный оператор $A$ отображает всё пространство $X$ на всё пространство $Y$ взаимно однозначно. Тогда существует обратный линейный ограниченный оператор $A^{-1}$.\\
\textbf{Доказательство.} Так как линейный оператор $A$ осуществляет взаимно однозначное соответствие между элементами пространств $X$ и $Y$, то он, очевидно, будет иметь обратный оператор $A^{-1}$, также являющийся линейным. Остаётся доказать ограниченность оператора $A^{-1}$.\\

Введём в рассмотрение множества $Y_n = \{ y \in Y: \; \|A^{-1} y \| \leqslant n \|y\|\}$. Любой элемент $y \in Y$ попадёт в множества $Y_n$ при целочисленных $n > \frac {\|A^{-1}y\|} {\|y\|}$. Следовательно, все пространство $Y$ можно представить в виде
$$
Y = \bigcup_{n=1}^\infty Y_n.
$$
Пространство $Y$ является банаховым, поэтому в силу теоремы Бэра о категориях оно не может быть представлено в виде счётного числа нигде не плотных множеств. Следовательно, найдётся хотя бы один номер $n_0$ такой, что множество $Y_{n_0}$ не является нигде не плотным. Это значит, что существует шар $S(y_0, r_0)$, в котором множество $S(y_0, r_0) \cap Y_{n_0}$ является всюду плотным, то есть $\overline{S(y_0, r_0) \cap Y_{n_0}} = S(y_0, r_0)$. Производя замыкание левой и правой частей этого множества, получим $\overline{\overline{S(y_0, r_0) \cap Y_{n_0}}} = \overline{S(y_0, r_0) \cap Y_{n_0}} = \overline{S(y_0, r_0)}$.\\

Рассмотрим замкнутый шар $\overline{S(y_1, r_1)} \subset S(y_0, r_0)$ и такой, что $y_1 \in Y_{n_0}$. Тогда справедливо неравенство
$$
\overline{Y_{n_0}} \supset \overline{S(y_0, r_0) \cap Y_{n_0}} = \overline{S(y_0, r_0)} \supset \overline{S(y_1, r_1)}.
$$
Следовательно, $\overline{S(y_1, r_1)} \subset \overline{Y_{n_0}}$. Возьмём произвольный элемент $y$ с нормой $\|y\| = r_1$. Тогда элемент $y + y_1 \in \overline{S(y_1, r_1)}$, так как $\|(y + y_1) - y_1\| = r_1$. В силу неравенства $\overline{S(y_1, r_1)} \subset \overline{Y_{n_0}}$ найдётся последовательность элементов $\lbrace z^{(k)} \rbrace$ из $S(y_1, r_1) \cap Y_{n_0}$ такая, что $z^{(k)} \to y + y_1$ при $k \to \infty$. Обозначим $y^{(k)} = z^{(k)} - y_1$, тогда $y^{(k)} \to y$ при $k \to \infty$. Кроме того, $\|y\| = r_1$, поэтому при $k \geqslant K$ справедливы неравенства $\frac {r_1} 2 \leqslant \|y^{(k)}\| \leqslant r_1$.\\

Так как $z^{(k)}$ и $y_1$ принадлежат $Y_{n_0}$, то
\begin{multline*}
\|A^{-1} y^{(k)}\| \leqslant \|A^{-1} z^{(k)}\| + \|A^{-1} y_1\| \leqslant \\
\leqslant n_0(\|z^{(k)}\| + \|y_1\|) \leqslant n_0(\|y^{(k)}\| + 2\|y_1\|) \leqslant \\
\leqslant \frac {2n_0} {r_1} (r_1 + 2\|y_1\|) \frac {r_1} 2 \leqslant \frac {2n_0} {r_1} (r_1 + 2 \|y_1\|) \|y^{(k)}\|.
\end{multline*}
Обозначим за $N$ наименьшее целое число, превосходящее
$$
\frac {2n_0} {r_1} (r_1 + 2 \|y_1\|) \|y^{(k)}\|.
$$
Тогда
$$
\|A^{-1} y^{(k)}\| \leqslant N \|y^{(k)}\|,
$$
поэтому все $y^{(k)} \in Y_{N}$. Другими словами, любой элемент $y \in Y$ с нормой $\|y\| = r_1$, можно аппроксимировать элементами $y^{(k)} \in Y_{N}$ ($y^{(k)} \to y$ при $k \to \infty$). Возьмём теперь произвольный элемент $y \in Y$. Рассмотрим элемент
$$
\xi = r_1 \frac y {\|y\|}.
$$
Получаем, что $\|\xi\| = r_1$. Поэтому найдётся последовательность $\lbrace \xi^{(k)} \rbrace \subset Y_{N}$, сходящаяся к $\xi$. Но тогда последовательность
$$
y^{(k)} = \xi^{(k)} \frac {\|y\|}{r_1} \to y.
$$
При этом
$$
\|A^{-1} y^{(k)}\| = \frac{\|A^{-1} \xi^{(k)}\| \|y\|}{r_1} \leqslant N \|\xi^{(k)}\| \frac{\|y\|}{r_1} = N \|y^{(k)}\|,
$$
то есть все $y^{(k)} \in Y_N$. Таким образом, пространство $Y_N$ является всюду плотным в $Y$.\\

Возьмём произвольный элемент $y \in Y$; пусть $ \|y\| = l$. Найдём элемент $y_1 \in Y_N$ такой, что
$$
\|y - y_1\| \leqslant \frac l 2, \quad \|y_1\| \leqslant l.
$$
Это можно сделать в силу того, что $y \in \overline{S(0, l)}$ и множество $\overline{S(0, l)} \cap Y_N$ является всюду плотным в $\overline{S(0, l)}$. Аналогично найдём элемент $y_2 \in Y_N$ такой, что
$$
\|(y - y_1) - y_2\| \leqslant \frac l 4, \quad \|y_2\| \leqslant \frac l 2.
$$
Продолжая так далее, построим элементы $y_k \in Y_N$ такие, что
$$
\|y - (y_1 + y_2 + \dots + y_k)\| \leqslant \frac l {2^k}, \quad \|y_k\| \leqslant \frac l {2^{k-1}}.
$$
Следовательно,
$$
y = \sum_{i = 1}^{\infty} y_i.
$$
Положим $x_k = A^{-1} y_k$, тогда
$$
\|x_k\| \leqslant N \|y_k\| \leqslant \frac {Nl} {2^{k-1}}.
$$
Это означает, что последовательность $\lbrace s_k \rbrace$, $s_k = \sum_{i = 1}^k x_i$ в силу полноты пространства $X$ сходится к некоторому пределу $x \in X$ при $k\to\infty$, так как является фундаментальной:
$$
\|s_{k+p} - s_k\| = \left\| \sum_{i = k+1}^{k+p} x_i \right\| < \frac {Nl} {2^{k-1}}.
$$
Следовательно,
$$
x = \lim_{k\to\infty} \sum_{i=1}^k x_i = \sum_{i=1}^\infty x_i.
$$
Тогда
$$
Ax = A \left( \lim_{k\to\infty} \sum_{i=1}^k x_i \right) = \lim_{k\to\infty} \sum_{i=1}^k Ax_i = \lim_{k\to\infty} \sum_{i = 1}^{\infty} y_i = y,
$$
поэтому
\begin{multline*}
\|A^{(-1)} y\| = \|x\| = \\
= \lim_{k\to\infty} \left\| \sum_{i=1}^k x_i \right\| \leqslant \lim_{k\to\infty} \sum_{i=1}^k \|x_i\| \leqslant
\lim_{k\to\infty} \sum_{i=1}^\infty \frac {Nl} {2^{i-1}} = \\
= 2Nl = 2N \|y\|.
\end{multline*}
Это и означает, что оператор $A^{-1}$ является ограниченным. $\quad \Box$\\






%%%%%%%%%%%%%%%%%%%%%%%
%     Lecture 10      %
%%%%%%%%%%%%%%%%%%%%%%%






\section*{ \S 9. Линейные функционалы}

Линейный непрерывный оператор, значения которого принадлежат пространству $\mathbb{R}_1$, называется \textit{линейным функционалом}.
$$
L: X \to \mathbb{R}_1.
$$

\textbf{Теорема 1 (теорема Хана-Банаха о продолжении линейного функционала).} Пусть $X$ --- линейное нормированное пространство, $L \subset X$ - линейное многообразие, на котором задан линейный функционал $f(x)$. Тогда $f(x)$ можно продолжить на всё пространство $X$ с сохранением нормы, то есть на $X$ существует линейный функционал $F(x)$ такой, что:
\begin{enumerate}
\item
$F(x)$ = $f(x)$ на L;
\item
$\|F\|_X = \|f\|_L$.
\end{enumerate}

\textbf{Доказательство.}
А) Возьмём элемент $x_0 \notin L$ и рассмотрим множество $L_0 = (L, x_0)$ элементов $u$ вида $u = x + tx_0$, где $x \in L$, а $t \in \mathbb{R}$ --- произвольное вещественное число. Очевидно,что $L_0$ является линейным многообразием. Докажем, что все его элементы однозначно представимы в виде $x + tx_0$. Допустим, имеются два представления
$$
u = x_1 + t_1 x_0 = x_2 + t_2 x_0,
$$
причём $t_1 \neq t_2$ (иначе из $x_1 + t_1 x_0 = x_2 + t_1 x_0$ следовало бы, что $x_1 = x_2$, то есть представление было бы единственным). Тогда
$$
x_2 - x_1 = (t_1 - t_2) x_0 \quad \Rightarrow \quad x_0 = \frac{x_2 - x_1}{t_1 - t_2}.
$$
Но $x_1, x_2 \in L$, поэтому и $x_0$ должен принадлежать $L$, что невозможно. Мы получили противоречие; следовательно, наше предположение неверно и представления элементов $L_0$ единственны.\\

Возьмём два элемента $x_1, x_2 \in L$. Имеем
$$
f(x_1) - f(x_2) = f(x_1 - x_2) \leqslant \|f\| \ \|x_1 - x_2\| \leqslant \|f\| \ [\|x_1 + x_0\| + \|x_2 + x_0\|].
$$
Отсюда
$$
f(x_1) - \|f\| \ \|x_1 + x_0\| \leqslant f(x_2) + \|f\| \|x_2 + x_0\|.
$$
Поскольку $x_1$ и $x_2$ --- произвольные элементы $L$, выбранные независимо друг от друга, то
$$
\sup_{x\in L} \left\{ f(x) - \|f\| \|x + x_0\| \right\} \leqslant
\inf_{x \in L} \left\{ f(x) + \|f\| \|x + x_0\| \right\}.
$$
Следовательно, существует вещественное число $c$ такое, что
$$
\sup_{x\in L} \left\{ f(x) - \|f\| \|x + x_0\| \right\} \leqslant c \leqslant
\inf_{x \in L} \left\{ f(x) + \|f\| \|x + x_0\| \right\}.
$$
Возьмём теперь произвольный элемент $u \in L_0$. Он имеет вид $u = x + tx_0$, где $x \in L$ и $t \in \mathbb{R}$ однозначно определены. Введём новый функцинал $\varphi(u)$, определив его для элемента $u = x + tx_0$ равенством
$$
\varphi(u) = f(x) - tc,
$$
где $c$ --- вещественное число, удовлетворяющее приведённому выше двойному неравенству.\\

Очевидно, что функционал $\varphi(u)$ является аддитивным ($\varphi(u_1 + u_2) = \varphi(u_1) + \varphi(u_2)$), а на $L$ совпадает с функционалом $f(x)$. Докажем, что $\varphi(u)$ ограничен и его норма совпадает с нормой $f(x)$.\\
Рассмотрим два случая:
\begin{enumerate}
\item $t > 0:$\\
$$
\varphi(u) = t \left( f \left(\frac{x}{t}\right) - c \right) \leqslant t \ \|f\| \ \|\frac{x}{t} + x_0\| =
\|f\| \ \|x + tx_0\| = \|f\| \ \|u\|.
$$

\item $t < 0:$\\
$$
\varphi(u) = t \left( f \left(\frac{x}{t}\right) - c \right) \leqslant -t \ \|f\| \ \|\frac{x}{t} + x_0\| = \|f\| \ \|u\|.
$$
\end{enumerate}
Таким образом, неравенство $\varphi(u) \leqslant \|f\| \ \|u\|$ справедливо для всех $u \in L_0$. Заменяя в нём $u$ на $(-u)$, получим неравенство $-\varphi(u) \leqslant \|f\| \ \|u\|$. Следовательно, и $|\varphi (u)| \leqslant \|f\| \ \|u\|$. Это значит, что $\|\varphi\| \leqslant \|f\|$. Поскольку функционал $\varphi$ является продолжением функционала $f$ с $L$ на $L_0$, верно и неравенство $\|\varphi\| \geqslant \|f\|$. Следовательно, $\|\varphi\| = \|f\|$.\\

Пространство называется \textit{сепарабельным}, если в нем существует счётное всюду плотное множество.\\

Завершение доказательства теоремы проведём только для случая, когда пространство $X$ является сепарабельным.\\

B) Так как $X$ сепарабельно, в нем существует счётное всюду плотное множество. Возьмём все элементы этого множества, не попавшие в $L$, и перенумеруем их:
$x_1, \ x_2, \ \dots \ .$
Построим соответствующие множества $L_k$ следующим образом:
$$
L_1 = (L_0; x_1), \ L_2 = (L_1; x_2), \ \dots \ .
$$
Эти множества являются линейными многообразиями, поэтому мы можем построить функционал $\varphi(u)$, являющийся продолжением функционала $f(x)$ с $L$ на $\widehat L = \bigcup\limits_{k=1}^\infty L_k$, причём $\|\varphi\| = \|f\|$.\\

Продолжим функционал $\varphi(u)$ на всё пространство $X$ по непрерывности. Если элемент $x \in X$, но $x \notin \widehat L$, то в силу того, что $\widehat L$ всюду плотно в $X$, найдётся последовательность $\lbrace \tilde{x}{}_k \rbrace$ элементов $\tilde{x}{}_k \in \widehat L$ такая, что при $n\to\infty$ $\tilde{x}{}_k \to x$ ($\|\tilde{x}{}_k - x\| \to 0$). Тогда
\begin{multline*}
|\varphi (\tilde{x}{}_n) - \varphi (\tilde{x}{}_m)| = |\varphi (\tilde{x}{}_n - \tilde{x}{}_m)| \leqslant \\
\leqslant \|\varphi\| \ \|\tilde{x}{}_n - \tilde{x}{}_m\| = \|f\| \ \|\tilde x_n - \tilde x_m\| \to 0 \text{ при } n,m \to \infty.
\end{multline*}
Таким образом, последовательность $\lbrace \varphi (\tilde{x}{}_n) \rbrace$ является фундаментальной, а потому сходится к некоторому пределу
$$
F(x) = \lim_{n\to \infty} \varphi (\tilde{x}{}_n).
$$
Кроме того,
$$
|\varphi (\tilde{x}{}_n)| \leqslant \|f\| \ \|\tilde x_n\| \quad \Rightarrow \quad |F(x)| \leqslant \|f\| \ \|x\|.
$$
Из этого следует, что $\|F\| \leqslant \|f\|$. Но, с другой стороны, функционал $F(x)$ является продолжением функционала $f(x)$ с $L$ на $X$, поэтому $\|F\| \geqslant \|f\|$. Следовательно, $\|F\| = \|f\|$. Искомый функционал построен. $\quad \Box$\\

\textbf{Следствие 1.} Пусть $X$ --- линейное нормированное пространство, $x_0 \neq 0$ --- произвольный элемент $X$. Тогда существует линейный функционал $f(x)$, определённый на всём пространстве $X$ и такой, что
\begin{enumerate}
\item
$\|f\| = 1;$
\item
$|f(x_0)| = \|x_0\|.$
\end{enumerate}

\textbf{Доказательство.} Рассмотрим линейное многообразие $L = \lbrace t x_0 \rbrace$, где $t$ пробегает всевозможные вещественные числа. Множество $L$ является подпространством пространства $X$, определяемым элементом $x_0$. Определим на $L$ функционал $\varphi(x)$ следующим образом: если $x = tx_0$, то
$$
\varphi (x) = t \|x_0\|.
$$
Очевидно, что
\begin{enumerate}
\item
$\varphi (x_0) = \|x_0\|;$

\item
$|\varphi (x)| = |t| \ \|x_0\| = \|x\|$, откуда $\|\varphi \| = 1$
\end{enumerate}

Продолжая функционал $\varphi(x)$ по теореме Хана-Банаха на всё пространство $X$, получим функционал $f(x)$, имеющий требуемые свойства. $\quad \Box$\\

\textbf{Следствие 2.} Пусть $X$ --- линейное нормированное пространство, элементы $x_1, x_2 \in X$ и $x_1 \neq x_2$. Тогда существует линейный функционал $f(x)$ такой, что $f(x_1) \neq f(x_2)$, $\|f\| = 1$.\\
\textbf{Доказательство.} Положим $x_0 = x_1 - x_2$. Тогда существование требуемого функционала вытекает из следствия 1.\\

\textbf{Теорема 2.}
Пусть $X$ --- банахово пространство, $\lbrace x_n\rbrace$ --- последовательность элементов из $X$ такая, что последовательность $\lbrace f(x_n)\rbrace$ ограничена для любого функционала $f \in X^*$. Тогда последовательность $\lbrace x_n\rbrace$ ограничена в $X$, то есть существует константа $C > 0$ такая, что $\|x_n\| \leqslant C$.\\
\textbf{Доказательство.} Рассмотрим последовательность операторов $A_n$ на $X^*$, таких что $A_nf = f(x_x)$.\\
Поскольку $X^*$ всегда является банаховым пространством, то по теореме Банаха-Штейнгауза $\|A_n\| \leqslant M$. \\
$\|A_n\| = \sup\limits_{\|f'\| = 1}|f'(x_n)|$. \\
По следствию из теоремы 1 существует функционал $f_0$, такой что $f_0(x_n) = \|x_n\|, \|f\| = 1$. \\
Тогда $\|x_n\| = f_0(x_n) \leqslant \|A_n\| \leqslant M$.

\textbf{Теорема 3.} Пусть $X$ --- банахово пространство, на котором задана последовательность линейных функционалов $\lbrace f_n \rbrace$, причём при любом $x \in X$ последовательность $\lbrace f_n \rbrace$ является ограниченной. Тогда найдётся константа $C > 0$ такая, что $\|f_n\| \leqslant C$. \\
\textbf{Доказательство.} Частный случай теоремы Банаха-Штейнгауза.\\

\textbf{Определение 1.} Последовательность $\lbrace x_n \rbrace$ элементов линейного нормированного пространства $X$ называется слабо сходящейся к элементу $x \in X$, если для любого линейного функционала $f \in X^*$ последовательность $\lbrace f(x_n) \rbrace$ сходится к $f(x)$ при $n\to\infty$.\\

\textbf{Утверждение.} Слабый предел у последовательности может быть только один. \\
\textbf{Доказательство.} Вытекает из следствия 2 к теореме Хана-Банаха: для двух различных пределов существует линейный оператор, который принимает на них разные значения. \\

\textbf{Следствие из теоремы 2.} Слабо сходящаяся последовательность ограничена.\\

Из сильной сходимости вытекает слабая сходимость, в силу неравенства
$$
|f(x_n) - f(x)| \leqslant \|f\| \|x_n - x\|.
$$
Из слабой сходимости, вообще говоря, сильная сходимость не вытекает (правда, в конечномерном пространстве эти сходимости равносильны, но в бесконечномерном пространстве это не так).

~

(Следующий пример не проверял --- ред.)

Например, рассмотрим в пространстве $l_p$, $p > 1$, последовательность элементов $e_n = (0, \ \dots , \ 0, \ 1, \ 0, \ \dots)$, где единица стоит на позиции с номером $n$; $f(e_n) = c_n$. Ряд $\sum\limits_{n=1}^\infty |c_n|^p$ сходится, поэтому $f(e_n) = c_n \to 0 = f(0)$ при $n\to\infty$, то есть последовательность $\lbrace x_n \rbrace$ слабо сходится к нулю. Но сильной сходимости здесь нет, так как последовательность $\lbrace x_n \rbrace$ не является фундаментальной:
$$
\|e_k - e_m\|_p = 2^{\frac 1 p} \not\to 0 \text{ при } k,m \to \infty.
$$

\textbf{Теорема 4.} Для того, чтобы слабо сходящаяся последовательность $\lbrace x_n \rbrace$ в банаховом пространстве $X$ являлась сильно сходящейся, необходимо и достаточно, чтобы для всех $f \in X^*$, $\|f\| \leqslant 1$, последовательность $\lbrace f(x_n) \rbrace$ сходилась равномерно.\\
\textbf{Доказательство.} Сначала докажем необходимость. Если $\lbrace x_n \rbrace$ сильно сходится к некоторому $x$, то в единичном шаре $\|f\| \leqslant 1$ из неравенства
$$
|f(x_n) - f(x)| \leqslant \|f\| \ \|x_n - x\| \leqslant \|x_n - x\|
$$
вытекает существование для любого числа $\varepsilon > 0$ номера $N = N(\varepsilon)$ такого, что при $n \geqslant N$ справедливо неравенство
$$
|f(x_n) - f(x)| < \varepsilon.
$$
Но это и означает равномерную сходимость последовательности $\lbrace f(x_n) \rbrace$ в единичном шаре $\|f\| \leqslant 1$.\\

Теперь докажем достаточность. Пусть последовательность $\lbrace f(x_n) \rbrace$ сходится равномерно в единичном шаре $\|f\| \leqslant 1$, то есть для любого числа $\varepsilon > 0$ существует номер $N = N(\varepsilon)$ такой, что при $n \geqslant N$ справедливо неравенство
$$
|f(x_n) - f(x)| < \varepsilon.
$$
Отсюда следует, что при $n \geqslant N$
$$
\sup_{\|f\| \leqslant 1} |f(x_n) - f(x)| \leqslant \varepsilon.
$$
Применим следствие 1 из теоремы Хана-Банаха к элементу $x_0 = x_n - x$. В силу этого следствия существует функционал $f_0(x)$ такой, что $\|f_0\| = 1$, а $f_0 (x_n - x_0) = \|x_n - x_0\|$. Но тогда
$$
\|x_n - x_0\| = f_0(x_n - x_0) \leqslant \sup_{\|f\| \leqslant 1} |f(x_n) - f(x)| \leqslant \varepsilon,
$$
что и означает сильную сходимость последовательности  $\lbrace x_n \rbrace$ к $x$. $\quad \Box$\\

Рассмотрим линейные функционалы в различных нормированных пространствах.

\begin{enumerate}
\item
Пусть $X = \mathbb{R}_n$ --- конечномерное пространство, а $\lbrace e_i \rbrace_{i=1}^n$ --- ортонормированный базис в нём. Тогда любой элемент $x \in \mathbb{R}_n$ однозначно представим в виде 
$$
x = \sum_{k=1}^n \xi_k e_k,
$$
где $\xi_k$ --- некоторые коэфициенты. Следовательно, любой линейный функционал $f(x)$ в пространстве $\mathbb{R}_n$ однозначно представим в виде
$$
f(x) = \sum_{k=1}^n \xi_k f(e_k) = \sum_{k=1}^n \xi_k f_k,
$$
то есть $f(x)$ однозначно определяется числами $f(e_k) = f_k$, $k = \overline{1, n}$.

\item
Пусть $X = l_p$, $p > 1$, --- бесконечномерное пространство, а $\lbrace e_i \rbrace_{i=1}^\infty$ --- ортонормированный базис в нём. Тогда $l_p$ представляет собой пространство элементов $x$ таких, что
$$
x = \sum_{k=1}^\infty \xi_k e_k, \quad \sum_{k=1}^\infty |\xi_k |^p < +\infty.
$$
Следовательно, любой линейный функционал $f(x)$ в $l_p$ имеет вид
$$
f(x) = \sum_{k=1}^\infty \xi_k f(e_k) = \sum_{k=1}^\infty \xi_k c_k,
$$
то есть $f(x)$ однозначно определяется числами $f(e_k) = c_k$, $k = \overline{1, \infty}$. Выясним свойства чисел $c_k$. Для этого рассмотрим последовательность $\lbrace x_n \rbrace$ элементов
$$
x_n = \sum_{k=1}^\infty \xi_k^{(n)} e_k, \quad \text{ где } 
\xi_k^{(n)} = \begin{cases}
\mathrm{sgn} \ c_k  \cdot |c_k|^{q-1}, & k \leqslant n; \\
0, & k > n
\end{cases} \quad \frac 1 p + \frac 1 q = 1.
$$
Тогда
\begin{multline*}
\|f(x_n)\| = \sum_{k=1}^n |c_k|^q \leqslant \|f\| \|x_n\| = \\
= \|f\| \left( \sum_{k=1}^n |c_k|^{p(q-1)} \right)^{\frac 1 p} =
\|f\| \left( \sum_{k=1}^n |c_k|^q \right)^{\frac 1 p}.
\end{multline*}
Следовательно,
$$
\left( \sum_{k=1}^n |c_k|^q \right)^{\frac 1 q} \leqslant \|f\|,
$$
поэтому $\|c\|_q \leqslant \|f\|$.
С другой стороны, в силу неравенства Гёльдера
$$
|f(x)| = \left| \sum_{k=1}^\infty \xi_k c_k \right| \leqslant
\left( \sum_{k=1}^\infty |c_k|^q \right)^{\frac 1 q} \ \left( \sum_{k=1}^\infty |\xi_k|^p \right)^{\frac 1 p} = \|x\|_p \|c\|_q.
$$
Отсюда следует, что $\|f\| \leqslant \|c\|_q$. Следовательно, $\|f\| = \|c\|_q$, то есть $l_p^* = l_q$.

\item
Пусть $X = L_p(E)$, $p > 1$. Можно показать, что линейный функционал $f(x(t))$ в $L_p(E)$ будет иметь вид
$$
f(x(t)) = \int\limits_E x(t) g(t) dt,
$$
где $g(t) \in L_q(E)$ --- функция, однозначно определяемая по функционалу $f(x(t))$, причём $\|f\| = \|g\|_{L_q(E)}$, а $L_p^* = L_q$.

\item
Пусть $X = C[a, b]$. $a = x_0 < x_1 < \dots < x_{n-1} < x_n = b$.\\
$\sup\limits_{T}\sum\limits_{i=1}^{n}|f(x_i) - f(x_{i - 1})| < +\infty$ --- функция с ограниченным изменением.\\
$F(g(t)) = \int\limits_a^bg(t)df(t)$.

\end{enumerate}




%%%%%%%%%%%%%%%%%%%%%%%
%     Lecture 11      %
%%%%%%%%%%%%%%%%%%%%%%%






\section*{ \S 10. Гильбертовы пространства}

\textbf{Определение 1.} Множество $H$ называется \textit{гильбертовым пространством}, если
\begin{enumerate}
\item
$H$ --- линейное пространство над полем действительных или комплексных чисел.
\item
Каждой паре элементов $x, y \in H$ поставлено в соответствие число $(x, y)$, комплексное или действительное, называемое \textit{скалярным произведением} этих элементов и удовлетворяющее следующим аксиомам:
	\begin{itemize}
	\item[a)]
	$(x,y) = \overline{(y,x)} \quad \forall x, y \in H;$
	\item[б)]
	$(\lambda x,y) = \lambda (x,y) \quad \forall x, y \in H, \; \forall \lambda \in \mathbb{R} \text{ или } \mathbb{C};$
	\item[в)]
	$(x + y, z) = (x, z) + (y, z) \quad \forall x, y, z \in H;$
	\item[г)]
	$\forall x \in H \quad (x, x) \geqslant 0; \quad (x, x) = 0 \Leftrightarrow x = 0$.
	\end{itemize}
Норма на $H$ вводится через скалаярное произведение: число $\|x\| = \sqrt{(x, x)}$ будем называть \textit{нормой} элемента $x$.
\item
$H$ --- полное в метрике $\rho (x, y) = \|x - y\|$ пространство.
\item
$H$ --- бесконечномерное пространство, то есть для любого натурального числа $n$ в нём найдётся $n$ линейно независимых элементов.
\end{enumerate}

\textbf{Пример.} $l_p$ и $L_p(E)$ при $p = 2$ являются гильбертовыми пространствами, если ввести скалярное произведение следующим образом:
	\begin{enumerate}
	\item
	В $l_2$ для $x = (\xi_1, \xi_2, \dots), y = (\eta_1, \eta_2, \dots)$
	$$
	(x, y) = \sum_{i=1}^\infty \xi_i \overline{\eta_i}.
	$$
	\item
	В $L_2(E)$ для $x(t), y(t)$
	$$
	(x(t), y(t)) = \int\limits_E x(t) \overline{y(t)} dt.
	$$
	\end{enumerate}
Аксиомы скалярного произведения проверяются непосредственно.\\

\textbf{Утверждение 1.} Пусть $H$ --- гильбертово пространство, тогда для любых $x, y \in H$ верно неравенство
$$
|(x,y)| \leqslant \|x\| \|y\|.
$$

\textbf{Доказательство.} При $y = 0$ справедливость утверждения очевидна, поэтому далее в доказательстве положим $y \neq 0$. Для произвольного $\lambda$ верно
$$
0 \leqslant (x - \lambda y, x - \lambda y) = (x, x) - \overline{\lambda} (x, y) - \lambda (y, x) + |\lambda |^2 (y, y).
$$ 
Положив $\lambda = \dfrac{(x, y)}{(y, y)}$, получим 
$$
0 \leqslant (x, x) - \frac{|(x, y)|^2}{(y, y)},
$$
откуда следует требуемое неравенство. $\Box$\\

Известно, что линейное нормированное пространство является гильбертовым тогда и только тогда, когда в нем выполняется соотношение, называемое \textit{равенством параллелограмма}:
$$
\|x - y\|^2 + \|x + y\|^2 = 2\|x\|^2 + 2\|y\|^2.
$$
Доказательство следует из следующих соотношений:
$$
\|x+y\|^2 = (x + y, x + y) = (x, x) + (y, x) + (x, y) + (y, y) = \|x\|^2 + 2\Re (x, y) + \|y\|^2;
$$
$$
\|x-y\|^2 = (x - y, x - y) = (x, x) - (y, x) - (x, y) + (y, y) = \|x\|^2 - 2\Re (x, y) + \|y\|^2.
$$
Равенство параллелограмма является, таким образом, критерием гильбертовости пространства. Если в линейном нормированном пространстве не выполнено равенство параллелограмма, то в нем нельзя ввести скалярное произведение таким образом, чтобы выполнялись все четыре аксиомы гильбертова пространства.\\

\textbf{Пример.} Рассмотрим пространство $C \left[0; \dfrac{\pi}{2} \right]$, норма в котором определена следующим образом:
$$
\|x(t)\| = \max_{0 \leqslant t \leqslant \dfrac{\pi}{2}} |x(t)|.
$$
Функции $x(t) = \sin t$, $y(t) = \cos t$, очевидно, принадлежат этому пространству, причем $\|x\| = \|y\| = 1$. Рассмотрим функции
$$
x(t) + y(t) = \sin t + \cos t = \sqrt{2} \sin \left(t + \frac{\pi}{4}\right) \text{ и}
$$
$$
x(t) - y(t) = \sin t - \cos t = \sqrt{2} \sin \left(t - \frac{\pi}{4}\right).
$$
Очевидно, что $\|x(t) + y(t)\|^2 = 2$ и $\|x(t) - y(t)\|^2 = 1$. Но тогда равенство параллелограмма не выполнено, так как $4 \neq 3 \Rightarrow$ рассматриваемое пространство гильбертовым не является.\\

\textbf{Определение 2.} Множество $W$ называется \textit{выпуклым}, если
$$
\forall \alpha \in [0; 1] \quad x, y \in W \ \Rightarrow \ \alpha x + (1 - \alpha) y \in W.
$$

\textbf{Теорема 1.} В произвольном гильбертовом пространстве $H$ любое замкнутое выпуклое множество содержит единственный элемент с наименьшей нормой.\\

\textbf{Доказательство.} Пусть $W$ --- замкнутое выпуклое множество в гильбертовом пространстве $H$. Обозначим
$$
d = \inf_{x \in W} \|x\|.
$$
Тогда существует последовательность ${x_n}$ элементов $W$ таких, что 
$$
d = \lim_{n \to \infty}\|x_n\|,
$$
причем $\forall n \; \|x_n\| \geqslant d$ (это следует из определения точной нижней грани).\\

Так как $W$ --- выпуклое множество, $\forall n, m \; \dfrac{x_n + x_m}{2} \in W$ и 
$$
\left\Vert \frac{x_n + x_m}{2} \right\Vert \geqslant d \Rightarrow \|x_n + x_m\|^2 \geqslant 4d^2.
$$
С другой стороны, $2\|x_n\|^2 + 2\|x_m\|^2 \to 4d^2$ при $n, m \to \infty$, поэтому из последних двух соотношений и равенства параллелограмма для $x_n, x_m$
$$
\|x_n - x_m\|^2 = 2\|x_n\|^2 + 2\|x_m\|^2 - \|x_n + x_m\|^2
$$
следует, что $\|x_n - x_m\| \to 0$ при $n, m \to \infty$, то есть последовательность ${x_n}$ фундаментальна. Тогда в силу полноты $H$ и замкнутости $W$ получаем, что существует предел этой последовательности $x_0 \in W$:
$$
\lim_{n \to \infty} x_n = x_0,
$$
а так как $|\|x_n\| - \|x_0\|| \leqslant \|x_n - x_0\|$, то
$$
\lim_{n \to \infty} \|x_n\| = \|x_0\| = d.
$$
Таким образом, существование элемента с наименьшей нормой доказано. Докажем его единственность. Предположим, что существует другой элемент $x_1 \in W, \|x_1\| = d$. Тогда $\|x_0 + x_1\| \geqslant 4d^2$ (аналогично $\|x_n + x_m\| \geqslant 4d^2$) и 
$$
\|x_0 - x_1\| = 2\|x_0\|^2 + 2\|x_1\|^2 - \|x_0 + x_1\|^2 \leqslant 0,
$$
откуда следует $\|x_0 - x_1\| = 0 \Rightarrow x_0 = x_1$. Единственность доказана. $\Box$\\

\textbf{Определение.} Два элемента называются \textit{ортогональными} ($x \perp y$), если их скалярное произведение равно 0. \\
\textbf{Определение.} Элемент называется \textit{ортогональным множеству} ($x \perp L$), если он ортогонален всем его элементам. \\
\textbf{Свойство:} $x \perp L \Rightarrow x \perp \overline{L}$ (замыкание).\\

\textbf{Теорема 2 (Леви).} Пусть $H$ --- гильбертово пространство, $L$ --- подпространство в $H$ (замкнутое относительно сходимости по норме линейное многообразие). Тогда любой элемент $x \in H$ можно единственным образом представить в виде
$$
x = y + z, \quad y \in L, \; z \perp L,
$$
причем
$$
\|x - y\| = \min_{u \in L} \|x - u\|.
$$
\textbf{Доказательство. } Пусть $x \in H$ --- произвольный элемент пространства. Рассмотрим следующее множество элементов H:
$$
W = \{\, w = x - u \mid u \in L\,\}.
$$
Легко проверить, что $W$ является замкнутым выпуклым множеством. Следовательно, по теореме 1 в $W$ существует элемент с наименьшей нормой:
$$
\exists y \in L: \; \min_{u \in L} \|x - u\| = \|x - y\|.
$$
Положим $z = x - y$. Докажем, что $z \perp L$; это будет означать, что нужное представление найдено.
Рассмотрим множество элементов $z - \alpha v, \ v \in L$. Все такие элементы принадлежат множеству $W$; следовательно,
$$
\|z\|^2 \leqslant \|z - \alpha v\|^2 = (z, z) - \alpha (v, z) - \overline{\alpha} (z, v) + |\alpha |^2 (v, v).
$$
Можем считать, что $v \neq 0$ (иначе рассматриваемые элементы совпадают с z). Положив 
$$
\alpha = \frac{(z, v)}{v, v},
$$
получим
$$
- \frac{|(z, v)|^2}{(v, v)} \geqslant 0 \; \Rightarrow \; z \perp v \ \forall v \in L \; \Rightarrow \; z \perp L.
$$
Осталось доказать единственность полученного представления. Допустим, что существует два представления:
$$
x = y_1 + z_1 = y_2 + z_2, \; y_1, y_2 \in L, \; z_1, z_2 \perp L.
$$
Тогда рассмотрим элемент $v = y_1 - y_2 = z_2 - z_1$. С одной стороны, он принадлежит $L$, так как $L$ --- подпространство и $y_1 - y_2 \in L$. С другой стороны, он ортогонален любому вектору из $L$. Но тогда он ортогонален и самому себе $\Rightarrow v = 0$, откуда $y_1 = y_2$ и $z_2 = z_1$. Единственность доказана. $\Box$\\

\textbf{Определение 3.} \textit{Ортогональным дополнением} к подпространству $L$ гильбертова пространства $H$ называется множество всех элементов, ортогональных $L$:
$$
L^\perp = \{\,z \in H \mid z \perp L\,\}.
$$ 

\textbf{Утверждение 2.} $L = \left(L^\perp\right)^\perp$.\\
\textbf{Доказательство.} Пусть $x \in \left(L^\perp\right)^\perp$. По теореме 2 элемент $x$ представим в виде
$$
x = y + z,\ y \in L,\ z \in L^\perp.
$$
Тогда $(x, z) = 0$. Кроме того, $(x, z) = (y, z) + (z, z)$; следовательно, 
$$
(z, z) = 0 \Rightarrow z = 0 \Rightarrow x = y \in L,
$$
то есть $\left(L^\perp\right)^\perp \subseteq L$.
Обратно, пусть $x \in L$, тогда он представим в виде
$$
x = y + z,\ y \in \left(L^\perp\right)^\perp,\; z \in L^\perp.
$$
Проводя аналогичные рассуждения, приходим к выводу, что 
$$
x \in \left(L^\perp\right)^\perp \Rightarrow L \subseteq \left(L^\perp\right)^\perp.
$$
Следовательно, $L = \left(L^\perp\right)^\perp$. $\Box$\\

Из теоремы 2 вытекает как следствие\\
\textbf{Теорема 3.} Пусть $H$ --- произвольное гильбертово пространство, $L$ --- подпространство в $H$. Тогда $H$ представимо в виде суммы $L$ и его ортогонального дополнения:
$$
H = L \oplus L^\perp.
$$ \\

\textbf{Определение.} \textit{Ортопроектором} на подпространство $L$ называется оператор $Px = y$, где $x = y + z$, $y \in L$, $z \perp L$.\\

\textbf{Определение 4.} \textit{Ядром} линейного функционала $f(x)$ называется множество всех элементов, для которых $f(x) = 0$:
$$
\ker f = \{\,x \mid f(x) = 0\,\}.
$$

\textbf{Лемма 1.} $\dim (\ker f)^{\perp} = 1$, если $f \neq 0.$\\
\textbf{Доказательство.} Для двух произвольных элементов $x_1, x_2 \in (\ker f)^\perp$ рассмотрим элемент
$$
x = x_1 f(x_2) - x_2 f(x_1),
$$
являющийся нетривиальной линейной комбинацией рассматриваемых элементов $(\ker f)^\perp$.
Очевидно, $f(x) = 0$, тогда 
$$
x \in \ker f,\ x \perp x \Rightarrow x = 0.
$$
Таким образом, любые два элемента $x_1, x_2 \in (\ker f)^\perp$ являются линейно зависимыми; следовательно,
$$
\dim (\ker f)^{\perp} = 1. \quad \Box
$$

\textbf{Теорема 3 (теорема Рисса --- Фреше о представлении линейного функционала).} Любой линейный функционал в гильбертовом пространстве $H$ представим в виде
$$
f(x) = (x, y),\ y \in H,
$$
причем элемент $y$ однозначно определяется по $f$ и $\|f\| = \|y\|$.\\

\textbf{Доказательство.} По лемме 1 любой элемент $x \in (\ker f)^\perp$ представим в виде 
$$
(x, e)e,\ \|e\| = 1,\ e \in (\ker f)^\perp.
$$ 
Кроме того, $\ker f$ является подпространством в $H$. В самом деле, $\ker f$ является линейным многообразием в силу линейности и однородности $f$, замкнутость следует из непрерывности $f$. Мы рассматриваем ограниченные (а следовательно, и непрерывные) функционалы, иначе нельзя говорить о норме функционала.\\

Следовательно, по теореме 2 для любого элемента $x \in H$ существует единственное представление
$$
x = Px + (x, e) e,\ Px \in \ker f,\ \|e\| = 1,\ e \in (\ker f)^\perp.
$$
Но тогда
$$
f(x) = f(Px) + (x, e) f(e) = (x, \overline{f(e)} e).
$$
Обозначим $y = \overline{f(e)}e$. Покажем, что $y$ --- искомый элемент. Для любого $x \in H$ верно
$$
|f(x)| = |(x, y)| \leqslant \|x\| \|y\| => \|f\| \leqslant \|y\|.
$$
С другой стороны, верно 
$$
f(y) = (y, y) = \|y\|^2 \Rightarrow \frac{\|f(y)\|}{\|y\|} = \|y\|,
$$
откуда следует, что $\|f\| \geqslant \|y\|$. Из этого и предыдущего неравенств следует, что $\|f\| = \| y\|$.\\

Таким образом, существование требуемого представления получено. Докажем единственность. Допустим, что существует два элемента $y_1$ и $y_2$, удовлетворяющих требованиям теоремы. Тогда 
$$
\forall x \in H\ f(x, y_1) = f(x, y_2) \Rightarrow (x, y_1 - y_2) = 0.
$$
Это значит, что $(y_1 - y_2, y_1 - y_2) = 0$; следовательно, $y_1 = y_2$. Единственность доказана. $\quad \Box$\\

\textbf{Лемма 2.} Для того, чтобы линейное многообразие $M$ было всюду плотно в гильбертовом пространстве $H$, необходимо и достаточно, чтобы не существовало никакого элемента $H$, кроме нулевого, ортогонального $M$.\\
\textbf{Доказательство.} Необходимость. Пусть линейное многообразие $M$ имеет замыкание, совпадающее с $H$. Тогда для любого элемента $y \in H$ существует последовательность ${y_n}$ элементов $M$ таких, что 
$$
\rho (y_n, y) = \|y_n - y\| < \frac{1}{n}.
$$
Рассмотрим элемент $x \perp M, x \in H$. Тогда $x \perp y_n$ для любого номера $n$, и
$$
\forall n \; (y, x) = (y - y_n, x) + (y_n, x) = (y - y_n, x) \leqslant \|y - y_n\|\|x\| < \frac{\|x\|}{n}. 
$$
Следовательно, $(y, x) = 0$. Таким образом, мы доказали, что верно 
$$
x \perp M \Rightarrow x \perp \overline{M},
$$
но тогда $x \perp H$. Следовательно, $x \perp x$. Отсюда вытекает, что $x = 0$.\\

Достаточность. Допустим, что $\overline{M} \neq H$. Рассмотрим элемент $x \notin \overline{M}$. Так как $\overline{M}$ --- подпространство в $H$, по теореме 2 элемент $x$ представим в виде
$$
x = y + z, y \in \overline{M}, z \perp \overline{M},
$$
причем $z \neq 0$ (иначе бы $x \in \overline{M}$). Так как элемент $z$ ортогонален замыканию многообразия $M$, он ортогонален и самому многообразию $M$. Получили $z \neq 0, \ z \perp M$, что противоречит условию. Значит, $\overline{M} = H$. $\quad \Box$\\

\textbf{Определение 5.} Система $\{e_i\}$ в гильбертовом пространстве называется \textit{ортонормированной}, если
$$
(e_i, e_j) = \delta_{i,j} =\begin{cases}
1, & i = j, \\
0, & i \neq j.
\end{cases}
$$

Любая система линейно независимых элементов может быть ортогонализирована по Шмидту. Суть процесса ортогонализации заключается в следующем. Предположим, что есть система линейно независимых элементов $h_1, h_2, \dots \in H$. Тогда в качестве первого элемента положим
$$
e_1 = \frac{h_1}{\|h_1\|}.
$$
Построим $e_2$. Сначала будем искать вектор $g_2$, ортогональный $e_1$, в виде $g_2 = h_2 - c_{2_1} e_1$:
$$
(g_2, e_1) = 0 \Rightarrow c_{2_1} = (h_2, e_1).
$$
Тогда в качестве $e_2$ возьмем вектор $e_2 = \frac{g_2}{\|g_2\|}$. Предположим, что $k - 1$ элементов уже построено. Ищем $k$-й элемент в виде
$$
g_k = h_k - \sum_{i=1}^{k-1} c_{k_i} e_i.
$$
Очевидно, если положить $c_{k_i} = (h_k, e_i),\ \forall i \in 1 \dots k - 1$ , то элемент $g_k$ будет ортогонален всем $e_i, \forall i \in 1 \dots k - 1$. Следовательно, осталось взять в качестве $k$-го элемента 
$$
e_k = \frac{g_k}{\|g_k\|},
$$
и мы получим $\{e_1, \dots, e_k\}$ --- ортонормированную систему.\\

Продолжая этот процесс для всех номеров $k$, мы получим ортонормированную систему $e_1, e_2, \dots \in H$.\\

Выбирая взвешенные метрики в качестве скалярного произведения над пространством функций можно построить следующие ортонормированные системы из полинонов (т. е. $ g_i = t^{i-1}$):

\begin{itemize}
\item $a = -1, b = 1, \rho(t) = 1$ --- полиномы Лежандра.
\item $a = 0, b = \infty, \rho(t) = e^{-t}$ --- полиномы Эрмита.
\item $a = -\infty, b = \infty, \rho(t) = e^{-t^2}$ --- полиномы Чебышева.
\end{itemize}

%\textbf{Определение.} Система $\{e_i\}$ из $H$ называется \textit{замкнутой}, если
%$\forall \varepsilon~>~0$, $\forall f \in H$ существуют такие $m \in \mathbb{N}$ и $\{c_i\}$ из поля, что $\|f - \sum\limits_{i = %1}^{m}c_ie_i\| < \varepsilon$.\\

%\textbf{Определение.} Система $\{e_i\}$ из $H$ называется \textit{полной}, если не существует элемента, %ортогонального всем $e_i$, кроме 0. \\

\textbf{Определение 6.} Любая ортонормированная система $e_1, e_2, \dots$ в гильбертовом пространстве $H$ называется \textit{ортонормированным базисом}, если замыкание ее линейной оболочки совпадает со всем пространством:
$$
\overline{L(e_1, e_2, \dots )} = H.
$$

\textbf{Лемма.} Пусть $\{e_1, \ldots ,e_n\}$ --- ортонормированная система в гильбертовом пространстве $H$. Тогда
$L(e_1, \ldots, e_n)$ --- подпространство в $H$.\\
\textbf{Доказательство.} То, что $L = L(e_1, \ldots, e_n)$ является линейным многообразием, следует из определения линейной оболочки. Поэтому достаточно доказать, что $L$ будет замкнутым относительно сходимости по норме множеством.
Пусть $\{x_n\}$ --- фундаментальная последовательность элементов $L$. В силу полноты пространства эта последовательность сходится к некоторому элементу из $H$, причем этот предел определен однозначно. Нужно показать, что этот предел будет принадлежать $L$.\\

Для любого $\varepsilon > 0$ найдется номер $N(\varepsilon)$ такой, что 
$$
\|x_m - x_k\| < \varepsilon, \qquad \forall m, k \geqslant N(\varepsilon). 
$$ 
Но так как $x_m, x_k \in L$, они представимы в виде
$$
x_m = \sum_{i=1}^n {{\alpha}_m}_i e_i, \qquad x_k = \sum_{i=1}^n {{\alpha}_k}_i e_i;
$$
следовательно,
$$
\|x_m - x_k\| = \sum_{i=1}^n |{{\alpha}_m}_i - {{\alpha}_k}_i|^2 < \varepsilon.
$$
Отсюда следует, что для любого номера $1 \leqslant i \leqslant n$ последовательность $\{{{\alpha}_p}_i\}$ является фундаментальной; следовательно,
$$
\exists \quad \lim_{p \to \infty} {{\alpha}_p}_i = {\alpha}_i, \quad \forall i = 1,\ldots, n.
$$ 
Но тогда для произвольного $\varepsilon > 0$ существует такой номер $N(\varepsilon)$, что для любого номера $p \geqslant N(\varepsilon)$ справедливо неравенство
$$
|{{\alpha}_p}_i - {\alpha}_i| < \frac{\varepsilon}{n}
$$
одновременно для всех номеров $i$, $1 \leqslant i \leqslant n$ (достаточно взять максимальный из таких номеров по всем $1 \leqslant i \leqslant n$). Но тогда элемент $x = \sum\limits_{i=1}^n {\alpha}_i e_i$ и является искомым пределом последовательности $\{x_n\}$, так как
$$
\|x_m - x\| = \sum_{i=1}^n |{{\alpha}_m}_i - {\alpha}_i|^2 < \varepsilon,
$$
причем $x \in L = L(e_1, \ldots, e_n)$. Таким образом, $L(e_1, \ldots, e_n)$ является подпространством. $\quad \Box$\\

\textbf{Теорема 4.} В любом сепарабельном гильбертовом пространстве существует счетный ортонормированный базис.\\
\textbf{Доказательство.} Пусть $H$ --- сепарабельное гильбертово пространство. Тогда существует счетное всюду плотное множество элементов $g_1, g_2, \ldots \in H$:
$$
\overline{L(g_1, g_2, \ldots)} = H.
$$
В качестве первого элемента $e_1$ искомой системы положим 
$$
e_1 = \frac{g_1}{\|g_1\|}. 
$$
Соответствующая $e_1$ линейная оболочка
$$
L(e_1) = \{\, \alpha e_1 \mid \alpha \in P\,\}
$$
является подпространством в $H$, поэтому по теореме 2 
$$
g_{h_2} = h_2 + z_2, h_2 \in L(e_1), z_2 \perp L(e_1),
$$
где $g_{h_2}$ линейно независим с $L(e_1)$ (такой элемент существует в силу бесконечномерности пространства). Выберем в качестве второго элемента
$$
e_2 = \frac{z_2}{\|z_2\|}, L(e_1, e_2) = \{\alpha e_1 + \beta e_2\}.
$$
Аналогично, для выбора третьего элемента находим элемент $g_{h_3}$, линейно независимый с $L(e_1, e_2)$:
$$
g_{h_3} = h_3 + z_3, h_3 \in L(e_1, e_2), z_3 \perp L(e_1, e_2).
$$
В качестве $e_3$ выбираем элемент
$$
e_3 = \frac{z_3}{\|z_3\|},
$$
и так далее.\\

По построению ${e_1, e_2, \ldots }$ --- ортонормированная система. Она является базисом, поскольку, опять же, по построению
$$
\overline{L(e_1, e_2, \ldots )} = \overline{L(g_1, g_2, \ldots )} = H. \quad \Box
$$

\textbf{Определение 4.} Ортонормированная система в гильбертовом пространстве называется \textit{полной}, если не существует никакого элемента, кроме $0$, ортогонального всем элементам системы. Система называется \textit{замкнутой}, если замыкание ее линейной оболочки совпадает со всем пространством.\\

Таким образом, замкнутость равносильна полноте (в силу леммы 2).\\

\textbf{Лемма.} Пусть $L \subset H$ --- некоторое замкнутое подпространство, $\{e_i\}_{i=1}^\infty$ --- ортонормированная система в $H$. Тогда $\forall f \in L \; \forall \varepsilon > 0$ существует $n \in \mathbb{N}$, $\{\alpha_i\}$ из поля, такие что выполняется $\|f - \sum\limits_{i=1}^n\alpha_i e_i\|^2 < \varepsilon$. \\
\textbf{Доказательство.}
\begin{multline*}
\|f - \sum\limits_{i=1}^n\alpha_i e_i\|^2 = \|f\|^2  - \sum\limits_{i=1}^n\overline{\alpha_i} (f, e_i) - \sum\limits_{i=1}^n\alpha_i (e_i, f) + \sum\limits_{i,j=1}^n\overline{\alpha_i} \alpha_j (e_i, e_j) = \\
= \{c_i = (f, e_i)\}= \|f\|^2  - \sum\limits_{i=1}^n|c_i|^2 + \sum\limits_{i=1}^n|c_i - \alpha_i|^2
\end{multline*}

Очевидно, что минимальная норма разности достигается при $\alpha_i = c_i$. Эти числа называются коэффициентами разложения в ряд Фурье.

Отсюда немедленно получаем неравенство Бесселя: $\sum\limits_{i=1}^\infty|(f,e_i)|^2 \leqslant \|f\|^2$.

Использовав замкнутость $L$ превращаем неравенство Бесселя в равенство Парсеваля: $\sum\limits_{i=1}^\infty|(f,e_i)|^2 = \|f\|^2$. % Это равенство означает, что $f = \sum\limits_{i=1}^\infty(f,e_i)e_i$.

\textbf{Теорема 5.} Любые два сепарабельные гильбертовы пространства изометричны (существует биекция, сохраняющая расстояния) и изоморфны (существует биекция, сохраняющая линейные комбинации) между собой.\\
\textbf{Доказательство.} Пусть $H_1, H_2$ --- произвольные гильбертовы пространства. Достаточно доказать, что $H_1$ и $H_2$ изометричны и изоморфны $l_2$, тогда они будут изоморфны и изометричны друг другу. Следовательно, достаточно доказать, что любое гильбертово пространство H изоморфно и изометрично $l_2$.\\

Возьмем произвольный элемент $x \in H$. По теореме 4 в $H$ существует базис и верно соотношение
$$
\|x\|^2 = \sum_{k=1}^\infty c_k^2,
$$
где $c_k$ --- коэффициенты Фурье разложения $x$ по этому базису. Тогда в $l_2$ существует элемент 
$$
\tilde x = (c_1, c_2, \dots).
$$
Очевидно, что $\|\tilde x\| = \|x\|$.\\

Обратно, покажем, что любому элементу в $l_2$ соответствует элемент в $H$, причем их нормы совпадают. Рассмотрим элемент $\tilde x = (c_1, c_2, \dots) \in l_2$. Рассмотрим в пространстве $H$ последовательность
$$
z_n = \sum_{k=1}^n c_k e_k.
$$
Эта последовательность будет фундаментальной (так как $\sum\limits_{k=n}^{\infty} |c_k|^2 \to 0, \quad n \to \infty $). В силу полноты $H$ существует элемент $x \in H$, являющийся пределом этой последовательности:
$$
\lim_{n \to \infty} z_n = z.
$$
В силу непрерывности скалярного произведения, $(z, e_k) = c_k$ для любого номера k. Тогда $\|z\| = \|\tilde x\|$. $\quad \Box$\\

\textbf{Теорема 6 (теорема Рисса --- Фишера).} $l_2$ и $L_2$ над одним полем изометричны и изоморфны.\\






%%%%%%%%%%%%%%%%%%%%%%%
%     Lecture 12      %
%%%%%%%%%%%%%%%%%%%%%%%






\textbf{Теорема 7 (о слабой компактности в H).} Пусть $H$ --- сепарабельное гильбертово пространство, $\{x_n\}$ --- последовательность элементов $H$ такая, что $\|x_n\| < C,\quad C > 0$. Тогда существует подпоследовательность $\lbrace x_{n_k} \rbrace$, сходящаяся слабо. (также последовательность $\{x_n\}$ называется слабо компактной) \\
\textbf{Доказательство.} По теореме 4 в $H$ существует базис $\{e_k\}$. Рассмотрим последовательность $\{(x_n, e_1)\}$. Она ограничена, следовательно, из нее можно выделить сходящуюся подпоследовательность $\{({x_n}_1, e_1)\}$. Далее, можно выделить $\{{x_n}_2\}$ --- подпоследовательность $\{{x_n}_1\}$, такую, что последовательность $\{({x_n}_2, e_2)\}$ будет сходящейся. Продолжая этот процесс, получим, что для любого номера $m$ существует подпоследовательность $\{{x_n}_m\}$ такая, что последовательность $\{({x_n}_m, e_m)\}$ будет сходящейся.\\

Выберем следующую (диагональную) подпоследовательность:
$$
{\tilde x}_n = {x_n}_n.
$$
Для неё последовательность $\{({\tilde x}_n, e_k)\}$ будет сходящейся для любого базисного элемента $e_k$.\\

В силу замкнутости базиса $\{e_k\}$ для любого элемента $z \in H$
$$
\forall \varepsilon > 0 \quad \exists \Psi_\varepsilon = \sum_{k=1}^m \alpha_k e_k, \|z - \Psi_\varepsilon\| < \frac{\varepsilon}{4C}.
$$
Кроме того, для любого номера $k$ последовательность $\{(\tilde x_n, e_k)\}$ фундаментальна, так как она является сходящейся. Тогда найдется номер $N(\epsilon)$ такой, что 
$$
|(\tilde x_p - \tilde x_n, e_k)| < \frac{\varepsilon}{2 \max_{1 \leqslant i \leqslant m} \alpha_i m} 
$$
одновременно для всех $k, 1 \leqslant k \leqslant m$.  
Тогда
\begin{multline*}
|(\tilde x_m - \tilde x_n, z)| = \\
= |(\tilde x_m - \tilde x_n, \Psi_\varepsilon) + (\tilde x_m - \tilde x_n, z - \Psi_\varepsilon) \leqslant |(\tilde x_m - \tilde x_n, \Psi_\varepsilon)| + \|\tilde x_n - \tilde x_m\| \|z - \Psi_\varepsilon\| \leqslant \\
\leqslant |(\tilde x_m - \tilde x_n, \sum_{k=1}^m \alpha_k e_k)| + 2C \cdot \frac{\varepsilon}{4C} = \sum_{k=1}^m |\alpha_k| |(\tilde x_m - \tilde x_n, e_k)| + \frac{\varepsilon}{2} < \\
< \max_{1 \leqslant i \leqslant m} \alpha_i m \cdot \frac{\varepsilon}{2 \max\limits_{1 \leqslant i \leqslant m} \alpha_i m} = \varepsilon.
\end{multline*}
Таким образом, последовательность $(\tilde x_n, z)$ является фундаментальной для любого $z \in H$.\\

Рассмотрим функционал $f(z) = \lim\limits_{n\to\infty} (\tilde x_n, z)$. По теореме Рисса-Фреше (теорема 3) существует единственный элемент $x_0 \in H$ такой, что
$$
f(z) = (x_0, z) \quad \forall z \in H.
$$
Тогда $f(z) = (x_0, z) = \lim\limits_{n\to\infty} (\tilde x_n, z)$, то есть $x_0$ является слабым пределом последовательности $\{\tilde x_n\}$. $\quad \Box$\\



\section*{ \S 11. Сопряженный оператор.}

\textbf{Определение 1.} Пусть задан линейный оператор $A: X \to Y$, $X$ и $Y$ --- линейные нормированные пространства. Тогда для любого линейного функционала $\varphi (y) \in Y^*$ определен функционал
$$
f(x) = \varphi (Ax), f \in X^*.
$$
Таким образом, можно определить отображение
$$
A^*: Y^* \to X^*, \text{обозначается}\quad f = A^* \varphi,
$$
называемое \textit{сопряженным оператором}.\\

Если сопряженный оператор существует, то он является линейным:
$$
(\alpha A + \beta B)^* = \alpha A^* + \beta B^*.
$$

\textbf{Теорема 1.} Пусть $X$, $Y$ --- линейные нормированные пространства и задан линейный ограниченный оператор $A: X \to Y$. Тогда существует сопряженный оператор $A^*: Y^* \to X^*$, который также является линейным и ограниченным и $\|A^*\| = \|A\|$.\\
\textbf{Доказательство.} Существование и линейность следуют непосредственно из определения. Поэтому остается доказать, что сопряженный оператор ограничен и его норма совпадает с нормой оператора $A$.\\

С одной стороны,
$$
|f(x)| = |\varphi (Ax)| \leqslant \|\varphi\| \|Ax\| \leqslant \|\varphi\| \|A\| \|x\|.
$$
Cледовательно, для любого $x \neq 0$ верно неравенство 
$$
\frac{|f(x)|}{\|x\|} \leqslant \|\varphi\|\|A\|;
$$
но тогда
$$
\|f\| = \sup_{x \neq 0} \frac{|f(x)|}{\|x\|} \leqslant \|\varphi\|\|A\|
$$
и $\|A^*\varphi\| \leqslant \|\varphi\|\|A\|$. Таким образом, сопряженный оператор ограничен и $\|A^*\| \leqslant \|A\|$.\\

С другой стороны, по следствию из теоремы Хана-Банаха о продолжении линейного функционала для любого элемента $x_0 \in X$ существует линейный функционал ${\varphi}_0$ такой, что
$$
\|{\varphi}_0\| = 1 \quad \text{и} \quad {\varphi}_0 (A x_0) = \|A x_0\|. 
$$
Но тогда получаем, что
\begin{multline*}
\|Ax_0\| = \varphi_0 (Ax_0) = f_0(x_0) \leqslant \\
\leqslant \|f_0\| \|x_0\| = \|A^*{\varphi}_0\|\|x_0\| \leqslant \\
\leqslant \|A^*\| \|\varphi_0\| \|x_0\| = \|A^*\| \|x_0\|.
\end{multline*}
Отсюда вытекает, что $\|A\| \leqslant \|A^*\|$. Следовательно, $\|A^*\| = \|A\|$. $\quad \Box$\\

\textbf{Следствие.} Если операторы $A$ и $A^*$ являются сопряженными в гильбертовом пространстве $H$, то для любых двух элементов $x, y \in H$ верно $(Ax, y) = (x, A^*y).$\\
\textbf{Доказательство} следует из теоремы Рисса --- Фреше.\\

\textbf{Определение 2.} \textit{Образом} оператора $A: X \to Y$ называется множество
$$
\Im A = \{\,y \in Y \mid y = Ax\,\}.
$$
\textit{Ядром} оператора $A: X \to Y$ называется множество
$$
\ker A = \{\,x \in X \mid Ax = 0\,\}.
$$

\textbf{Теорема 2.} Пусть $H$ --- гильбертово пространство, $A$ --- оператор, действующий в $H$, $A^*$ --- сопряженный к $A$ оператор. Тогда
$$
H = \overline{\Im A} \oplus \ker A^*.
$$
\textbf{Замечание.} Очевидно, $\ker A = \overline{\ker A}$, однако образ оператора, вообще говоря, замкнутым не является. В доказательстве же существенно используется тот факт, что $\overline{\Im A}$ является подпространством. Поэтому в формулировке фигурирует именно замыкание образа.\\
\textbf{Доказательство.} Достаточно доказать, что
$$
\ker A^* = \overline{\Im A}^\perp,
$$
так как $\overline{\Im A}$ --- подпространство и для любого подпространства L $H = L \oplus L^\perp$.\\

Рассмотрим произвольный элемент $x \in \ker A^*$. Для него верно $A^*x = 0$. По следствию из теоремы 1 для любого элемента $y \in H$ справедливо $(Ay, x) = (y, A^*x)$. Значит, $x \in {\Im A}^\perp$, откуда следует, что $x\in \overline{\Im A}^\perp$ (доказательство проводится аналогично доказательству леммы 2 параграфа 10). Следовательно, $\ker A^* \subseteq \overline{\Im A}^\perp$.\\

Рассмотрим теперь произвольный элемент $x \in \overline{\Im A}^\perp$. Очевидно, $x \in {\Im A}^\perp$. Для любого элемента $y \in H$ справедливо $(Ay, x) = (y, A^*x)$. С другой стороны, $(Ay, x) = 0$, поэтому
$$
\forall y \in H, (y, A^*x) = 0 \Rightarrow (A^*x, A^*x) = 0 \Rightarrow \|A^*x\|^2 = 0 
\Rightarrow A^*x = 0.
$$
Получаем, что $x \in \ker A^*$. Следовательно, $\overline{\Im A}^\perp \subseteq \ker A^*$. Таким образом, $\overline{\Im A}^\perp = \ker A^*$. $\quad \Box$\\



\section*{ \S 12. Компактные и вполне непрерывные операторы.}

\textbf{Определение 1.} Множество $M$ линейного нормированного пространства $X$ называется \textit{компактным}, если любая последовательность элементов множества $M$ содержит подпоследовательность, сходящуюся к элементу из пространства $X$.\\

Множество $M$ называется \textit{предкомпактным} или \textit{относительно компактным}, если любая последовательность элементов $M$ содержит фундаментальную подпоследовательность.\\

Если пространство полное, то любая фундаментальная последовательность сходится к элементу этого пространства, тогда компактность и предкомпактность совпадают.

\textbf{Определение 2.} Линейный оператор, действующий из линейного нормированного пространства $X$ в линейное нормированное пространство $Y$, называется \textit{компактным}, если он любое ограниченное множество переводит в предкомпактное.\\

\textbf{Определение.} Линейный оператор, действующий из линейного нормированного пространства $X$ в линейное нормированное пространство $Y$, называется \textit{вполне непрерывным}, если он любую слабо сходящуюся последовательность переводит в сильно сходящуюся.\\

В банаховом пространтсве компактность равносильна предкомпактности.

%В банаховом пространстве компактный оператор является вполне непрерывным, и наоборот (то есть, в банаховом %пространстве компактность равносильна предкомпактности).\\
%
%Всякое компактное множество ограниченно, но не всякое ограниченное множество компактно.\\
%
%Ограниченный оператор компактное множество переводит в компактное.\\
%
%Если $A$, $B$ --- ограниченный и компактный операторы соответственно, то $AB$ и $BA$ являются компактными %операторами.\\
%
%\textbf{Критерий компактности} в пространствах $C(E)$ и $L_p(E), p \geqslant 1$ ($E$ --- замкнутое ограниченное %множество):\\
%
%Множество $M \subset C(E) \quad (M \subset L_p(E), p \geqslant 1)$ является компактным
%$$
%\Updownarrow
%$$
%\begin{enumerate}
%\item
%$M$ ограничено;
%\item
%$M$ равностепенно непрерывно.
%\end{enumerate}

%Соответственно, множество $M \subset C(E)$ ($E$ --- ограниченное и замкнутое) компактно
%$$
%\Updownarrow
%$$
%\begin{enumerate}
%\item
%$\exists C > 0: \qquad |x(t)| \leqslant C, \qquad \forall x(t) \in M;$
%\item
%$\forall x(t) \in M, \; \forall \varepsilon > 0$ существует $\delta(\varepsilon) > 0$ такое, что
%$$
%\forall t', t'' \in E, \quad |t' - t''| < \delta(\varepsilon) \quad \Rightarrow \quad |x(t') - x(t'')| < \varepsilon.
%$$
%\end{enumerate}
%
%Множество $M \subset L_p(E)$ ($E$ --- замкнутое и ограниченное, $p \geqslant 1$) компактно
%$$
%\Updownarrow
%$$
%\begin{enumerate}
%\item
%$\exists C > 0: \qquad {\|x(t)\|}_{L_p(E)} \leqslant C \qquad \mbox{для всех}\quad x(t) \in M;$
%\item
%$\forall x(t) \in M, \quad \forall \varepsilon > 0$ существует $\delta(\varepsilon) > 0$ такое, что
%$$
%\forall h, \quad |h| < \delta \quad \Rightarrow \quad {\|x(t + h) - x(t)\|}_{L_p(E)} < \varepsilon
%$$
%(функция $x(t)$ продолжена вне $E$ тождественным нулем).
%\end{enumerate}
%
%\textbf{Пример.} Рассмотрим следующий оператор, действующий в пространстве $C[a;b]$:
%$$
%Ax(t) = \int\limits_a^b k(s, t) x(s) ds,
%$$
%где $k(s,t) \in C[a, b] \times C[a, b]$ --- непрерывное ядро. Этот оператор является вполне непрерывным (на основании %критерия компактности непосредственно проверяется, что образ любого ограниченного подмножества $M \subset %C[a;b]$ является компактным).\\

\textbf{Лемма 1.} Если последовательность $\{x_n\}$ в банаховом пространстве $X$ является слабо сходящейся и компактной, то она является сильно сходящейся.\\
\textbf{Доказательство.} Пусть последовательность $\{x_n\}$ слабо сходится к $x \in X$. Предположим, она не является сильно сходящейся, тогда найдутся такие $\varepsilon > 0$ и подпоследовательность $\{x_{n_k}\}$, что для любого номера $n_k$ будет верно
$$
\|x_{n_k} - x\| \geqslant \varepsilon.
$$
Так как исходная последовательность компактна, из последовательности $\{x_{n_k}\}$ можно выделить сильно сходящуюся подпоследовательность
$$
x_{n_{k_l}} \to y.
$$
Из сильной сходимости вытекает слабая:
$$
x_{n_{k_l}} \xrightarrow{\text{сл.}} y.
$$
Но если слабый предел существует, то он определен однозначно; следовательно, $y = x$. Тогда, с одной стороны, 
$$
x_{n_{k_l}} \to x,
$$
а с другой стороны,
$$
\|x_{n_{k_l}} - x\| \geqslant \varepsilon, \quad \forall n_{k_l}.
$$
Получили противоречие. $\quad \Box$\\

\textbf{Лемма 2.} Пусть $X$ и $Y$ --- банаховы пространства, $A$ --- компактный оператор, действующий из $X$ в $Y$.
Тогда он любую слабо сходящуюся последовательность переводит в сильно сходящуюся последовательность (то есть является вполне непрерывным)%:
%$$
%x_n \xrightarrow{\text{сл.}} x \quad \Rightarrow \quad Ax_n \to Ax.
%$$
\\
\textbf{Доказательство.} Пусть последовательность $\{x_n\}$ элементов пространства $X$ слабо сходится к $x \in X$. Рассмотрим произвольный линейный функционал $\varphi (y) \in Y^*$:
$$
\varphi (Ax) = f(x), f \in X^*.
$$
По определению слабой сходимости
$$
\varphi (Ax_n) \to \varphi (Ax) \quad \Rightarrow \quad Ax_n \xrightarrow{\text{сл.}} Ax.
$$
Последовательность $\{Ax_n\}$ ограничена в силу слабой сходимости, поэтому оператор $A$ переводит ее в предкомпактную последовательность $\{Ax_n\}$. Тогда $\{Ax_n\}$ --- предкомпактная и сходится слабо в банаховом пространстве $\Rightarrow$ по лемме 1 она является сильно сходящейся. $\quad \Box$\\

Таким образом, компактный оператор является вполне непрерывным в банаховом пространстве.
Вполне непрервный оператор в банаховом пространстве переводит ограниченную последовательность в ограниченную последовательность.

\textbf{Лемма 3.} Пусть $A$ --- вполне непрерывный оператор, действующий из $H$ в $H$, где $H$ --- сепарабельное гильбертово пространство. Тогда сопряженный оператор $A^*$ также вполне непрерывен.\\
\textbf{Доказательство.} Пусть последовательность $\{x_n\}$ элементов пространства $H$ слабо сходится к $x \in H$. Тогда справедливы следующие соотношения:
\begin{multline*}
\|A^*(x_n - x)\|^2 = (A^*x_n - A^*x, A^*x_n - A^*x) = \\
= (x_n - x, AA^*(x_n - x)) \leqslant \|x_n - x\| \|AA^* (x_n - x)\|.
\end{multline*}
Последовательность $\{x_n\}$ сходится слабо, поэтому она ограничена. Следовательно, $\|x_n - x\| \leqslant C$ для некоторого $C > 0$.\\

Так как $A$ вполне непрерывен, он ограничен:
$$
\left\Vert A \frac{x}{\|x\|} \right\Vert \leqslant C \quad \Rightarrow \|Ax\| \leqslant C\|x\|, \quad \forall x \neq 0
$$
(образ ограниченного множества является компактом, следовательно, он является ограниченным множеством.)
Тогда $A^*$ является ограниченным (по теореме 1 параграфа 11), а $AA^*$ --- вполне непрерывным оператором (как произведение вполне непрерывного и ограниченного). Следовательно,
$$
\|AA^*(x_n - x)\| \to 0, \quad n \to \infty.
$$
Поэтому $\|A^*x_n - A^*x\| \to 0, n \to \infty$. Таким образом, оператор $A^*$ любую слабо сходящуюся последовательность переводит в сильно сходящуюся.\\

Для любого ограниченного множества $M \subset H$ рассмотрим его образ $M'$ при действии оператора $A^*$:
$$
M' = \{\,y \in H \mid y = A^*x, \quad x \in M\,\}.
$$
Произвольная последовательность элементов множества $M'$ имеет вид $\{A^*x_n\}$. Последовательность $\{x_n\}$ ограничена, и по теореме 7 параграфа 10 из нее можно выделить слабо сходящуюся подпоследовательность $\{x_{n_k}\}$, которую оператор $A^*$ переводит в $\{A^*x_{n_k}\}$ - сильно сходящуюся последовательность. Таким образом, из любой последовательности элементов $M'$ можно выделить сильно сходящуюся подпоследовательность, но тогда $M'$ --- компактное множество по определению, и $A^*$ является непрерывным оператором. Так как мы рассматриваем гильбертово пространство, $A^*$ является и вполне непрерывным. $\quad \Box$\\

%\textbf{Теорема 1.} Для того, чтобы линейный и ограниченный оператор $A$, действующий в гильбертовом %пространстве, был вполне непрерывным, необходимо и достаточно, чтобы он любую слабо сходящуюся %последовательность переводил в сильно сходящуюся последовательность.\\

\section*{ \S 13. Теория Фредгольма.}

Ранее (см. параграф 6) с помощью аппарата сжатых отображений доказывалось утверждение о существовании и единственности решения интегрального уравнения Фредгольма второго рода
$$
x(t) = \lambda \int\limits_E k(s,t) x(s) ds + f(t), \quad k(s, t) \in L_2((a;b) \times (a;b)), f(t) \in L_2(a;b)
$$
при достаточно малых значениях $\lambda$. Однако в случае
$$
x(t) - \int\limits_E k(s,t) x(s) ds = f(t)
$$
этот метод не дает результата. Нужен другой подход.\\

Пусть $A$ --- вполне непрерывный оператор, действующий в гильбертовом пространстве $H$. Будем искать решения уравнения 
$$
Lx = f, \quad L = I - A, \quad x \in H, f \in H.
$$
Очевидно, $L^* = I - A^*$.\\

\textbf{Лемма 1.} $\Im L = \overline{\Im L}$.\\
\textbf{Доказательство.} Пусть есть последовательность $\{y_n\}$, сходящаяся к некоторому элементу $y$ пространства:
$$
y_n \in \Im L, \quad y_n \to y, \quad n \to \infty.
$$
Надо доказать, что $y \in \Im L$. Заметим также, что $y_n = x_n - Ax_n = Lx_n$, и рассмотрим последовательность $\{x_n\}$.\\

Если существует подпоследовательность $\{x_{n_k}\}$ такая, что $x_{n_k} \in \ker L$ для любого $n_k$, то в силу замкнутости ядра
$$
y_{n_k} = Lx_{n_k} = 0 \to y = 0 \in \Im L.
$$
Поэтому мы можем считать, что, начиная с некоторого номера $N$, все $x_n$ ортогональны ядру $L$. Таким образом, достаточно рассмотреть последовательность $\{x_n\}$, в которой 
$$
\forall n \; x_n \perp \ker L.
$$
Докажем, что последовательность $\{x_n\}$ ограничена, то есть $\|x_n\| \leqslant C$ для некоторого положительного $C$. Допустим, что это неверно; тогда можно выделить подпоследовательность, стремящуюся к бесконечности: $\|x_n'\| \to \infty$.
В этом случае 
$$
\frac{\|y'_n\|}{\|x'_n\|} = \frac{\|x'_n - Ax'_n\|}{\|x'_n\|} \to 0 \text{ при } n \to \infty,
$$
поскольку все $\|y'_n\|$ ограничены (так как последовательность $\{y'_n\}$ сходится). \\

$A$ --- вполне непрерывный оператор, поэтому ограниченную последовательность $\dfrac{x'_n}{\|x'_n\|}$ он переводит в компактную $\dfrac{Ax'_n}{\|x'_n\|}$, в которой существует сходящаяся подпоследовательность $\dfrac{Ax''_n}{\|x''_n\|}$. Так как последовательности
$$
\frac{x''_n - Ax''_n}{\|x''_n\|}, \quad \frac{Ax''_n}{\|x''_n\|}
$$
сходятся, то будет сходиться и последовательность $\frac{x''_n}{\|x''_n\|}$:
$$
z_n = \frac{x''_n}{\|x''_n\|} \to z \text{ при } n \to \infty.
$$
При этом $\|z_n\| = 1$ для любого $n$, поэтому $\|z\| = 1$. По определению последовательности $\{z_n\}$
$$
Lz_n \to 0, \quad z_n \perp \ker L,
$$
но в силу непрерывности $L$ 
$$
Lz_n \to Lz, \quad n \to \infty.
$$
Поэтому $Lz = 0$ и $z \in \ker L$, но по построению $z \perp \ker L$ (так как $(\ker L)^\perp$ замкнуто), откуда $z = 0$. Получили противоречие с $\|z\| = 1$.\\

Полученное противоречие доказывает, что $\{x_n\}$ ограничена. Следовательно, $A$ переводит ее в компактную, из которой можно выделить сходящуюся подпоследовательность $A\tilde x_n$. Но последовательность
$$
\tilde y_n = \tilde x_n - A\tilde x_n
$$
также сходится как подпоследовательность сходящейся $\{y_n\}$. Поэтому последовательность $\tilde x_n$ сходится к некоторому элементу $x \in H$, но тогда в силу непрерывности оператора $A$ верно $A\tilde x_n \to Ax,\quad n \to \infty$. Получаем, что $y = x - Ax = Lx$. $\quad \Box$\\

\textbf{Лемма 2.} Пространство $H$ разложимо в прямую сумму
$$
H = \Im L \oplus \ker L^*.
$$
\textbf{Замечание.} Или, что то же самое,
$$
H = \Im L^* \oplus ker L.
$$
\textbf{Доказательство} следует из леммы 1 и теоремы 2 параграфа 11, в силу которой
$$
H = \overline{\Im A} \oplus \ker A^*
$$
для любого линейного ограниченного оператора $A$. $\quad \Box$\\

\textbf{Теорема 1 (первая теорема Фредгольма).} Для того, чтобы операторное уравнение $Lx = f$ было разрешимо, необходимо и достаточно, чтобы $f$ был ортогонален ядру сопряженного оператора:
$$
f \perp y, \; L^*y = 0 \; \forall y.
$$
\textbf{Доказательство.} Данное утверждение является прямым следствием леммы 2. $\quad\Box$\\






%%%%%%%%%%%%%%%%%%%%%%%
%     Lecture 13      %
%%%%%%%%%%%%%%%%%%%%%%%






Обозначим
$$
\Im L = H^1, \ldots, \Im L^k = H^k.
$$
Очевидно, выполнено соотношение $H = H^0 \supseteq H^1 \supseteq \ldots$.\\

\textbf{Лемма 3.} Существует такой номер $k$, что $H^k = H^{k-1}$.\\
\textbf{Доказательство.} Предположим, что такого номера $k$ не существует. По лемме 1 образ оператора $L$ --- замкнутое множество, поэтому оно является подпространством. Применим теорему Леви:
$$
H = H^1 \oplus (H^1)^\perp.
$$
Тогда существует элемент $x_1 \in (H^1)^\perp$ такой, что $\|x_1\| = 1$ (иначе $H = H^1, k = 1$). Применим теорему Леви для $H^1$:
$$
H^1 = H^2 \oplus (H^2)^\perp.
$$
Аналогично предыдущему случаю, существует элемент $x_2 \in (H^2)^\perp, \|x_2\| = 1$, причем $x_1 \perp x_2$. \\

Так как по предположению не существует номера $k$, при котором наступает стабилизация, продолжая этот процесс по всем $k, k = 1, 2, \ldots$, получим счетную ортонормированную систему.\\

Рассмотрим теперь два элемента полученной системы $x_l$ и $x_k$, где $l > k$. Справедливо равенство
$$
Lx_k - Lx_l = x_k - x_l - Ax_k + Ax_l, 
$$ 
из которого следует, что
$$
Ax_k - Ax_l = x_k - x_l - (Lx_k - Lx_l), \quad x_k \in (H^{k+1})^\perp, -x_l - (Lx_k - Lx_l) \in H^{k+1}.
$$
Это означает, что
$$
\|Ax_k - Ax_l\|^2 = \|x_k\|^2 + \|-x_l - (Lx_k - Lx_l)\|^2 \geqslant \|x_k\|^2 = 1,
$$
поэтому из последовательности $\{Ax_n\}$ нельзя выделить сходящуюся подпоследовательность. Но это противоречит тому, что $A$ является вполне непрерывным оператором. Полученное противоречие доказывает требуемое утверждение. $\quad\Box$\\

\textbf{Лемма 4.} Если ядро оператора $L$ не содержит отличного от нуля элемента, то его образ совпадает со всем пространством:
$$
\ker L = 0 \quad \Rightarrow \quad \Im L = H.
$$
\textbf{Доказательство.} Предположим обратное. Пусть $\Im L = H^1 \neq H$, тогда существует элемент $x_0 \in H \setminus H^1$. По лемме 3, существует номер $k$, такой что $H^k = H^{k+1}$. Следовательно,
$$
L^k x_0 \in H^k \quad \text{и} \quad \exists y \in H, \quad L^{k+1}y = L^k x_0.
$$
Тогда справедлива цепочка равенств
$$
L^k y = L^{k-1} x_0, \quad L^{k-1}y = L^{k-2} x_0, \quad \ldots, \quad Ly = x_0.
$$
Получили $x_0 \in H^1$, что противоречит изначальному предположению $x_0 \in H \setminus H^1$. $\quad\Box$\\

\textbf{Замечание.} Аналогично доказывается, что $\ker L^* = 0 \Rightarrow \Im L^* = H$.\\

\textbf{Лемма 5.} Если образ оператора $L$ совпадает со всем пространством, его ядро содержит только нулевой элемент:
$$
\Im L = H \quad \Rightarrow \quad \ker L = 0.
$$

\textbf{Доказательство.} По лемме 2 $H$ представимо в виде
$$
H = \Im L \oplus \ker L^*.
$$
Из этого разложения следует, что $\ker L^* = 0$. Cледовательно, по лемме 4 $\Im L^* = H$. С другой стороны, $H$ представимо в виде
$$
H = \Im L^* \oplus \ker L,
$$
откуда следует, что $\ker L = 0$.$\Box$\\

\textbf{Теорема 2 (вторая теорема Фредгольма, альтернатива Фредгольма).} Либо операторное уравнение $Lx = f$ разрешимо при любой правой части, либо соответствующее однородное уравнение $Lx = 0$ имеет нетривиальное решение.\\
\textbf{Доказательство.} Допустим, что уравнение $Lx = f$ разрешимо при любой правой части. Это означает, что $\Im L = H$, и по лемме 5 $\ker L = 0$. То есть однородное уравнение имеет только тривиальное решение.\\

Если же уравнение $Lx = f$ разрешимо не для всех $f$, то $\Im L \neq H$. Следовательно, $\ker L \neq 0$ и однородное уравнение имеет нетривиальное решение. $\quad\Box$\\

\textbf{Теорема 3 (третья теорема Фредгольма).} Размерности ядра оператора $L$ и сопряженного оператора $L^*$ конечны и равны:
$$
\dim \ker L = \dim \ker L^* < +\infty.
$$
\textbf{Доказательство.} Обозначим
$$
\dim \ker L = \nu, \quad \dim \ker L^* = \mu.
$$
Если $\nu = +\infty$, то в $\ker L$ можно выбрать счетный ортонормированный базиз $\{e_k\}$. Но тогда
$$
Ae_k = e_k \quad \text{для любого} \quad k \quad \text{и} \quad \|Ae_k - Ae_l\| = \|e_k - e_l\| = \sqrt{2},
$$
то есть из $\{Ae_k\}$ нельзя выбрать сходящейся подпоследовательности. Это противоречит компактности оператора $A$.\\

Аналогичным образом доказывается, что $\mu < +\infty$.\\

Предположим теперь, что $\mu > \nu$. Выберем в $\ker L$ ортонормированный базис $\{{\varphi}_k\}$, а в $\ker L^*$ --- ортонормированный базис $\{\psi_l\}$.\\

Рассмотрим следующий оператор:
$$
Sx = Lx - \sum_{k=1}^\nu (x, \varphi_k) \psi_k.
$$
Он будет являться вполне непрерывным. Если $Sx = 0$, то
$$
(Lx, \psi_l) = \sum_{k=1}^\nu (x, \varphi_k)(\psi_k, \psi_l) = (x, \varphi_l) = 0.
$$
Из последнего соотношения получаем, что все коэффициенты Фурье разложения элемента $x$ по базису $\{\varphi_k\}$ равны нулю, поэтому $Lx = 0$ и одновременно выполнено $x \perp \ker L$ и $x \in \ker L$. Следовательно, $x = 0$.\\

Таким образом, если $Sx = 0$, то $x = 0$. По второй теореме Фредгольма в этом случае существует решение уравнения
$$
Sy = Ly - \sum_{k=1}^\nu (x, \varphi_k) \psi_k = \psi_{\nu + 1}.
$$
Умножив скалярно обе части на $\psi_{\nu + 1}$, получим $0 = 1$ (так как $\ker L^* = (\Im L)^\perp$ и в силу того, что $\{\psi_k\}$ --- ортонормированная система). Следовательно, $\mu \leqslant \nu$.\\

Аналогичным образом рассматривая случай $\nu > \mu$, получим $\nu \leqslant \mu$. Следовательно, $\nu = \mu$. $\quad\Box$\\



\section*{ \S 14. Спектральная теория в бесконечномерном пространстве.}

Рассмотрим линейный ограниченный оператор $A$, действующий из линейного нормированного пространства $X$ в линейное нормированное пространство $Y$: $A \in L(X \to Y).$ Если $A \in L(X \to X)$, то для произвольного $\lambda \in \mathbb{C}$ можем определить оператор $\lambda I - A$.\\

\textbf{Определение 1.} Число $\lambda \in \mathbb{C}$ называется \textit{регулярным значением} оператора $A$, если оператор $B = (\lambda I - A)^{-1}$ определен на всем пространстве $X$ и является ограниченным:
% // !!! Или здесь "если выполнены следующие условия то лямбда --- регулярное значение A"? \\

\begin{itemize}
\item $\ker (\lambda I - A) = \{0\}$ (иначе не существует обратного)
\item $\Im (\lambda I - A) = X$ (чтобы действовал на всем пространстве)
\item $\|(\lambda I - A)^{-1}\| < +\infty$
\end{itemize}

Если $X$ --- банахово, то из ограниченности $A$ по теореме Банаха будет следовать ограниченность оператора $B$.\\

\textbf{Определение.} \textit{Резольвентой} называется $P(\lambda, A) = (\lambda I - A)^{-1}$.

\textbf{Определение 2.} Множество всех регулярных значений оператора $A$ называется \textit{резольвентным множеством} оператора $A$:
 %$$ // !!! шо такое D?
%\rho (A) = \{\,\lambda \in \mathbb{C} \mid D((\lambda I - A)^{-1}) = X\,\}.
%$$

\textbf{Определение 3.} \textit{Спектром} оператора $A$ называется множество всех комплексных чисел, не являющихся регулярными значениями $A$:
$$
\sigma (A) = \mathbb{C} \setminus \rho (A).
$$

%Если выполнены следующие условия:
%\begin{enumerate}
%\item
%$\ker (\lambda I - A) = 0$
%\item
%$\Im (\lambda I - A) = X,$
%\end{enumerate}
%то $\lambda$ будет регулярным значением $A$.\\

\textbf{Определение 4.} Если $\ker (\lambda I - A) \neq \{0\}$, то $\lambda$ называется \textit{собственным значением} оператора $A$.\\

Элемент $x \neq 0, \quad x \in \ker (\lambda I - A)$ называется в таком случае \textit{собственным элементом} оператора $A$.\\

В конечномерном пространстве спектр состоит из собственных значений.
Спектр в бесконечномерном пространстве может состоять не только из собственных значений. В доказательство можно привести следующий\\
\textbf{Пример.} Рассмотрим оператор $A(x(t)) = tx(t)$, действующий в пространстве $C[0;1]$. Для него верны следующие соотношения:
$$
(\lambda I - A) x(t) = (\lambda - t) x(t);
$$
$$
(\lambda I - A)^{-1} x(t) = \frac{x(t)}{\lambda - t}.
$$
Допустим, $x(t) \in \ker (\lambda I - A)$, тогда $(\lambda - t)x(t) \equiv 0$. Cледовательно, $x(t) \equiv 0$. Это означает, что у оператора $A$ нет ни одного собственного значения.\\

Посмотрим на спектр оператора $A$. Если оператор $(\lambda I - A)^{-1}$ существует, то он имеет вид, указанный выше. Поэтому при любых $\lambda \in \mathbb{R} \setminus [0;1]$ он определен на всем пространстве. Следовательно, спектром оператора является отрезок $\sigma (A) = [0; 1]$. При этом, как уже было показано выше, $A$ не имеет ни одного собственного значения. $\quad \Box$\\

По определению спектра, если резольвентное множество открыто, то спектр будет являться замкнутым множеством.\\

\textbf{Теорема 1 (Гильберта - Шмидта).} Пусть $A$ --- самосопряженный оператор, действующий в гильбертовом пространстве $H$. \\

Если $A$ является вполне непрерывным, то любой элемент $x \in \Im A$ представим в виде
$$
x = \sum_{\lambda_k \neq 0} (x, e_k) e_k,
$$ 
где $\{\lambda_k\}$ --- собственные значения оператора $A$, а $\{e_k\}$ --- соответствующие им собственные элементы.

\end{document}

